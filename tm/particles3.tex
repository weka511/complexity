% !TeX program = lualatex
\documentclass[]{article}

\usepackage{caption,subcaption,graphicx,float,url,amsmath,amssymb,amsthm,tocloft,cancel,thmtools,gensymb,braket,bm,tikz-feynman}
\usepackage[toc,nonumberlist]{glossaries}
\usepackage{glossaries-extra}
\usepackage[toc,page]{appendix}

\newcommand\numberthis{\addtocounter{equation}{1}\tag{\theequation}}

\newtheorem{thm}{Theorem}
\newtheorem{defn}[thm]{Definition}
\newtheorem{cor}[thm]{Corollary}
\newtheorem{lemma}[thm]{Lemma}
\graphicspath{{figs/}}
\widowpenalty10000
\clubpenalty10000
\setcounter{tocdepth}{2}
\tikzfeynmanset{compat=1.0.0}

%opening
\title{Theoretical Minimum\\Particle Physics 3\\Supersymmetry and Grand Unification}
\author{}

\begin{document}

\maketitle

\begin{abstract}
	These are my notes for the \emph{New Revolutions in Particle Physics 3} lectures from Leonard Susskind's \emph{Theoretical Minimum} series\cite{susskind2010supersymmetry}.
\end{abstract}

\tableofcontents
\listoffigures
\listoftables
\listoftheorems

\section{Renormalization  and dimensional analysis}

\subsection{Renormalization}

Most of the basic ideas of this quarter originated from questions having to do with renormalization of the standard model. Renormalization is a combination of two things:
\begin{itemize}
	\item Learning how to eliminate out of the description of physics things arising from distances that are so small that they are irrelevant to the questions you are asking
	\item learning how to think about how dimensional analysis tells you how to answer some of the difficult questions about field theory that have to do with distances much smaller than you might be interested in.
\end{itemize}

Examples of getting rid of things that are too small to be of interest.
\begin{itemize}
	\item Studying the nucleus at protons and neutrons instead of quarks. Use QCD to figure out properties of protons and neutrons and their forces. Nucleons move slowly, so we can ignore relativity. Now forget quarks.
	\item For atoms, can forget nucleons.
\end{itemize}

The result is a coarse-grained description that is less accurate byt more useful.

In field theory each possible wavelength represents a degree of freedom. In describing things at one length scale we don't want to deal with shorter scales. We find a way to sum up the shorter stuff and replace by effective new parameters.

Example: atoms and atomic forces.

Small usually goes with fast.

Consider two atoms, each including a cloud of electrons. We want to replace Figure \ref{fig:3-2-atoms} with two simple atom,es with forces between them. Electrons are very fast compared to nuclei, so we cold almost think of atom as a bowling ball with little flies.

\begin{figure}[H]
	\begin{center}
		\caption{Two atoms with clouds of electrons}\label{fig:3-2-atoms}
		\includegraphics[width=0.5\textwidth]{3-2-atoms}
	\end{center}
\end{figure}

First approximation: the atoms are so heavy that they don't move at all. Start by writing down Hamiltonian.

\begin{align*}
	E=& \underbrace{\frac{P_1^2}{M} + \frac{P_2^2}{M} + \frac{e^2}{R_{12}}}_\text{protons} + \underbrace{\sum \frac{q^2}{2m} + \frac{e^2}{r_{ij}}- \frac{e^2}{R_{1i}} - \frac{e^2}{R_{2i}}}_\text{electrons} \numberthis \label{eq:hamiltonian:H2}
\end{align*}

There are two time scales:
\begin{itemize}
	\item slow for nuclei, which are heavy
	\item fast for electrons, which are a blur.
\end{itemize}

To a first approximation fix the protons and treat the Hamiltonian as being for electrons. We solve the Schr\"odinger equation for the lowest energy state, $E_{electrons}(R_1,R_2)$. Then (\ref{eq:hamiltonian:H2}) becomes:


\begin{align*}
E=& \underbrace{\frac{P_1^2}{M} + \frac{P_2^2}{M} + \frac{e^2}{R_{12}}}_\text{protons} +\underbrace{ E_{electrons}(R_1,R_2)}_\text{Part of potential energy}
\end{align*}

We don't have to think about electrons again. All renormalization is based on this idea. 

\subsection{Dimensional Analysis}

In physics we have three scales, distance, time, and mass, and we can get rid of two of them by setting:
\begin{align*}
	c=&1\\
	\hslash=& 1
\end{align*} 
but we still need one dimension, which we will take to be a length.

\begin{align*}
	[m] =& [E]\\
	=& [P]\\
	[l] =& [t]\\
	=& \frac{1}{[m]}
\end{align*}

\subsection{A typical quantum field theory}

Renormalizing a scalar field $\Phi$. Our Lagrangian is:

\begin{align*}
	\mathcal{L} =& (\partial_\mu \Phi)^2 - V(\Phi)\\
	S =& \int d^4 x \mathcal{L} \text{ action - same units as $\hslash=1$}
\end{align*}

This means that $\mathcal{L}$ had dimensions $[l^{-4}]$

\begin{align*}
	[(\frac{\partial \Phi}{\partial x})^2]=l^{-4}\\
	[\Phi] = l^{-1}
\end{align*}
We imagine that the potential is something like:
\begin{align*}
	V(\Phi) =& \frac{m^2}{2} \Phi^2 + g \Phi^3 + \underbrace{\lambda}_\text{dimensionless} \Phi^4 \numberthis \label{eq:potential}
\end{align*}

Feynman diagrams are based on two things:
\begin{itemize}
	\item vertices, read off from V--Figure \ref{fig:3-1-feynman-vertices}
	\item propagators: represent motion from one point to another--Figure \ref{fig:3-1-feynman-propagator}.
\end{itemize}

\begin{figure}[H]
	\begin{center}
		\caption{Elements of a Feynman diagram}
		\begin{subfigure}[t]{0.45\textwidth}
			\caption{Feynman Vertices for (\ref{eq:potential}). Cross in left hand vertex indicates absorption/re-emission}\label{fig:3-1-feynman-vertices}
			\includegraphics[width=\textwidth]{3-1-feynman-vertices}
		\end{subfigure}
			\begin{subfigure}[t]{0.45\textwidth}
			\caption{Propagator $\braket{0|\Phi(y)\Phi(x)|0}$: amplitude that particle created at $x$ found at $y$. Dimension $[l^-2]$--$\frac{1}{|x-y|^2} if no mass$. If x and y close, propagator blows up--this is the source of all divergences in QFT.}\label{fig:3-1-feynman-propagator}
		\includegraphics[width=\textwidth]{3-1-feynman-propagator}
	\end{subfigure}
	\end{center}
\end{figure}

Let's start with renormalization of the mass. Notice that (\ref{eq:potential}) contains $m^2$ -- usually mass only appears as $\sqrt{m^2}$. So what is renormalization of $m^2$? In Figure \ref{fig:3-1-feynman-vertices}, do we care if the emission and absorption points are really the same? We aren't interested in anything that is finer than the resolution of our accelerators, so if we can find a process that mimics vertex,  even if blurred, any process that absorbs and re-emits can be counted as part of mass term--for example Figure \ref{fig:3-1-feynman-mimic}.


\begin{figure}[H]
	\caption{Mimicking the Vertex}
	\begin{subfigure}[t]{0.3\textwidth}
		\caption{Particle emitted and returns to same point: scale so small we don't notice}\label{fig:3-1-feynman-mimic}
		\includegraphics[width=\textwidth]{3-1-feynman-mimic}
	\end{subfigure}
	\begin{subfigure}[t]{0.3\textwidth}
		\caption{Amplitude following cut-off}\label{fig:3-1-cutoff}
		\includegraphics[width=\textwidth]{3-1-cutoff}
	\end{subfigure}
		\begin{subfigure}[t]{0.3\textwidth}
		\caption{Original mass term}\label{fig:3-1-original}
		\includegraphics[width=\textwidth]{3-1-original}
	\end{subfigure}
\end{figure}

We saw in Figure \ref{fig:3-1-feynman-propagator} that the propagator has amplitude:
\begin{align*}
	\braket{0|\Phi(y)\Phi(x)|0}=&\frac{1}{|x-y|^2}
\end{align*}

Let us introduce a cut-off, saying that we are not interested in scales smaller than $\delta$; we smear the point over $\delta$--Figure \ref{fig:3-1-cutoff}. Figure \ref{fig:3-1-original} shows the original mass term. If we include Figure \ref{fig:3-1-cutoff} we see that the effect of ignoring terms smaller than $\delta$ is to increase the effective mass: the amplitude becomes $\frac{m_0^2}{2}+\frac{1}{\delta^2}$.  \emph{This is what we'd see in the laboratory if our resolution isn't high enough to resolve lengths less than $\delta$.}  

But there are more Feynman diagrams!

\url{https://youtu.be/W6srShxBCrk?t=2705}
	
\section{Fermions and bosons}

Supersymmetry has to do with Fermions and Bosons, and their symmetries. We will Fermions and Bosons again, but, before that, we'll look at rotations.

Normally we are told that a rotation of $2\pi$ is equivalent to no rotation at all, but it is not true.

\begin{thm}[Rotation by $4\pi$ is equivalent to no rotation at all]
	Take a box, put a basketball in, and connect it to the walls with strings. A rotation by $2\pi$ tangles the strings; a rotation by $4\pi$ is equivalent to no rotation.
\end{thm}

Imagine that we have a particles, and we are interested in the spin $s$. What happens to the state when you rotate spin by $2\pi$.
\begin{itemize}
	\item apply string magnetic field and rotate;
	\item use field to precess particle.
\end{itemize}

Normally we'd expect it to come back to itself, but we know that $2\pi$ is not the identity. Perhaps it multiplies by a number. If we multiply by a phase, $\xi$, we won't be able to detect it. Multiplication by $4\pi$ squares phase, since it does nothing, so $\xi^2=1$, so $\xi=\pm1$.
\begin{align*}
	\ket{s} \rightarrow& +\ket{s} \text{, or}\\
	\ket{s} \rightarrow& -\ket{s}
\end{align*}

Can we detect this experimentally? It was long thought that we could not. If we take a family of electrons, all prepared identically, stick half into a magnetic field and rotate $2\pi$, we can't tell the two lots apart.

We can, however, do a 2 slit experiment\cite{aharonov1967observability}--Figue \ref{fig:particles3-2-2slit}.

\begin{figure}[H]
	\begin{center}
		\caption[A 2 slit experiment to detect rotation by $2\pi$]{A 2 slit experiment to detect rotation by $2\pi$. There is a magnetic field behind each slit, which traps electron temporarily. While an electron is trapped we rotate one field or the other, which can be detected in the interference pattern.}\label{fig:particles3-2-2slit}
		\includegraphics[width=0.7\textwidth]{particles3-2-2slit}
	\end{center}
\end{figure}

\section{Propagators and renormalization of mass}

TBP

\section{Symmetry and Grassmann numbers}

TBP

\section{A first supersymmetric model}

TBP

\section{Supersymmetry building blocks}

TBP

\section{Lagrangians that preserve supersymmetry}

TBP

\section{Generalizing supersymmetry to 3+1 spacetime, and QFT}

TBP

\section{Supersymmetry breaking and an introduction to grand unified theories}

TBP

\section{GUTs, the SU(5) representation, proton decay}

TBP


\bibliographystyle{unsrt}
\addcontentsline{toc}{section}{Bibliography}
\raggedright
\bibliography{tm}

\end{document}
