\documentclass[]{article}
\usepackage{caption,subcaption,graphicx,float,url,amsmath,amssymb,amsthm,tocloft,cancel,thmtools,gensymb,braket}
\usepackage[toc,nonumberlist]{glossaries}
\usepackage{glossaries-extra}
\newcommand\numberthis{\addtocounter{equation}{1}\tag{\theequation}}

\newtheorem{thm}{Theorem}
\newtheorem{defn}[thm]{Definition}
\newtheorem{cor}[thm]{Corollary}
\newtheorem{lemma}[thm]{Lemma}
\graphicspath{{figs/}}
\widowpenalty10000
\clubpenalty10000
\setcounter{tocdepth}{2}

%opening
\title{Theoretical Minimum\\Particle Physics 1\\Basic Concepts}
\author{Simon Crase(compiler)}

\begin{document}

\maketitle

\begin{abstract}
These are my notes for \emph{New Revolutions in Particle Physics 1} lectures from Leonard Susskind's Theoretical Minimum series.
\end{abstract}

\tableofcontents
\listoffigures
\listoftables
\listoftheorems

\section{Particles and Light}

This lecture contains  facts from other courses, which will be used on later lectures.

Particle physics is about the question: is matter discrete? If so we will call the smallest things "particles". If matter forms a continuum, we have fields.

Which is correct? Both and neither.

First evidence for atoms came from chemistry. John Dalton looked at mass mole: each substance was an integer multiple of mass of a mole of hydrogen; it suggested that there are building blocks. We now know that mass of molecule is mass of protons, electrons (very small), and neutrons, which have nearly the same mass as protons. 
 
Figure \ref{fig:em:wave} illustrates an electromagnetic wave. We need the concepts of wavelength, $\lambda$, and period, $T$.

We have
\begin{align*}
\frac{\lambda}{T}=&c\\
f=&\frac{1}{T} \text{, so}\\
\lambda f =& c\text{. Physicists tend to measure frequency in radians per second:}\\
\omega =& 2 \pi f \text{, so} \\
\omega =& \frac{2 \pi c}{\lambda} \numberthis \label{eq:omega:lambda}
\end{align*}

\begin{figure}[H]
	\caption{An electromagnetic wave}\label{fig:em:wave}  
	\includegraphics[width=0.9\textwidth]{Wavelength}
\end{figure}

Wave/particle duality.

A photon has energy:
\begin{align*}
E=&\hslash \omega \text{. Energy of single photon.} \numberthis\label{eq:E_omega}\\
E_{ray}=&n \hslash \omega \text{. Energy of ray.}
\end{align*}
 

In modern usage, mass is what used to be called rest mass. $E = m c^2$ only for stationary object.

In particle physics we choose units to set $c=1$ and $\hslash=1$. In this lecture $c$ and $\hslash$ will appear explicitly only when Prof. Susskind wants to show the magnitude of some quantity.

Energy depends on the motion of the object as seen by the observer: it isn't a universal thing that everyone will agree on.

Light has energy and momentum (not mass): 

\begin{align*}
\left|P\right|=&\frac{E}{c} \\
=&\frac{\hslash \omega}{c} \text{, from (\ref{eq:E_omega})}\\
=& \frac{2 \pi \hslash}{\lambda} \text{, from (\ref{eq:omega:lambda}). c.f. Harmonic Oscillator.}\numberthis\label{eq:P:lambda}
\end{align*}

This is why we need larger and larger instruments to observe smaller particles!
To see really small things, we need particles with high energy and momentum.


\section{Quantum field theory}

\subsection{Mathematical Preliminaries}

Consider a classical wave $A \sin(kx)$. Energy is $A^2$, but it is also proportional to number of photons, $n$: $E \propto n \hslash \omega$, $A \propto \sqrt(n)$.

Generally, equations containing $c$ apply for photons only. For non-relativistic particles:
\begin{align*}
E=&\frac{p^2}{2m}\\
f =& \frac{h}{2 m \lambda^2} \text{. c.f. Schr\"odinger equation!}
\end{align*}

\begin{itemize}
	\item Phase Velocity: velocity of wave packet;
	\item Group velocity: velocity of troughs and peaks--identified with particle velocity.
\end{itemize}

For light waves, phase and group velocities the same, but this isn't true for most particles. Phase velocity can exceed $c$, but can't transmit information.

Space is infinite. But physics is the art of getting problems into a shape that computers can solve, so we need to remove infinities.

Let's look at 1 dimensional waves. We can restrict to fixed length $L$: reflecting boundary condition at ends. But this violates conservation of momentum. Instead use a topological "circle" circumference $L$--periodic boundary conditions. As a consequence momentum is quantized: 

\begin{align*}
\lambda =& \frac{L}{N}\\
P =& \frac{h}{\lambda}\\
=& \frac{h N}{L}
\end{align*}

Now make into a real circle, radius $R$, and redefine $L$ to be \emph{angular momentum}.

\begin{align*}
P =& \frac{h N}{2 \pi R}\\
L =& N \frac{h}{2 \pi}\\
=& N \hslash \text{ independent of $R$.}
\end{align*}

\subsection{Quantum field theory}

Start with harmonic oscillator: a wave is a collection of harmonic oscillators. Energy is quantized: 0, $\hslash \omega$, $2\hslash \omega$...

Introduce operators:

\begin{align*}
a^+ \ket{n} =& \sqrt{n+1} \ket{n+1} \text{, creation operator} \numberthis \label{eq:creation}\\
a^- \ket{n} =& \sqrt{n} \ket{n-1} \text{, annihilation operator} \numberthis \label{eq:annihilation} \\
a^+ a^- \ket{n} =& n \ket{n} \text{, or}\\
a^+ a^-  =& n \text{, similarly}\\
a^- a^+  =& n+1\\
[a^-, a^+] =& 1
\end{align*}

Return to world on a circle--$\omega_N$--equivalent to oscillator. State is $\ket{n_1,n_2, n_3,...}$--occupation numbers. 
\begin{defn}[quantum field]
	A quantum field is a collection of harmonic oscillators, togther with annihilation operators and creation operators.
\end{defn}

\section{Quantum fields and particles}



\begin{align*}
\text{Wave} =& e^{i k x} \text{, $k$ is wave number}\\
P=&\hslash k\\
n(k) =& \text{ occupation number.}
\end{align*}

We represent the state of a 1D circular system by occupation numbers, $\ket{...n(-1), n(0), n(1), n(2)...}$, and introduce operators $a^+(k)$, $a^-(k)$. 
\begin{align*}
a^+(1)\ket{...n(-1), n(0), n(1), n(2)...}=& \sqrt{n(1)+1}\ket{...n(-1), n(0), n(1)+1, n(2)...}
\end{align*}
$a^+(k)$, $a^-(k)$ are quantum mechanical versions of the Fourier coefficients of a field $\Psi$.

Here is a classical wave:
\begin{align*}
\Psi(x) =& \sum_k \alpha(k) e^{ikx}\\
\Psi^*(x) =& \sum_k \alpha^*(k) e^{-ikx}
\end{align*}

We quantize as shown in (\ref{eq:q:1}) and (\ref{eq:q:2}) to produce a quantum field:
\begin{align*}
\alpha(k) \rightarrow& a^-(k) \numberthis \label{eq:q:1}\\
\alpha^*(k) \rightarrow& a^+(k) \numberthis \label{eq:q:2}\\
\Psi(x) =& \sum_k a^-(k) e^{ikx} \numberthis \label{eq:Psi}\\
\Psi^\dagger(x) =& \sum_k a^+(k) e^{-ikx} \numberthis \label{eq:Psi:dagger}
\end{align*}

We need to justify (\ref{eq:q:1}) and (\ref{eq:q:2}) in some appropriate limit.

How can we describe scattering? Imagine scattering so a particle with momentum $k_7$ becomes a particle with momentum $k_9$: $a^+(k_9)a^-(k_7)\ket{0,0,..1,0,0}$.

What if laws of physics allows number of particles to change? $a^+(k_{16})a^+(k_9)a^-(k_7)\ket{0,0,..1,0,0}$. We can't do this in regular quantum mechanics.

What does $\Psi^\dagger$ do? Start with the vacuum $\ket{0}$.

\begin{align*}
\Psi^\dagger(x)\ket{0} =& \sum_k a^+(k) e^{-ikx} \ket{0} \text{ from (\ref{eq:Psi:dagger})}\\
=& \sum_k e^{-ikx} \underbrace{a^+(k) \ket{0}}_\text{One particle state with momentum $k$} \\
=& \sum_k e^{-ikx} \ket{k} \text{. Superposition gives particle at definite position.}
\end{align*}

If we make the circle larger and larger, $\sum \rightarrow \int$.

What does $\Psi$ do? It annihilates a particle at $x$.

Locality: when something happens, it happens at one spot. Figure \ref{fig:ex:locality} is an example of one particle being replaced by two.

\begin{figure}[H]
	\caption{Example of locality: $\Psi^\dagger(x)\Psi^\dagger(x)\Psi(x)\ket{...}$}\label{fig:ex:locality}
	\includegraphics[width=0.8\textwidth]{split-particle}
\end{figure}

Start with state $\ket{0,...\underbrace{1}_\text{particle with momentum $k$},0,000}$.

We assume that $k_i$ denotes the index of the only momentum with a non-zero index.
\begin{align*}
\Psi^\dagger(x)\Psi^\dagger(x)\Psi(x)\ket{0,...1,0,000}=&\sum_m a^+(m) e^{-imx} \sum_l a^+(l) e^{-ilx} \sum_k a^-(k) e^{ikx} \ket{...}\\
=&\sum_m a^+(m) e^{-imx} \sum_l a^+(l) e^{-ilx} a^-(k_i) e^{ik_ix} \ket{...}\\
=&\sum_m a^+(m) e^{-imx} \sum_l a^+(l) e^{-ilx}  e^{ik_ix} \ket{0}\\
=&\sum_{l,m}  e^{-imx}  e^{-ilx}  e^{ik_ix} \underbrace{\ket{001001...}}_\text{$l$ and $m$ are indices of $1s$.}\\
\triangleq &\sum_{l,m}   e^{i(i_k-l-m)x} \ket{l,m}
\end{align*}

This isn't true if $l=m$! We have two application of $a^+(m)$, so vacuum becomes $\sqrt{2}\ket{...2...}$. Probability of two particles coming out in same state is twice probability of them being in different states.

Ignore conservation laws!

Consider generating photon at a position where there is a pre-existing photon with momentum $l$--Figure \ref{fig:simple:decay}.
\begin{figure}[H]
	\caption[Generating photon  where there is a pre-existing photon]{Generating photon at a position where there is a pre-existing photon with momentum $l$.}\label{fig:simple:decay}
	\includegraphics[width=0.8\textwidth]{decay}
\end{figure}

\begin{align*}
\Psi(x)\ket{l} =& \sum_k a^+(k) e^{-ikx} \ket{l}\\
=& \sum_{k\ne l} e^{-ikx} \ket{k,l} + \sqrt{2} e^{-ilx} \ket{l,l}
\end{align*}

Stimulated emission: higher probability of generating particles that are already there (c.f. spontaneous emission). This is true for Bosons, but not Fermions.
 
 Laws for Bosons
 \begin{align*}
 [a^+(k),a^+(l)] =& 0\\
 [a^-(k),a^-(l)] =& 0\\
 [a^-(k),a^+(l)] =& \delta_{k,j}\\
 \Psi(x) =& \sum_k a^-(k) e^{ikx} \\
 \Psi^\dagger(x) =& \sum_k a^+(k) e^{-ikx}
 \end{align*}
 
 \begin{align*}
 \frac{1}{L}\int dx \Psi^\dagger(x) \Psi(x) =& \int  \sum_k a^+(k) e^{-ikx} \sum_l a^-(l) e^{ilx}\\
 =&  \sum_k \underbrace{a^+(k) a^-(k)}_\text{Occupation number}\\
 =& N \text{, total number of particles.}\\
 \Psi^\dagger(x) \Psi(x) =& \text{ density of particles at $x$}.
 \end{align*}

\url{https://youtu.be/UgPBUJRrYz8?t=5689}


\section{More quantum field theory}

\subsection{Mathematical Preliminaries}

\subsubsection{Dirac Delta function}

\begin{align*}
\text{For } k =& \frac{2 \pi n}{L}\\
\int_{-\frac{L}{2}}^{\frac{L}{2}} e^{ikx} dx =& L \text{, if $k =0$}\\
=& 0 \text{, otherwise}\\
\rightarrow& 2 \pi \delta(k) \text{ as $k \rightarrow \inf$}
\end{align*}

We will use ket vectors for initial states, bra for final. This will help with bookkeeping.

\subsubsection{Creation and annihilation operators operating on bra vectors}

\begin{align*}
\braket{n|m} =& \delta_{n,m}\numberthis \label{eq:orthogonal} \text{. We want the following}\\
\braket{n|(a^+|m)} =& \braket{(n|a^+)|m} \text{ from (\ref{eq:creation})} \numberthis \label{eq:consistency}\\
\braket{n|(a^+|m)} =& \sqrt{m+1} \braket{n|m+1}\\
=& \sqrt{m+1} \delta_{n,m+1} \text{ from (\ref{eq:orthogonal})} \text{, so we define}\\
\bra{n} a^+ \triangleq & \sqrt{n} \bra{n-1} \text{ to satisfy (\ref{eq:consistency}), and} \numberthis \label{eq:bra:create}\\
\bra{n} a^= \triangleq & \sqrt{n+a} \bra{n-1} \numberthis \label{eq:bra:annihilate}
\end{align*}
Example: calculate an expression two different ways.
\begin{align*}
\braket{n|(a^+a^-|n)} =& \sqrt{n}\braket{n|(a^+|n-1)}\\
=& n \braket{n|n} \text{ from (\ref{eq:bra:create})}\\
=& n\\
\braket{(n|a^+a^-)|n} =& \sqrt{n} \braket{(n-1|a^-)|n}\\
=& n \braket{n|n} \text{ from (\ref{eq:bra:annihilate})}
\end{align*}

\subsection{Quantum Fields}

The only way we can study particles is to collide them; quantum field theory is the mathematical tool.

We will set $\hslash=1$, so $P=k$, and we will take $X$ and $P$ to be 3 dimensional.
\begin{align*}
\Psi^\dagger(X)(X,t) \triangleq& \sum_k a^+(k) e^{-ikX} e^{i \omega(k) t} \text{, but}\\
\omega =& E \text{ and}\\
E =& \frac{P^2}{2m} \text{, whence}\\
\omega =& \frac{k^2}{2m} \numberthis \label{eq:omega:k}\\
\Psi \triangleq& \sum_k a^-(k) e^{ikX} e^{- i \omega(k) t}
\end{align*}
We want to determine the differential equation satisfied by $\Psi^\dagger(X)$.

\begin{align*}
\dot{\Psi^\dagger(X)} =&  \sum_k i \omega(k) a^+(k) e^{-ikX} e^{i \omega(k) t}\\
\frac{\partial^2 \Psi^\dagger(X)}{\partial X^2} =& \sum_k -k^2 a^+(k) e^{-ikX} e^{i \omega(k) t} \numberthis \label{eq:omega:dot}\\
 =& - \sum_k 2m \omega a^+(k) e^{-ikX} e^{i \omega(k) t} \text{ from (\ref{eq:omega:k})}\\
 \frac{1}{2m} \frac{\partial^2 \Psi^\dagger(X)}{\partial X^2} =& - \sum_k \omega a^+(k) e^{-ikX} e^{i \omega(k) t}\\
  i \dot{\Psi^\dagger(X)} =&  \frac{1}{2m} \frac{\partial^2 \Psi^\dagger(X)}{\partial X^2} \text{ From (\ref{eq:omega:dot})--Schr\"odinger for field!}
\end{align*}

Consider particle scattering from heavy target; conserves energy, but not momentum--Figure \ref{fig:particle:heavy:target}. Assume scattering probability independent of time.

\begin{figure}[H]
	\caption{Particle being absorbed and then re-emitted}\label{fig:particle:heavy:target}
	\includegraphics[width=0.8\textwidth]{particle-heavy-target}
\end{figure}

We want the probability of emitting one particle in final state (Introduce coupling constant $g$)--$\big| \bra{k_f} g \int dt \Psi^\dagger(0,t)\Psi(0,t)\ket{k_i} \big|^2$

\begin{align*}
&\bra{k_f} g \int dt \sum_l \underbrace{a^+(l) e^{i \omega_l t}}_\text{contributes iff $l==k_f$} \sum_k \underbrace{a^-(k) e^{-i \omega_k t} \ket{k_i}}_\text{contributes iff $k==k_i$}\\
=&\bra{k_f} g \int dt a^+(k_f) e^{i \omega_{k_f} t}  a^-(k_i) e^{-i \omega_{k_i} t} \ket{k_i}\\
=& g \int dt e^{i(\omega_{k_f}-\omega_{k_i})}\cancel{<0|0>}\\
=& 2 \pi g \delta(\omega_{k_f}-\omega_{k_i}) \text{--conservation of energy.} \numberthis \label{eq:conservation:energy}
\end{align*}

Conservation of energy works only because of time symmetry!

Probability $\propto g^2$.

Prof. Susskind has shown:
\begin{itemize}
	\item Definition of coupling constant;
	\item Integration over time is the thing that ensures energy conservation;
	\item Scattering amplitude.
\end{itemize}

NB.\begin{itemize}
	\item  If X not zero (say $X_s$), then amplitude gets a phase of $e^{i(k_f-k_i)S_s}$, which doesn't affect probability.
	\item if $g$ is a function of time, it generally means that we are ignoring something.
\end{itemize}

To reiterate: $\Psi$ and $\Psi^\dagger$ are the tools that let us discuss particle interactions.

What if we are dealing with an electron? Each particle has its own $\Psi$. What if one electron goes in but two come out-- $\Psi^\dagger(0,t)\Psi^\dagger(0,t)\Psi(0,t)$? Or two in, two out--$\Psi^\dagger(0,t)\Psi^\dagger(0,t)\Psi(0,t)\Psi(0,t)$? Which are allowed? One rule is that thee must be the same number of $\Psi^\dagger$s and $\Psi$s. Imagine multiplying by a phase:

\begin{align*}
\Psi \rightarrow & e^{i\alpha} \Psi \numberthis \label{eq:conservation:charge1}\\
\Psi^\dagger \rightarrow & e^{-i\alpha} \Psi^\dagger \numberthis \label{eq:conservation:charge2}
\end{align*}

Allowable combinations are invariant under multiplication by $e^{i\alpha}$--charge is conserved. Table \ref{table:conserve} shows some pairs of symmetries and conservation laws.

\begin{table}[H]
	\caption{Invariance and Conservation Laws}\label{table:conserve}
	\begin{center}
			\begin{tabular}{|l|l|l|}  \hline
				Invariance&Conservation&Ref\\ \hline
				Time& Energy&(\ref{eq:conservation:energy})\\  \hline
				Space&Charge&(\ref{eq:conservation:charge1}),  (\ref{eq:conservation:charge2})\\ \hline
				Phase&Momentum&Section \ref{sec:momentum:conservation}\\ \hline
			\end{tabular}
	\end{center}
\end{table}

NB, this isn't a real electron, as $\Psi$ is for Bosons. A negative pion would be OK.


\section{Energy conservation and waves}\label{sec:energy:conservation}

\subsection{Phase Velocity and Group Velocity}

\subsubsection{Phase Velocity and Group Velocity of non-relativistic particles}
Generally phase velocity doesn't have anything to do with measurable things in quantum mechanics. Group velocity is what counts.

Consider $\sin(kx-\omega t)$. The wave moves so that $kx-\omega t$ remains constant.
\begin{align*}
x =& \underbrace{\frac{\omega}{k}}_\text{phase velocity} t\\
\omega =& \frac{2 \pi}{T}\\
k =& \frac{2 \pi}{\lambda}
\end{align*}

This is connected with Schr\"odinger equation, since $\omega$ is energy, and $k$ momentum, so:
\begin{align*}
\omega=&\frac{k^2}{2m} \text{, whence}\\
\frac{\omega}{k}=&\frac{k}{2m} \text{, phase velocity.}\\
=& \frac{v}{2}  \text{, where $v$ is velocity of classical particle. Let's add a constant energy:}\\
\omega=&\frac{k^2}{2m}+c \text{, changes phase velocity--but changing $E$ has no physical effect!} \numberthis \label{fig:phase:c}
\end{align*}
Now we write:
\begin{align*}
	\Psi =& \sum_k a^-(k) e^{ikx}e^{- i \omega(k) t} e^{-i c t}
\end{align*}
Given solution to Schr\"odinger equation, adding constant to phase just multiplies $\Psi$ by $e^{-i c t}$. Now all physical quantities depend on $\Psi\Psi^\dagger$--$e^{-i c t}$ cancels. So the phase velocity isn't physical.

We will try to follow where the constructive interference takes place between two waves.

\begin{align*}
	\sin(kx-\omega(k)t) +& \sin(k^\prime x-\omega(k^\prime)t)\text{ will reinforece each other when}\\
	kx-\omega(k)t=&k^\prime x-\omega(k^\prime)t\\
	x =& \frac{\omega-\omega^\prime}{k-k^\prime}t\\
	\rightarrow & \big[\underbrace{\frac{d \omega}{dk}}_\text{group velocity}\big] t \text{, as $k-k^\prime \rightarrow 0$}\\
	\frac{d \omega}{dk}=&\frac{k}{m} \text{, from (\ref{fig:phase:c})}
\end{align*}

\begin{itemize}
	\item Group velocity does not depend on $c$
	\item Group velocity matches velocity of relativistic particle.
\end{itemize}

\subsubsection{Phase Velocity and Group Velocity of non-relativistic particles}

\begin{align*}
E =& \sqrt{P^2 + m^2}\\
\omega =& \sqrt{k^2 + m^2}\\
\omega =& \left| k \right| \text{ at speed of light(m=0), so}\\
V_p =& \frac{\omega}{k} \text{ phase velocity of photon waves}\\
 =&1\\
 V_g =& \frac{d\omega}{dk} \text{ group velocity of photon waves}\\
 =&1 \text{. Now lets us make $m>0$}\\
 V_p =& \frac{\omega}{k}\\
 =&\sqrt{\frac{k^2+m^2}{k^2}}\\
 =&\sqrt{1+\frac{m^2}{k^2}}\\
 >&1 \\
 V_g =& \frac{d\omega}{dk} \\
 =& \frac{\cancel{2}k}{\cancel{2}\sqrt{k^2+m^2}}\\
 =& \frac{1}{\sqrt{1+\frac{m^2}{k^2}}}\\
 <&1\\
 V_p V_g =& 1
\end{align*}
Difference between group and phase velocities is connected to dispersion.

\subsection{Momentum Conservation}\label{sec:momentum:conservation}

Consider a bunch of operators action on an initial state to produce a final state. If Figure \ref{fig:particle:heavy:target} we derived conservation of energy. Figure \ref{fig:split:particle} show conservation of momentum for splitting a particle. 
 
\begin{figure}[H]
	\caption{Split Particle at any position $x$}\label{fig:split:particle}
	\includegraphics[width=0.8\textwidth]{split-particle1}
\end{figure}

\begin{align*}
\bra{k_1^\prime,k_2^\prime}&\int \Psi^\dagger(x)\Psi^\dagger(x)\Psi(x) dx\ket{k}\\
=&\bra{k_1^\prime,k_2^\prime}\int a^+(k_1^\prime) a^+(k_2^\prime) a^-(k) \underbrace{e^{ikx} e^{-i(k_1^\prime+k_2^\prime)x}}_\text{integrates to Dirac-$\delta$--(\ref{eq:conservation:energy})} dx\ket{k}
\end{align*}

So invariance under space translation yields conservation of momentum.

Another possible symmetry: $\Psi \rightarrow e^{i \lambda} \Psi$: no change to probabilities if number of $\Psi$s is same as number of $\Psi^\dagger$s. This os connected with conservation of charge.

\subsection{Fermions}

\subsubsection{Introduction to Fermions}

\begin{itemize}
	\item Can have any number of Bosons in same state (same place);
	\item They have a tendency to be in same state. E.g. particles decays and emits photon. If lots of other photons already on some state, decay will preferentially favour that state.
\end{itemize}

Fermions cannot be in same state (Pauli Exclusion Principle).

\begin{align*}
\Psi(x) =& \sum_k a^-(k) e^{ikx}\\
State:& \ket{001...01010...} \text{, ones and zeroes only!}\\
\Psi^\dagger(x) =& \sum_k a^+(k) e^{-ikx}
\end{align*}

Focus on one state, either full $(\ket{1})$ or empty $(\ket{0})$: those are only two possibilities. What are the algebraic rules for creation and annihilation operators?

\begin{align*}
c^+\ket{0} =& \ket{1} \text{ for Fermions used $c$ rather than $a$}\\
c^+\ket{1} =& 0 \text{, i.e. $\psi(x) \equiv 0$}\\
c^-\ket{1} =& \ket{0}\\
c^-\ket{0} =& 0\\
\end{align*}

We deduce:
\begin{align*}
c^+c^-\ket{0} =& 0\\
c^-c^+\ket{0} =& \ket{0}\\
(c^+c^-+c^-c^+)\ket{0} =& \ket{0}\\
c^+c^-\ket{1} =& \ket{1} \\
c^-c^+\ket{1} =&0 \\
(c^+c^-+c^-c^+)\ket{1} =& \ket{1}\\
\{c^+,c^-\} =& 1 \text{, where anticommutator is defined}\\
\{c^+,c^-\} \triangleq(c^+c^-+c^-c^+)& \text{. We can also show}\\
\{c^+,c^+\} =&0 \text{ and}\\
\{c^-,c^-\} =&0
\end{align*}

So Fermions have parallel mathematics: $[,] \leftrightarrow \{,\}$. Cannot put two particles into same state, as $c^+c^+=0$.

What about different momenta? We can show that:
\begin{align*}
\{c^+_k,c^-_{k^\prime} \} =& \delta_{k,k^\prime} \\
\{c^+_k,c^+_{k^\prime} \} =&0 \\
\{c^-_k,c^-_{k^\prime} \} =&0
\end{align*}

Can we have two Fermions at the same position? Let's try to create them at the origin.

\begin{align*}
\Psi^\dagger(0) \Psi^\dagger(0) \ket{0} =& \sum_{k,k^\prime} c^+_k c^+_{k^\prime} \ket{0}\\
=& 0
\end{align*}

So the answer is no: we can't have two Fermions at the same position. So we can't have an electron laser.

\subsubsection{Ground State}
We now compare the ground state of a system of Bosons with the ground state of a system of Fermions. In Figure \ref{fig:electons:confined:momentum} we have some particles confined in momentum space.

\begin{figure}[H]
	\caption{Particles confined in momentum space (3D)}\label{fig:electons:confined:momentum}
	\includegraphics[width=0.8\textwidth]{electons-confined-momentum}
\end{figure}

Allowable values form a lattice.
\begin{itemize}
	\item We can put lots and lots of Bosons in the lowest energy (zero momentum), giving a Bose condensate. Complete uncertainty of position.
	\item Have to assign different momentum to electrons--Figure \ref{fig:adding:more:fermions}. Many Fermion configuration has much more energy that corresponding Boson configuration. 
\end{itemize}

\begin{defn}[Bose condensate]
	Many Bosons in the lowest energy  state.
\end{defn}

\begin{figure}[H]
	\caption{Adding more Fermions}\label{fig:adding:more:fermions}
	\begin{subfigure}{0.32\textwidth}
		\caption{Two electrons, lowest possible energy}\label{fig:electons:lowest:energy}
		\includegraphics[width=0.9\textwidth]{electons-lowest-energy}
	\end{subfigure}
	\begin{subfigure}{0.32\textwidth}
		\caption{Filled lowest two energy levels.}
		\includegraphics[width=0.9\textwidth]{5-electrons}
	\end{subfigure}
	\begin{subfigure}{0.32\textwidth}
		\caption{$E=\frac{k^2}{2m}$, so there are more positions in each energy level as energy increases.}
		\includegraphics[width=0.9\textwidth]{more-fermions}
	\end{subfigure}
\end{figure}

\begin{defn}[Fermi Sphere]
	Electrons fill a sphere in momentum space, whose radius depends on number of electrons. This is called the Fermi Sphere.
\end{defn}

We can move Bosons and Fermions out of the ground state, as shown in Figures \ref{fig:bosons:move:ground:state} and \ref{fig:fermions:move:ground:state}.
\begin{figure}[H]
	\caption{Moving Bosons and Fermions from lowest state}
	\begin{subfigure}{0.5\textwidth}
		\caption{Bosons in Ground State}\label{fig:bosons:move:ground:state}
		\includegraphics[width=0.9\textwidth]{bosons-ground-state}
	\end{subfigure}
	\begin{subfigure}{0.5\textwidth}
		\caption{Moving Fermions. Take from surface of Fermi Sphere to minimize energy required. Leaves hole behind.}\label{fig:fermions:move:ground:state}
		\includegraphics[width=0.9\textwidth]{moving-outside-fermi}
	\end{subfigure}
\end{figure}

In Figure \ref{fig:fermions:move:ground:state} a charged hole is left behind.
If an electron drops into hole, a photon is emitted.

Consider atom, al electrons in their ground state. Low energy electron is struck by photon, which takes electron to excited state.

\begin{defn}[Fermi Sea]
	Electrons in Fermi Sphere are said to be in Fermi Sea.
\end{defn}

Electrons that are deep in Fermi Sea can't be excited by low energy photons. Only shallow electrons are accessible.

\subsubsection{Dirac and antiparticles}

We consider the 1-dimensional case. Imagine a wave propagating to the tight with speed $c$.
\begin{align*}
\omega =& k\\
\frac{\partial \psi}{\partial t} =& - \frac{\partial \psi}{\partial x}  \numberthis \label{eq:rh:dirac}\\
\psi =& e^{i(kx-\omega t)} \numberthis \label{eq:rh:dirac:wave}
\end{align*}
What if $k$ negative? Then we have negative energy. But if we had arbitrarily negative energies, we could keep emitting photons all the way down. So (\ref{eq:rh:dirac}) would not make sense for Bosons.

Dirac said to fill all negative energy states: so lowest energy state has all negative energies filled. Now we can take an electron out, leaving hole. Now we have two positive energy objects, electron and positron.




\section{Dirac equation and Higgs particles}

\subsection{Dirac equation}

The right-handed Dirac equation (\ref{eq:rh:dirac}) has the following solution:
\begin{align*}
\psi =& e^{i(kx-\omega t)} \text{, from (\ref{eq:rh:dirac:wave}), whence:}\\
V_g =& \frac{\omega}{k} \text{, from which}\\
\frac{\partial \psi_R}{\partial t} + \frac{\partial \psi_R}{\partial x} =&0
\end{align*}

If $k<0$, {$\omega$<0}--negative energy, which is not a good thing. So we fill up the Dirac Sea. Then we can remove one electron, creating a hole in the Dirac Sea; particle and hole both have positive momentum and positive energy.

We now introduce left moving electrons.

\begin{align*}
\frac{\partial \psi_L}{\partial t} =& \frac{\partial \psi_L}{\partial x} \numberthis \label{eq:lh:dirac:wave}\\
\frac{\omega}{k} =& -1
\end{align*}
Fill sea with left moving negative energy.

Define $\psi=\begin{pmatrix}
	\psi_R\\
	\psi_L
\end{pmatrix}$, and write (\ref{eq:rh:dirac:wave}) and (\ref{eq:lh:dirac:wave}):

\begin{align*}
	\frac{\partial \psi}{\partial t} =& - \alpha \frac{\partial \psi}{\partial x} \text{, where}\\
	\alpha =& \begin{pmatrix}
		1&0\\
		0&-1
	\end{pmatrix} \text{ we will find the following useful:} \numberthis\label{eq:alpha}\\
	\omega = \alpha k \numberthis \label{eq:dirac:1}
\end{align*}

Now we give the particles mass $m$.

\begin{align*}
	\omega^2 =& k^2 + m^2 \text{. Now we generalize (\ref{eq:dirac:1})}\\
	\omega =& \alpha k + m \beta \text{. This is the 1D Dirac equation.} \numberthis \label{eq:Dirac:1D}\\
	\omega^2 =& \alpha^2 k^2 + m^2 \beta^2 + m k (\alpha \beta + \beta \alpha) \text{, so we want:}\\
	\alpha^2 =& 1 \text{, satsfied by (\ref{eq:alpha})}\\
	\beta^2 =& 1 \numberthis \label{eq:beta2}\\ 
	\{\alpha,\beta\} =& 0 \numberthis \label{eq:alpha:beta}\\
	\beta =& \begin{pmatrix}
	0 & 1\\
	1 & 0
	\end{pmatrix} \text{ satisfies (\ref{eq:beta2}) and (\ref{eq:alpha:beta}).}
\end{align*}

Mass term couples equations for $\Psi_L$ and $\Psi_R$. Dirac equation is Lorentz covariant.

Let us make particle stand still, i.e. $k=0 \implies \omega = \beta m$.

\begin{align*}
	i \frac{\partial \psi}{\partial t} =& \beta m \Psi\\
	i \dot{\psi_R} =& m \psi_L\\
	i \dot{\psi_L} =&  m \psi_R \\
	i \dot{\psi_+} =&  m \psi_+ \text{, where $\psi_+\triangleq\psi_R + \psi_L$--corresponds to $\omega=m$}\\
	i \dot{\psi_-} =&  m \psi_- \text{, where $\psi_-\triangleq\psi_R - \psi_L$--corresponds to $\omega=-m$}
\end{align*}

So particle at rest still has positive and negative energies: throw latter in Dirac Sea.

\subsection{Higgs particles}

In the presence of the Higgs field, $\phi(x)$, the Dirac equation becomes:
\begin{align*}
i \dot{\psi_R} =& -i \partial_x \psi_R + g \phi \psi_L\\
i \dot{\psi_L} =& i \partial_x \psi_L + g \phi \psi_R \text{, which gives massless electron if $\phi=0$}
\end{align*}
What if lowest energy of Higgs favours a non-zero value, $\phi(x)=\phi$? The $m=g\phi$. This could come about as the result of non-lienar interactions between Boson and Fermions.

Higgs field is a Bosonic field, which has an energy from a potential, $V(\phi)$. There is a symmetry so $V(\phi)=V(-\phi)$.

\begin{figure}[H]
	\caption[Higgs potential]{Higgs potential: reason for shape not fully understood at time of lecture.}
	\begin{subfigure}{0.5\textwidth}
		\caption{Higgs potential, showing two minima. Either one gives rise to mass of $g\phi$}\label{fig:higgs:potential}
		\includegraphics[width=0.8\textwidth]{higgs-potential}
	\end{subfigure}
	\begin{subfigure}{0.5\textwidth}
		\caption{Higgs field vibrating around vacuum. Frequency is related to mass of Higgs particle: Higgs corresponds to quanta of vibration.}
		\includegraphics[width=0.8\textwidth]{higgs-vibrating}
	\end{subfigure}
\end{figure}

If Higgs field is coupled in an interesting way to Fermions, Higgs will respond to collisions of Fermions and create Higgs particles. In the Standard Model there is only one Higgs, Supersymmetry has two.

If vacuum were symmetric--Figure \ref{fig:higgs:potential}-- $\phi$ would be zero: it is the breaking of the symmetry which gives electron its mass. All particles except photon get their mass through the Higgs Mechanism. Except for Higgs particle: it gets its mass from a different mechanism.

\subsection{3+1 Dimensional Dirac Equation}

Dirac started with generalization.

\begin{align*}
\omega =& \underbrace{\alpha . k}_{\alpha_1 k_1 + \alpha_2 k_2 + \alpha_3 k_3} +\beta m\\
\omega =& \alpha_1 k_1 + \alpha_2 k_2 + \alpha_3 k_3 +\beta m\\
\omega^2 =& k_1^2 + k_2^2 +k_3^2 +m^2\\
\omega^2 =& (\alpha_1 k_1 + \alpha_2 k_2 + \alpha_3 k_3 +\beta m)(\alpha_1 k_1 + \alpha_2 k_2 + \alpha_3 k_3 +\beta m) \text{, whence}\\
\alpha_i^2 =& 1  \text{ and} \numberthis \label{eq:dirac:ac1}\\
\beta^2 =& 1 \text{ and} \numberthis \label{eq:dirac:ac2}\\
\{\alpha_i,\alpha_j]\} =& 2 \delta_{i,j} \numberthis \label{eq:dirac:ac3}\\
\{\alpha_i,\beta\} =& 0 \numberthis \label{eq:dirac:ac4}
\end{align*}

Cannot satisfy these equations for 2 dimensional matrices. $4 \times 4$ are the first case.

Here are the Dirac Matrices (not unique, but equivalent).

\begin{align*}
	\beta =& \begin{pmatrix}
		1&0&0&0\\
		0&1&0&0\\
		0&0&-1&0\\
		0&0&0&-1
	\end{pmatrix} \numberthis \label{eq:beta:1}\\
	\alpha_1 =& \begin{pmatrix}
		0&0&0&1\\
		0&0&1&0\\
		0&1&0&0\\
		1&0&0&0
	\end{pmatrix}\numberthis \label{eq:alpha1:1}\\ 
	\alpha_2 =& \begin{pmatrix}
		0&0&0&-i\\
		0&0&i&0\\
		0&-i&0&0\\
		i&0&0&0
	\end{pmatrix} \numberthis \label{eq:alpha2:1} \\ 
	\alpha_3 =& \begin{pmatrix}
		0&0&1&0\\
		0&0&0&-1\\
		1&0&0&0\\
		0&-1&0&0 
	\end{pmatrix} \numberthis \label{eq:alpha3:1}
\end{align*}

Partitioning, and using the Pauli matrices:
\begin{align*}
	\beta = \begin{pmatrix}
		I&0\\
		0&I
	\end{pmatrix}\\
	\alpha = \begin{pmatrix}
		0&\sigma\\
		-\sigma&0
	\end{pmatrix}
\end{align*}

\begin{align*}
	\psi =& \begin{pmatrix}
		\psi_1\\
		\psi_2\\
		\psi_3\\
		\psi_4
	\end{pmatrix} \text{ spinor--left/right and spin up/down}\\
	i \frac{\partial \psi}{\partial t} =& -i \alpha_i \frac{\partial \psi}{\partial x^i} + \beta m \psi \text{, the famous Dirac equation} \numberthis \label{eq:famous:dirac}
\end{align*}


\section{Angular momentum}

Angular momentum is a vector, which points along the direction of rotation and follows height hand rule: fingers follow rotation, thumb points along vector.

There are two kinds:
\begin{itemize}
	\item orbital angular momentum--motion of centre of mass;
	\item spin angular momentum--angular momentum in frame where momentum is zero.
\end{itemize}

Is rotating nucleus the same object as non-rotating? Note that angular momentum is quantized, so it takes a discrete amount of energy to start rotation. Now an atom with too much rotation will fly apart--definitely different. 

What about electron? Is it too small for us to rotate? The smaller the object, the more energy to rotate with given amount of angular momentum.Maybe electron too small.

Angular momentum for small object like electron characterizes object.

Let us invent a particle with definite angular momentum, a "spintron". Can it point in any direction? Yes, since laws of physics are rotationally invariant. It is quantized.

Theory is mathematical: it seems abstract, but corresponds to experiment.

Classically:
\begin{align*}
mr^2=&I\\
E=&\frac{L^2}{2I}\\
=&\frac{L^2}{2mr^2}\\
\rightarrow &\inf \text{ as } r \rightarrow 0 \text{ for discrete $L$}
\end{align*}

Figure \ref{fig:two:particles:rotating} depicts two particles rotating.
\begin{figure}[H]
	\caption[Two particles rotating]{Two particles rotating. Total momentum is zero, but there is angular momentum.}\label{fig:two:particles:rotating}
	\includegraphics[width=0.8\textwidth]{two-particles-rotating}
\end{figure}

Orbital angular momentum of a point particle moving about origin. 
\begin{align*}
\vec{L} =& \vec{R} \times \vec{p} \text{ accordng to convention -- c.f. right hand rule.}\\
L_x =& y P_z - z P_y\\
L_y =& z P_x - x P_z\\
L_z =& x P_y - y P_x
\end{align*}

We will determine commutation relations of angular momentum operators.

We have:
\begin{align*}
	[X_i,X_j] =& 0\\
	[P_i,P_j]=& 0\\
	[X_i,P_j]=& i \hslash \delta_{i,j}`
\end{align*}

\begin{align*}
	[L_x,L_y]=& [y P_z - z P_y,z P_x - x P_z]\\
	=&[y P_z, z P_x] + [z P_y,x P_z] \text{, since other two terms vanish}\\
	=& y P_x[P_z,z] + P_y x[z,P_z]\\
	=& -i \hslash y P_x + i \hslash P_y x\\
	=& i \hslash L_z \text{. Similarly} \numberthis \label{eq:lx:ly}\\
	[L_y,L_x]=& i \hslash L_x \numberthis \label{eq:ly:lz}\\
	[L_z,L_x]	=& i \hslash L_y \numberthis \label{eq:lz:lx}	
\end{align*}

 These relationships:
\begin{itemize}
	\item remain true if we add angular momenta for many particles;
	\item are true for spin
	 \item are relationally invariant.
\end{itemize}

We focus on the $z$-axis ("quantization axis") and introduce two new quantities--raising and lowering operators. Choosing units where $\hslash-1$:

\begin{align*}
	L_\pm =& L_x \pm i L_y \numberthis \label{eq:L_pm} \\
	[L_+,L_z] =& [Lx+i L_y,L_z]\\
	=& -iL_y + i^2 L_x \text{, from (\ref{eq:lz:lx}) and (\ref{eq:lx:ly})}\\
	=& -L_+ \text{ from (\ref{eq:L_pm}). Similarly} \numberthis \label{eq:L+:Lz}\\
	[L_-,L_z] =& + L_- \numberthis \label{eq:L-:Lz}
\end{align*}

\begin{thm}[$L_+$ is a creation operator for $L_z$]
	\begin{align*}
		L_z \ket{m} =& m \ket{m} \numberthis \label{eq:L:eigen}\\
		\implies&\\
		L_z (L_+ \ket{m}) =& (m+1) (L_+ \ket{m})
	\end{align*}	
\end{thm}

\begin{proof}
	\begin{align*}
		L_+L_z-L_zL_+ =& -L_+ \text{ from (\ref{eq:L+:Lz}), whence:}\\
		L_zL_+ =& L_+L_z + L_+\ \text{ and}\\
		L_z L_+ \ket{m} =& (L_+L_z + L_+)\ket{m}\\
		=& L_+ (L_z + I)\ket{m}\\
		=& L_+ (m + 1) \ket{m} \text{ from (\ref{eq:L:eigen})}\\
		=&  (m + 1) (L_+ \ket{m})
	\end{align*}
\end{proof}

\begin{cor}[$L_-$ is an annihilation  operator for $L_z$]
	\begin{align*}
	L_z \ket{m} =& m \ket{m} \\
	\implies&\\
	L_z (L_- \ket{m}) =& (m-1) (L_+ \ket{m})
	\end{align*}	
\end{cor}

Angular momentum spectrum is spaced by integers. How far can we bump it up before the object changes? Imagine a classical object with fixed $L^2$. We can maximize $L_z$  by pointing it upward.

Our theorem ignore possibility that $L^+m_{max}=0$ and $L^-m_{min}=0$. (c.f. harmonic oscillator). By relational invariance, $m_{min}=-m_{max}$. Must be symmetric about zero.

\begin{itemize}
	\item $-m_{max},...,-1,0,1,...m_{max}$
	\item $-m_{max},...,-\frac{1}{2},\frac{1}{2},...m_{max}$
\end{itemize}

\begin{defn}[Spin]
	$\frac{1}{2}$ is known as the spin. It can be 0. There are no elementary particles known with spin 3, but there are atoms. Similarly no half-spin particle over $\frac{3}{2}$.
\end{defn}

What about $L^2$? If you know $L_z$ at maximum, there will be uncertainty in other two components, so $L^2$ is not quite $m_{max}^2$.

\begin{align*}
	L^2 =& L_- L_+ + L_z^2 \text{, classically, but not quantum because of commutator}\\
	L^2 =& L_- L_+ + L_z^2 +L_z {, quantum}\\
	L^2 \ket{m_{max}}=& L_- L_+ \ket{m_{max}} + L_z^2 \ket{m_{max}} +L_z \ket{m_{max}}\\
	=& (0 +m_{max}^2 + m_{max})\ket{m_{max}} \\
	=& m_{max} (m_{max}+1)\ket{m_{max}}
\end{align*}

All states have same value of $\braket{L^2}$. Show $[L^2,L_i]=0$.

Consider spin $\frac{1}{2}$--call them $\ket{+}$ and  $\ket{-}$.

\begin{align*}
e^{i\theta} \alpha \ket{+} + e^{i\theta}\beta \ket{-} \text{, has probability $\alpha^2$ of being in $\ket{+}$}
\end{align*}
So can ignore 1 degree of freedom. $\alpha$ and $\beta$ have 2 degrees of freedom each. Phase eliminates one, and $alpha^2+\beta^2=1$ eliminates one other, leaving two. This is the same a number of degrees of freedom in direction.

Measured value is always $\pm \frac{1}{2}$; average may be zero; it behaves like a classical variable.

\begin{thm}[Spin statistics]
		All particles with half integral spin are Fermions;
		All particles with  integral spin are Bosons.
\end{thm}

\begin{proof}
	TBP
\end{proof}
Take two half spin particles: composite behaves like a Boson--e.g. Hydrogen atom; deuteron is Boson, deuterium is Fermion.


\section{Spin}

\subsection{Combining Fermions to make Bosons}

Paradox: if two Fermions make a Boson, can we have two of these Bosons in the same state.

Wave function of Boson is supposed to be symmetric:

\begin{align*}
	\psi(x,y) =&\psi(y,x) \text{e.g. }\\
	\psi(x,y) =& \phi(x)\phi(y)\\
	\psi(x,y) =&-\psi(y,x) \text{ for Fermions}
\end{align*}

Now imagine a wave function,$\psi$ , that isn't symmetric:
\begin{align*}
\psi(x,y)+& \psi(y,x) \text{ symmetric}\\
\psi(x,y)-& \psi(y,x) \text{ antisymmetric}
\end{align*}

Now imagine 4 Fermions--2 electons(X) and 2 protons(Y).

\begin{align*}
\psi(x_1,x_2,y_1,y_2) = &-\psi(x_2,x_2,y_1,y_2)\\
=& \psi(x_1,x_2,y_2,y_1)
\end{align*} 

Hydrogen atom at origin plus another hydrogen atom:$\psi(x_1,y_1) \phi(x_2,y_2)$.
If we antisymmetrize:
\begin{align*}
	\psi(x_1,y_1) \phi(x_2,y_2) -& \psi(x_2,y_1) \phi(x_1,y_2) 
	-\psi(x_1,y_2) \phi(x_2,y_1) + \psi(x_2,y_2) \phi(x_1,y_1)\\
	x_1 \leftrightarrow& x_2 \text{ changes sign, and}\\
	y_1 \leftrightarrow& y_2 \text{ changes sign, but}\\
	1 \leftrightarrow& 2 \text { preserves sign.}
\end{align*}

What if both atoms have same wave function? Ignoring a factor of $2$:

\begin{align*}
	\psi(x_1,y_1) \psi(x_2,y_2) -& \psi(x_2,y_1) \psi(x_1,y_2) 
\end{align*}

Both atoms can be in same place, but we can't find 2 electrons in same place.

\subsection{Mathematics of Spin}

Mathematics of Spin needed for Colour, Isospin, etc.

\subsubsection{Half spin}

We have two values along the z axis: $S_z = m \hslash$. We will use $S$ for spin, $L$ for orbital angular momentum.

\begin{align*}
	[S_x,S_y] =& i S_z \numberthis \label{eq:sx:sy}\\
	[S_y,S_z] =& i S_x  \numberthis \label{eq:sy:sz}\\
	[S_z,S_x] =& i S_y  \numberthis \label{eq:sz:sx}\\
	S_\pm =& S_x \pm S_y
\end{align*}

Can express spin as $\begin{pmatrix}
\alpha\\
\beta
\end{pmatrix}$ or $\alpha\ket{+} + \beta\ket{-}$ 

Components are 2D matrices satisfying algebra (\ref{eq:sx:sy}), (\ref{eq:sy:sz}), and (\ref{eq:sz:sx}). They are not unique, but equivalent. We will use the Pauli spin\footnote{Without $\frac{1}{2}$ they are known as Pauli matrices.} matrices:
\begin{align*}
	S_x =& \frac{1}{2}\begin{pmatrix}
		0&1\\
		1&0
	\end{pmatrix}\\
	S_y =& \frac{1}{2}\begin{pmatrix}
		0&-i\\
		i&0
	\end{pmatrix}\\
	S_z =& \frac{1}{2}\begin{pmatrix}
		1&0\\
		0&-1
	\end{pmatrix}
\end{align*}

\subsubsection{Spin 1}

Three states $\{-1,0,1\}$, a Boson. Are there $3\times3$ matrices satisfying (\ref{eq:sx:sy}), (\ref{eq:sy:sz}), and (\ref{eq:sz:sx})?

\begin{align*}
	S_z = i \begin{pmatrix}
		0&1&0\\
		-1&0&0\\
		0&0&0
	\end{pmatrix}\\
	S_y = i \begin{pmatrix}
		0&0&1\\
		0&0&0\\
		-1&0&0
	\end{pmatrix}\\
	S_x = i \begin{pmatrix}
		0&0&0\\
		0&0&1\\
		0&-1&0
	\end{pmatrix}
\end{align*}

Find eigenstates of $S_z$:
\begin{align*}
	i \begin{pmatrix}
		0&1&0\\
		-1&0&0\\
		0&0&0
	\end{pmatrix}\begin{pmatrix}
		\alpha\\
		\beta\\
		\gamma
	\end{pmatrix}=&\begin{pmatrix}
		\alpha\\
		\beta\\
		\gamma
	\end{pmatrix}\\
	=&i\begin{pmatrix}
		\beta\\
		-\alpha\\
		0
	\end{pmatrix}
\end{align*}

Return to real electron that can move in space. We turn the 2D spinors into functions of position: $\begin{pmatrix}
\psi_u(x)\\
\psi_d(x)
\end{pmatrix}$.

For spin 1: $\begin{pmatrix}
\phi_1(x)\\
\phi_2(x)\\
\phi_3(x)
\end{pmatrix}$.

\subsection{Dirac Equation: connection with Spin}

\begin{align*}
i \frac{\partial \psi}{\partial t} =& - i \alpha_i \frac{\partial \psi}{\partial x^i} + \beta m \psi\\
\omega =& \sum_i \alpha_i k_i + \beta m\\
\omega^2 =& k^2 + m^2
\end{align*}
Where the $\alpha_i$ and $\beta$ satisfy (\ref{eq:dirac:ac1}), (\ref{eq:dirac:ac2}), (\ref{eq:dirac:ac3}), and (\ref{eq:dirac:ac4}).


\begin{thm}[Dirac anticommutation relations.] 
	Equations (\ref{eq:dirac:ac1}), (\ref{eq:dirac:ac2}), (\ref{eq:dirac:ac3}), and (\ref{eq:dirac:ac4}) can be satisfied by suitably chosen $2\times4$ matrices, and they cannot be satisfies by $n \times n$ for $n<4$.
\end{thm}

One set of solutions, which is not the same as (\ref{eq:beta:1}), (\ref{eq:alpha1:1}), (\ref{eq:alpha2:1}), and (\ref{eq:alpha3:1}):
\begin{align*}
	\alpha_i=&\begin{pmatrix}
		\sigma_1&0\\
		0&-\sigma_i
	\end{pmatrix}\\
	\beta =& \begin{pmatrix}
		0&I\\
		I&0
	\end{pmatrix}
\end{align*}

Two kinds of $2\times2$: 2 blocks of $2\times2$ matrices. This corresponds to the physic - there are two energies, positive and negative, and two spins.
The 4 components are not a relativistic 4-vector! The easiest way to see this is to set momentum to zero, then $\alpha$s drop out.

\begin{align*}
i \frac{\partial \psi}{\partial t} =&  \beta m \psi \text{ for particle at rest.}\\
i \frac{\partial}{\partial t}\begin{pmatrix}
\psi_+\\
\psi_-
\end{pmatrix}=&m \begin{pmatrix}
0&I\\
I&0
\end{pmatrix}\begin{pmatrix}
\psi_+\\
\psi_-
\end{pmatrix}\\
i \frac{\partial \psi_+}{\partial t} =&   m \psi_-\\
i \frac{\partial \psi_-}{\partial t} =&   m \psi_+ \\
i \frac{\partial (\psi_++\psi_-)}{\partial t} =&   m (\psi_++\psi_-) \text{ positive energy--$\omega=m$}\\
i \frac{\partial (\psi_+-\psi_-)}{\partial t} =&   -m (\psi_+-\psi_-) \text{ negative energy--$\omega=-m$}
\end{align*}

So we end up with two uncoupled equations! $2\times2$ is connected to positive and negative energies. $2\times2$ of $\psi_+$ arises from spin.


\section{Equations of motion of particles and fields}

Equations of motion of fields that describe  particles. The fields all come from a Lagrangian. In quantum field theory, Lagrangian describes everything.

\subsection{Review of classical fields}

\begin{align*}
\mathcal{L}=& \mathcal{L}(\phi,\partial_t \phi, \partial_x \phi)\\
=&\mathcal{L}(\phi,\partial_{\mu} \phi)\\
\frac{\partial}{\partial t} \frac{\partial \mathcal{L}}{\partial \frac{\partial \phi}{\partial t}}+\sum_i \frac{\partial}{\partial x_i} \frac{\partial \mathcal{L}}{\partial \frac{\partial \phi}{\partial x_i}}=&\frac{\partial \mathcal{L}}{\partial \phi} \text{, equation of motion} \numberthis \label{eq:lagrange}
\end{align*}

If $\mathcal{L}$ depends on several fields, there will be a version of (\ref{eq:lagrange})  for each field.

Example.

\begin{align*}
	\mathcal{L} =& \frac{1}{2} (\frac{\partial \phi}{\partial t})^2 - \frac{1}{2} (\frac{\partial \phi}{\partial x})^2 - \frac{1}{2} (\frac{\partial \phi}{\partial y})^2 - \frac{1}{2} (\frac{\partial \phi}{\partial z})^2 - V(\phi)
\end{align*}

Substituting in (\ref{eq:lagrange}):
\begin{align*}
	\ddot{\phi} - \frac{\partial^2 \phi}{\partial x^2}  - \frac{\partial^2 \phi}{\partial y^2}  - \frac{\partial^2 \phi}{\partial z^2} =& - \frac{\partial V}{\partial \phi} \text{. The simplest interesting case is:}\\
	V(\phi) =& \frac{m^2 \phi^2}{2}\\
	\ddot{\phi} - \frac{\partial^2 \phi}{\partial x^2}  - \frac{\partial^2 \phi}{\partial y^2}  - \frac{\partial^2 \phi}{\partial z^2} =& - m^2 \phi \numberthis \label{eq:simplest:non:trivial}
\end{align*}




Differentiating a wave function $e^{-i \omega t} e^{-i k x}$ twice gives a factor $-\omega^2$. Substituting in (\ref{eq:simplest:non:trivial}):
\begin{align*}
-\omega^2 \cancel{\phi} + (k_x^2+k_y^2+k_z^s)\cancel{\phi} =& -m^2 \cancel{\phi}\\
-\omega^2  + (k_x^2+k_y^2+k_z^s) =& -m^2 \\
\omega^2 + P^2 = m^2
\end{align*}

If $\mathcal{L}$ is a scalar, equations of momentum will be Lorentz Invariant.

What about a more complex $V(\phi)$?
\begin{align*}
	V(\phi) =& \frac{m^2 \phi^2}{2} + g \phi^3\\
	\ddot{\phi} - \frac{\partial^2 \phi}{\partial x^2}  =& - m^2 \phi - 3 g \phi^2
\end{align*}

Quadratic terms in Lagrangian lead to linear equations; higher powers lead to non-linearities--cannot superpose. May scatter.

Consider two fields:
\begin{align*}
	\mathcal{L} =& \frac{1}{2}(\partial_{\mu} \phi)^2 + \frac{1}{2}(\partial_{\mu} \rho)^2 - \frac{m^2 \phi^2}{2} - \frac{M^2 \rho^2}{2} - \rho \phi^2\\
	\ddot{\phi]} - \partial_x^2 \phi =& -m^2 \phi - \rho \phi
	\\
	\ddot{\rho]} - \partial_x^2 \rho =& -m^2 \rho -phi^2
\end{align*}

These are non-linear and coupled, so the particles scatter each other.

Consider the Dirac equation.

\begin{align*}
\dot{\psi} =& i \alpha \frac{\partial \psi}{\partial x} + m \beta \psi\\
\mathcal{L} =& i \big( \psi^\dagger \partial_t \psi +  \psi^\dagger \alpha \partial_x \psi + \psi^\dagger \beta \psi m \big)
\end{align*}

Suppose there is another field: can add scalar field to Dirac field and get scattering, and generate equations of motion for both fields.

\begin{align*}
\mathcal{L} =& i \big( \psi^\dagger \partial_t \psi +  \psi^\dagger \alpha \partial_x \psi + \psi^\dagger \beta \psi m \big) - \psi^\dagger \beta \psi \phi
\end{align*}

Lagrangian is what specifies a field theory.

\subsection{Quantum Fields}

Imagine Lagrangian containing $\phi^3$, where $\phi = (a_+ + a_-)$.
Figure \ref{fig:phi:3:x} shows the kinds of interaction that may occur. Quadratic terms in Lagrangian give rise to motion, higher powers to interactions: 1 particle $\rightarrow$ 2, 2 particle3 $\rightarrow$ 1, 0 particles $\rightarrow$ 3, 3 particles $\rightarrow$ 0.
 
\begin{figure}[H]
	\caption{$\phi^3(x)$ gives rise to many possible interactions.}\label{fig:phi:3:x}
	\includegraphics[width=0.8\textwidth]{phi-3-x}
\end{figure}

\begin{align*}
(\frac{\partial \phi}{\partial x})^2 =& \big( \phi(x) - \phi(x^\prime)\big)^2\\
=& \phi(x)^2 + \phi(x^\prime)^2 - 2 \underbrace{ \phi(x) \phi(x^\prime)}_\text{absorbs particle at $x$, reemits at $x^\prime$--i.e. move particle}
\end{align*}
\url{https://youtu.be/D7yXoNAg3J8?t=2134}


\section{Field Lagrangians and path integrals}


\end{document}
