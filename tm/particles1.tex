\documentclass[]{article}
\usepackage{caption,subcaption,graphicx,float,url,amsmath,amssymb,amsthm,tocloft,cancel,thmtools,gensymb,braket}
\usepackage[toc,nonumberlist]{glossaries}
\usepackage{glossaries-extra}
\newcommand\numberthis{\addtocounter{equation}{1}\tag{\theequation}}

\newtheorem{thm}{Theorem}
\newtheorem{defn}[thm]{Definition}
\newtheorem{cor}[thm]{Corollary}
\newtheorem{lemma}[thm]{Lemma}
\graphicspath{{figs/}}
\widowpenalty10000
\clubpenalty10000
\setcounter{tocdepth}{1}

%opening
\title{Theoretical Minimum\\Particle Physics 1\\Basic Concepts}
\author{}

\begin{document}

\maketitle

\begin{abstract}
These are my notes Particle Physics 1 lectures from Leonard Susskind's Theoretical Minimum series.
\end{abstract}

\tableofcontents

\section{Particles and light}
\section{Quantum field theory}
\section{Quantum fields and particles}
\section{More quantum field theory}
\section{Energy conservation and waves}
\section{Dirac equation and Higgs particles}
\section{Angular momentum}
\section{Spin}
\section{Equations of motion of particles and fields}
\section{Field Lagrangians and path integrals}




\end{document}
