\documentclass[]{article}
\usepackage{caption,subcaption,graphicx,float,url,amsmath,amssymb,amsthm,tocloft,cancel,thmtools}
\newcommand\numberthis{\addtocounter{equation}{1}\tag{\theequation}}
\newtheorem{thm}{Theorem}
\newtheorem{lemma}{Lemma}

%opening
\title{Statistical Mechanics}
\author{Simon Crase}

\begin{document}

\maketitle

\begin{abstract}
These are my notes from the Statistical Mechanics lectures from Leonard Susskind's Theoretical Minimum series.
\end{abstract}

\tableofcontents

\listoftheorems[ignoreall,onlynamed]

\section{Entropy and conservation of information}
This lecture focuses on the law of conservation of energy, the $-1^{th}$ law of physics. Liouville's theorem.

''In physics, Liouville's theorem, named after the French mathematician Joseph Liouville, is a key theorem in classical statistical and Hamiltonian mechanics. It asserts that the phase-space distribution function is constant along the trajectories of the system—that is that the density of system points in the vicinity of a given system point traveling through phase-space is constant with time. This time-independent density is in statistical mechanics known as the classical a priori probability.''

\begin{align*}
S \triangleq \log(M) \numberthis \label{eq:entropy:simple}
\end{align*}

Proper laws of physics are reversible and therefore preserve the distinctions between states - i.e. information.  In this sense, the conservation of information is more fundamental that other physical quantities such as temperature or energy.  
\begin{itemize}
	\item $S$ conserved as long as we can follow in detail, e.g. if we can follow the motion of each molecule that is disturbed by Friction.
	\item $S$ increases as our ignorance goes up;
	\item $S$ depends on the system and our state of knowledge.
\end{itemize}

\begin{align*}
S \triangleq - \sum_{i=1}^{N} p_i \log{p_i} \numberthis \label{eq:entropy}
\end{align*}


In (\ref{eq:entropy:simple}) and (\ref{eq:entropy}) we use the logarithm so that entropy adds, rather than multiplies.

The First law of Thermodynamics is energy conservation. For a closed system:
\begin{align*}
\frac{dE}{dt} =& 0\\
\sum \frac{dE_i}{dt} =& 0
\end{align*}


\section{Temperature}

Temperature is really a measure of Energy. So the energy of one molecule of an Ideal Gas is given by:
\begin{align*}
E =& \frac{3}{2} k_B t\text{, where:}\\
k_B =& \text{Boltzmann's constant}\\
T =& \text{temperature in Kelvin}
\end{align*}
 In natural units we define a new unit of temperature, $T=k_Bt$, so $k_B=1$.
 
 We want to redefine temperature in therms of Entropy and Energy, and then show that this new definition agrees with our everyday idea of temperature. Temperature is not a fundamental quantity, but is derived as the amount of energy required to add an incremental amount of entropy to a system.  As the energy of a system increases, the number of possible states of a system increases, which means that the entropy increases.  This is the concept behind the second law of thermodynamics, and implies that temperature is always positive.
 
 Our definition reduces to:
\begin{align*}
\Delta E =& \underbrace{\frac{\partial E}{\partial S}}_\text{1 Degree of T} \Delta S\text{,or }\\
dE =& T dS \numberthis \label{eq:T}
\end{align*}

Consider two systems, $A$ and $B$, joined by a narrow pipe. Theorem \ref{thm:heat:flow} shows that T captures our intuition about temperature: heat flows from a higher temperature to lower.

\begin{thm}[Heat Flow]\label{thm:heat:flow}
	Given:
	\begin{align*}
	dE_A + dE_B =& 0 \numberthis \label{eq:thm:dE}\\
	dS_A + dS_B>& 0 \numberthis \label{eq:thm:eS} \\
	dE_i =& T_i dS_i \numberthis \label{eq:thm:T} \\
	T_A>&T_B \numberthis \label{eq:thm:Tgt}
	\end{align*}
	then energy flows from A to B, i.e.
	\begin{align*}
	dE_A<0
	\end{align*}
\end{thm}
\begin{proof}
	\begin{align*}
	T_A dS_A =& dE_A \text{, from (\ref{eq:thm:T})}\\
	=& - dE_B \text{, from (\ref{eq:thm:dE})}\\
	=& - T_B dS_B  \text{, from (\ref{eq:thm:T})}\\
	\implies&\\
	T_A dS_A + T_A dS_B =& T_A dS_B -T_B dS_B\\
	\implies&\\
	T_A (dS_A + dS_B) =& (T_A  -T_B) dS_B\\
	\implies& \text{, using (\ref{eq:thm:eS}) and (\ref{eq:thm:Tgt})}\\
	dS_B >& 0\\
	\implies& \text{using (\ref{eq:thm:T})}\\
	dE_B >& 0\\
	\implies& \text{, using (\ref{eq:thm:dE})}\\
	dE_A<0
	\end{align*}
\end{proof}



\section{Maximizing entropy}

Number of arrangements $\frac{N!}{\prod_{i}n_i!}$. We want to maximize this subject to two constraints:
\begin{align*}
\sum_{i=1}^{N} n_i =& N \numberthis\label{eq:sum:probilities}\\
\sum_{i=1}^{N} E_i n_i =& E N \numberthis\label{eq:total:energy}
\end{align*}

We will find Lemma \ref{lemma:Stirling} useful.
\begin{lemma}[Stirling's approximation]\label{lemma:Stirling}
	\begin{align*}
	N! \approxeq& N^N e^{-N}\numberthis\label{eq:stirling}
	\end{align*}
\end{lemma}

\begin{proof}
	\begin{align*}
	\log(N!) =& \sum_{i=1}^{N}\log (i)\\
	\approxeq &\int_{1}^{N} \log(x) dx\\
	\approxeq& \big[x \log(x) - x \big]_1^N\\
	\approxeq& N\log(N)-N\\
	N! \approxeq& N^N e^{-N}
	\end{align*}
\end{proof}	

\begin{thm}[To maximize the number of arrangements we need to maximize the entropy.]
	To maximize the number of arrangements we need to maximize the entropy.
\end{thm}

\begin{proof}
	\begin{align*}
	C=&\frac{N!}{\prod_{i}n_i!}\\
	\approxeq& \frac{N^N e^{-N}}{\prod_{i}n_i^{n_i} e^{-\sum_{i=1}^{N}n_i}}\tag*{from Lemma \ref{lemma:Stirling}}\\
	\approxeq& \frac{N^N \bcancel{e^{-N}}}{\prod_{i}n_i^{n_i} \bcancel{e^{-N}}}\tag*{using (\ref{eq:sum:probilities})}\\
	\log(C) \approxeq & N \log(N) - \sum_{i=1}^{N} n_i \log(n_i)\\
	\approxeq& N \log(N) - \sum_{i=1}^{N} N p_i \log(N p_i)\\
	\approxeq& N \log(N) - \sum_{i=1}^{N} N p_i \big(\log(N) + \log(p_i))\big)\\
	\approxeq& N \log(N) -  N \log(N) \bcancel{\sum_{i=1}^{N} p_i}  - N \sum_{i=1}^{N}p_i \log(p_i)\tag*{using (\ref{eq:sum:probilities})}\\
	\approxeq& \bcancel{N \log(N)} - \bcancel{N \log(N)}  - N \sum_{i=1}^{N}p_i \log(p_i)\\
	\approxeq& NS \text{, using (\ref{eq:entropy})} \numberthis \label{eq:entropy:logC}
	\end{align*}
\end{proof}

\section{The Boltzmann distribution}

To maximize S subject to constraints (\ref{eq:sum:probilities}) and (\ref{eq:total:energy}), we minimize -S, using Lagrange Multipliers, i.e., minimize
\begin{align*}
\sum_{i=1}^{N}p_i \log(p_i) + \alpha(\sum_{i=1}^{N} p_i-1) + \beta(\sum_{i=1}^{N} E_i p_i-E)
\end{align*}

Differentiating with respect to $p_i$:
\begin{align*}
\log(p_i)+1 + \alpha +\beta E_i=&0\text{, whence}\\
p_i =& e^{-(1+\alpha)} e^{-\beta E_i}\\
=& \frac{1}{Z} e^{-\beta E_i}\text{, where $Z\triangleq e^{(1+\alpha)}$} \numberthis\label{eq:maximize:entropy}
\end{align*}

From (\ref{eq:sum:probilities}) and (\ref{eq:maximize:entropy})
\begin{align*}
1 =& \frac{1}{Z}  \sum_{i=1}^{N} e^-{\beta E_i}\\
Z(\beta) =& \sum_{i=1}^{N} e^-{\beta E_i} \text{. $Z$ is known as the Partition Function.} \numberthis \label{eq:partition:function}
\end{align*}

\begin{thm}[$\beta$ is the inverse temperature]\label{thm:inverseT}
	\begin{align*}
		T =& \frac{1}{\beta}\numberthis\label{eq:inverseT}
	\end{align*}
\end{thm}

\begin{proof}
	From (\ref{eq:total:energy}) and (\ref{eq:maximize:entropy})
	\begin{align*}
	E =& \frac{1}{Z}  \sum_{i=1}^{N} e^{-\beta E_i} E_i \numberthis \label{eq:E}\\
	\frac{\partial Z}{\partial \beta} =& - \sum_{i=1}^{N} e^{-\beta E_i} E_i \numberthis \label{eq:dZ}\\
	E(\beta) =& - \frac{1}{Z} \frac{\partial Z}{\partial \beta} = - \frac{\partial \log(Z)}{\partial \beta} \numberthis \label{eq:E:beta}
	\end{align*}
	
	\begin{align*}
	S =& -\sum_{i=1}^{N} p_i \log(p_i)\\
	=& - \sum_{i=1}^{N} \frac{1}{Z} e^{- \beta E_i}\big[- \beta E_i -\log(Z)\big]\\
	=& \beta E +  \bcancel{\frac{1}{Z}} \log(Z)  \bcancel{\sum_{i=1}^{N} e^{-\beta E_i}}\\
	=& \beta E +   \log(Z)  \numberthis \label{eq:S_Z}
	\end{align*}
	
	\begin{align*}
	dS =& \beta dE + E d\beta + \frac{\partial \log(Z)}{\partial \beta} d\beta \text{, from (\ref{eq:S_Z})}\\
	=& \beta dE \text{, using (\ref{eq:E:beta})}\\
	=& \frac{1}{T} dE \text{, using (\ref{eq:T})}\text{, whence}\\
	T =& \frac{1}{\beta}
	\end{align*}
\end{proof}

Ideal Gas: assume molecules don't interact.
State: 3N coordinates $\{X_1,...X_{3N}\}$ and momenta $\{P_1,...P3N\}$
\begin{align*}
Z =& \int d^{3N}x d^{3N}p . e^{- \frac{\beta}{2m} \sum_{n=1}^{3N} p_n^2}\\
=& \frac{V^N}{N!} \int d^{3N}p . e^{- \frac{\beta}{2m} \sum_{n=1}^{3N} p_n^2}\text{Factorial is controversial}\\
=& \frac{V^N}{N!} \big[\int dp. e^{-\frac{\beta}{2m}p^2}\big]^{3N}\\
=& \frac{V^N}{N!} \big[\frac{2 m \pi}{\beta}\big]^{\frac{3N}{2}}\numberthis\label{eq:Z_beta}
\end{align*}
We now use Stirling's formula (\ref{eq:stirling}).
\begin{align*}
\frac{V^N}{N!} \approxeq& \big(\frac{eN}{N}\big)^N \\
=& \big(\frac{e}{\rho}\big)^N,\text{, and (\ref{eq:E:beta}) becomes}\\
Z \approxeq& \big(\frac{e}{\rho}\big)^N .  \big[\frac{2 m \pi}{\beta}\big]^{\frac{3N}{2}}
\end{align*}

\begin{align*}
\log(Z) =& - \frac{3N}{2} \log(\beta) + const\\
E =& - \frac{\partial \log(Z)}{\partial \beta}\numberthis \label{eq:E:Z}\\
=& \frac{3N}{2} T
\end{align*}


\section{Pressure of an ideal gas and fluctuations}

\subsection{Pressure of an ideal gas}

Motion is isotropic, even near wall. We know this from an honest use of statistical mechanics. Give up intuition because it might not be correct: ''ride the mathematics. All very great physicists are master of thermodynamics.''

We want to determine the Equation of State.

From (\ref{eq:S_Z}) and Theorem \ref{thm:inverseT}.
\begin{align*}
S =& \frac{E}{T} + \log(Z)\\
\underbrace{E - TS}_{A\triangleq\text{Helmholz Free Energy}} =& -T \log(Z) \numberthis \label{eq:helmholz}
\end{align*}

Variables come in pairs, e.g. (V,P):
\begin{itemize}
	\item Control Parameter (V);
	\item Conjugate Thermodynamic Variable (V).
\end{itemize}

\begin{lemma}[Derivatives along contour]\label{thm:derivative:contour}
	Given two dependent variables, S and E, say, and two independent variables, T and V, say:
	\begin{align*}
		\frac{\partial E}{\partial V}\bigg|_S = \frac{\partial E}{\partial V}\bigg|_T - 	\frac{\partial E}{\partial S}\bigg|_V \frac{\partial S}{\partial V}\bigg|_T
	\end{align*}
\end{lemma}
\begin{proof}
	Along a contour of fixed S:
	\begin{align*}
		\Delta E =& \frac{\partial E}{\partial V} \bigg|_T \Delta V + \frac{\partial E}{\partial T} \bigg|_V \Delta T\\
		\frac{\partial E}{\partial V} \bigg|_S=& \frac{\partial E}{\partial V} \bigg|_T  + \frac{\partial S}{\partial T} \bigg|_V \frac{\partial E}{\partial S} \bigg|_V \frac{\Delta T}{\Delta V} \numberthis\label{eq:dEdV}\\
		\text{Now }\Delta S =& \frac{\partial S}{\partial V} \bigg|_T \Delta V + \frac{\partial S}{\partial T} \bigg|_V \Delta T\\
		=& 0\text{, along a contour of constant S, whence}\\
		\frac{\Delta T}{\Delta V} =& - \Bigg[\frac{\frac{\partial S}{\partial V}\big|_T}{\frac{\partial S}{\partial T}\big|_V}\Bigg]\text{, along a contour of constant S, and (\ref{eq:dEdV}) becomes:}\\
		\frac{\partial E}{\partial V} \bigg|_S=& \frac{\partial E}{\partial V} \bigg|_T  - \cancel{\frac{\partial S}{\partial T} \bigg|_V} \frac{\partial E}{\partial S} \bigg|_V \frac{\frac{\partial S}{\partial V}\big|_T}{\cancel{\frac{\partial S}{\partial T}\big|_V}}\\
		=& \frac{\partial E}{\partial V} \bigg|_T  -  \frac{\partial E}{\partial S} \bigg|_V \frac{\partial S}{\partial V}\bigg|_T
	\end{align*}
\end{proof}

NB. For most situations, S is a monotonic increasing function of E. As E increases, distribution gets broader, and S increases. 

Imagine piston is vertical.
\begin{itemize}
	\item Move piston out, gas does work, which is paid for from energy.
	\item Move piston in, we do work, which adds to energy.
\end{itemize}

Move piston slowly, and don't let energy in. Adiabatic: slowly, and no heat in or out. $dE = - P dV$

\begin{align*}
\frac{\partial E}{\partial V}\bigg|_S =& - P\numberthis \label{eq:defP}
\end{align*}


\begin{thm}[Adiabatic Theorem]\label{thm:adiabatic}
	Change volume adiabatically, the system will ride along the energy level: it will change, but not jump.
\end{thm}

One consequence of Theorem \ref{thm:adiabatic} is that the $p_i$ are fixed.

Isentropic is also adiabatic.

Easier to use T as it appears in Bolzmann.

\begin{align*}
P =& - \frac{\partial E}{\partial V}\bigg|_S \\
=& - \frac{\partial E}{\partial V}\bigg|_T + \frac{\partial S}{\partial V}\bigg|_T \frac{\partial E}{\partial S}\bigg|_V\text{, from Lemma \ref{thm:derivative:contour}}\\
=& - \frac{\partial E}{\partial V}\bigg|_T + \frac{\partial S}{\partial V}\bigg|_T T\text{, from (\ref{eq:T})}\\
=& - \frac{\partial(E-TS)}{\partial V}\bigg|_T \text{, from (\ref{eq:helmholz})}\\
=& T \frac{\partial \log(Z)}{\partial V}\bigg|_T 
\end{align*}

\begin{itemize}
	\item We haven't used assumption that we are dealing with a gas, so this is completely general.
	\item could have used different conjugate pair.
\end{itemize}

The partition function is given by:
\begin{align*}
Z(\beta) =& \int dx dp e^{- \beta \frac{p^2}{2m}}\\
=& \frac{V^N}{N!} f(\beta)\\
\log(Z) =& N \log(V) + \log(f(\beta))\text{, so (\ref{eq:P})  becomes:}\\
P =& \frac{NT}{V}\text{, or}\\
P =& \rho T
\end{align*}

\subsection{Fluctuations}

\begin{align*}
	\langle E -  \langle E \rangle \rangle^2 =& \langle E^2 \rangle - \langle E \rangle^2\text{. Now, from (\ref{eq:E:beta})}\\
	\langle E \rangle=& - \frac{\partial \log(Z)}{\partial \beta}\text{, and from (\ref{eq:dZ})}\\
	 \langle E^2 \rangle =& \frac{1}{Z} \frac{\partial^2 Z}{\partial \beta^2} \text{, whence}\\
	 \langle E^2 \rangle =& \frac{1}{Z} \frac{\partial^2 Z}{\partial \beta^2} - \frac{1}{Z^2} (\frac{\partial Z}{\partial \beta})^2 \numberthis \label{E2Z}
\end{align*}
All of statistical mechanics is about the power of the partition function.

\begin{align*}
	\frac{\partial^2 \log(Z)}{\partial \beta^2} =& \frac{\partial}{\partial \beta} \frac{1}{Z} \frac{\partial Z}{\partial \beta}\\
	=& \frac{1}{Z} \frac{\partial^2 Z}{\partial \beta^2} - \frac{1}{Z^2} (\frac{\partial Z}{\partial \beta})^2\\
	(\Delta E)^2 =& \frac{\partial}{\partial \beta} \frac{1}{Z} \frac{\partial Z}{\partial \beta}\text{, from (\ref{E2Z})}\\
	=& - \frac{\partial}{\partial \beta} \langle E \rangle\\
	=&  \underbrace{\frac{\partial}{\partial T} \langle E \rangle}_{C_V=^\text{ heat capacity}} T^2\\
	=& C_V T^2
\end{align*}

\begin{itemize}
	\item Einstein? Gibbs?
	\item True for any system: we haven't assumed ideal gas.
\end{itemize}


\section{Weakly interacting gases, heat, and work}


\section{Entropy vs. reversibility}


\section{Entropy, reversibility, and magnetism}


\section{The Ising model}


\section{Liquid-gas phase transition}

\end{document}
