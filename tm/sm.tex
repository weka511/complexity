\documentclass[]{article}
\usepackage{caption,subcaption,graphicx,float,url,amsmath,amssymb,amsthm,tocloft,cancel}
\newcommand\numberthis{\addtocounter{equation}{1}\tag{\theequation}}
\newtheorem{thm}{Theorem}

%opening
\title{Statistical Mechanics}
\author{Simon Crase}

\begin{document}

\maketitle

\begin{abstract}
These are my notes from the Statistical Mechanics lectures from Leonard Susskind's Theoretical Minimum series.
\end{abstract}

\tableofcontents

\section{Entropy and conservation of information}
This lecture focuses on the law of conservation of energy, the $-1$law of physics. Liouville's theorem.

''In physics, Liouville's theorem, named after the French mathematician Joseph Liouville, is a key theorem in classical statistical and Hamiltonian mechanics. It asserts that the phase-space distribution function is constant along the trajectories of the system—that is that the density of system points in the vicinity of a given system point traveling through phase-space is constant with time. This time-independent density is in statistical mechanics known as the classical a priori probability.''

\begin{align*}
S \triangleq \log(M) \numberthis \label{eq:entropy:simple}
\end{align*}

Proper laws of physics are reversible and therefore preserve the distinctions between states - i.e. information.  In this sense, the conservation of information is more fundamental that other physical quantities such as temperature or energy.  
\begin{itemize}
	\item $S$ conserved as long as we can follow in detail, e.g. if we can follow the motion of each molecule that is disturbed by Friction.
	\item $S$ increases as our ignorance goes up;
	\item $S$ depends on the system and our state of knowledge.
\end{itemize}

\begin{align*}
S \triangleq - \sum_{i=1}^{N} p_i \log{p_i} \numberthis \label{eq:entropy}
\end{align*}


In (\ref{eq:entropy:simple}) and (\ref{eq:entropy}) we use the logarithm so that entropy adds, rather than multiplies.

The First law of Thermodynamics is energy conservation. For a closed system:
\begin{align*}
\frac{dE}{dt} =& 0\\
\sum \frac{dE_i}{dt} =& 0
\end{align*}


\section{Temperature}

Temperature is really a measure of Energy. So the energy of one molecule of an Ideal Gas is given by:
\begin{align*}
E =& \frac{3}{2} k_B t\text{, where:}\\
k_B =& \text{Boltzmann's constant}\\
T =& \text{temperature in Kelvin}
\end{align*}
 In natural units we define a new unit of temperature, $T=k_Bt$, so $k_B=1$.
 
 We want to redefine temperature in therms of Entropy and Energy, and then show that this new definition agrees with our everyday idea of temperature. Temperature is not a fundamental quantity, but is derived as the amount of energy required to add an incremental amount of entropy to a system.  As the energy of a system increases, the number of possible states of a system increases, which means that the entropy increases.  This is the concept behind the second law of thermodynamics, and implies that temperature is always positive.
 
 Our definition reduces to:
\begin{align*}
\Delta E =& \underbrace{\frac{\partial E}{\partial S}}_\text{1 Degree of T} \Delta S\text{,or }\\
dE =& T dS \numberthis \label{eq:T}
\end{align*}

Consider two systems, $A$ and $B$, joined by a narrow pipe. Theorem \ref{thm:heat:flow} shows that T captures our intuition about temperature: heat flows from a higher temperature to lower.

\begin{thm}\label{thm:heat:flow}
	Given:
	\begin{align*}
	dE_A + dE_B =& 0 \numberthis \label{eq:thm:dE}\\
	dS_A + dS_B>& 0 \numberthis \label{eq:thm:eS} \\
	dE_i =& T_i dS_i \numberthis \label{eq:thm:T} \\
	T_A>&T_B \numberthis \label{eq:thm:Tgt}
	\end{align*}
	then energy flows from A to B, i.e.
	\begin{align*}
	dE_A<0
	\end{align*}
\end{thm}
\begin{proof}
	\begin{align*}
	T_A dS_A =& dE_A \text{, from (\ref{eq:thm:T})}\\
	=& - dE_B \text{, from (\ref{eq:thm:dE})}\\
	=& - T_B dS_B  \text{, from (\ref{eq:thm:T})}\\
	\implies&\\
	T_A dS_A + T_A dS_B =& T_A dS_B -T_B dS_B\\
	\implies&\\
	T_A (dS_A + dS_B) =& (T_A  -T_B) dS_B\\
	\implies& \text{, using (\ref{eq:thm:eS}) and (\ref{eq:thm:Tgt})}\\
	dS_B >& 0\\
	\implies& \text{using (\ref{eq:thm:T})}\\
	dE_B >& 0\\
	\implies& \text{, using (\ref{eq:thm:dE})}\\
	dE_A<0
	\end{align*}
\end{proof}



\section{Maximizing entropy}


Number of arrangements $\frac{N!}{\prod_{i}n_i!}$. We want to maximize this subject to two constraints:
\begin{align*}
\sum_{i=1}^{N} n_i =& N \numberthis\label{eq:sum:probilities}\\
\sum_{i=1}^{N} E_i n_i =& E N \numberthis\label{eq:total:energy}
\end{align*}

We use Stirling's approximation

\begin{align*}
\log(N!) =& \sum_{i=1}^{N}\log (i)\\
\approxeq &\int_{1}^{N} \log(x) dx\\
\approxeq& \big[x \log(x) - x \big]_1^N\\
\approxeq& N\log(N)-N\\
N! \approxeq& N^N e^{-N}\numberthis\label{eq:stirling}
\end{align*}

\begin{align*}
C=&\frac{N!}{\prod_{i}n_i!}\\
 \approxeq& \frac{N^N e^{-N}}{\prod_{i}n_i^{n_i} e^{-\sum_{i=1}^{N}n_i}}\tag*{from (\ref{eq:stirling})}\\
\approxeq& \frac{N^N \bcancel{e^{-N}}}{\prod_{i}n_i^{n_i} \bcancel{e^{-N}}}\tag*{using (\ref{eq:sum:probilities})}\\
\log(C) \approxeq & N \log(N) - \sum_{i=1}^{N} n_i \log(n_i)\\
\approxeq& N \log(N) - \sum_{i=1}^{N} N p_i \log(N p_i)\\
\approxeq& N \log(N) - \sum_{i=1}^{N} N p_i \big(\log(N) + \log(p_i))\big)\\
\approxeq& N \log(N) -  N \log(N) \bcancel{\sum_{i=1}^{N} p_i}  - N \sum_{i=1}^{N}p_i \log(p_i)\tag*{using (\ref{eq:sum:probilities})}\\
\approxeq& \bcancel{N \log(N)} - \bcancel{N \log(N)}  - N \sum_{i=1}^{N}p_i \log(p_i)\\
\approxeq& NS \text{, using (\ref{eq:entropy})} \numberthis \label{eq:entropy:logC}
\end{align*}

\section{The Boltzmann distribution}

Minimize -S, using Lagrange Multipliers, i.e., minimize
\begin{align*}
\sum_{i=1}^{N}p_i \log(p_i) + \alpha(\sum_{i=1}^{N} p_i-1) + \beta(\sum_{i=1}^{N} E_i p_i-E)
\end{align*}

Differentiating with respect to $p_i$:
\begin{align*}
\log(p_i)+1 + \alpha +\beta E_i=&0\text{, whence}\\
p_i =& e^{-(1+\alpha)} e^{-\beta E_i}\\
=& \frac{1}{Z} e^{-\beta E_i}\text{, where $Z\triangleq e^{(1+\alpha)}$} \numberthis\label{eq:maximize:entropy}
\end{align*}

From (\ref{eq:sum:probilities}) and (\ref{eq:maximize:entropy})
\begin{align*}
1 =& \frac{1}{Z}  \sum_{i=1}^{N} e^-{\beta E_i}\\
Z(\beta) =& \sum_{i=1}^{N} e^-{\beta E_i} \text{. $Z$ is known as the Partition Function.} \numberthis \label{eq:partition:function}
\end{align*}
We will see that $\beta$ is the inverse temperature.

From (\ref{eq:total:energy}) and (\ref{eq:maximize:entropy})
\begin{align*}
E =& \frac{1}{Z}  \sum_{i=1}^{N} e^{-\beta E_i} E_i \numberthis \label{eq:E}\\
\frac{\partial Z}{\partial \beta} =& - \sum_{i=1}^{N} e^{-\beta E_i} E_i \numberthis \label{eq:dZ}\\
E(\beta) =& - \frac{1}{Z} \frac{\partial Z}{\partial \beta} = - \frac{\partial \log(Z)}{\partial \beta} \numberthis \label{eq:E:beta}
\end{align*}

\begin{align*}
S =& -\sum_{i=1}^{N} p_i \log(p_i)\\
=& - \sum_{i=1}^{N} \frac{1}{Z} e^{- \beta E_i}\big[- \beta E_i -\log(Z)\big]\\
=& \beta E +  \bcancel{\frac{1}{Z}} \log(Z)  \bcancel{\sum_{i=1}^{N} e^{-\beta E_i}}\\
=& \beta E +   \log(Z)  \numberthis \label{eq:S_Z}
\end{align*}

\begin{align*}
dS =& \beta dE + E d\beta + \frac{\partial \log(Z)}{\partial \beta} d\beta \text{, from (\ref{eq:S_Z})}\\
=& \beta dE \text{, using (\ref{eq:E:beta})}\\
=& \frac{1}{T} dE \text{, using (\ref{eq:T})}\text{, whence}\\
T =& \frac{1}{\beta}
\end{align*}
Ideal Gas: assume molecules don't interact.
State: 3N coordinates $\{X_1,...X_{3N}\}$ and momenta $\{P_1,...P3N\}$
\begin{align*}
Z =& \int d^{3N}x d^{3N}p . e^{- \frac{\beta}{2m} \sum_{n=1}^{3N} p_n^2}\\
=& \frac{V^N}{N!} \int d^{3N}p . e^{- \frac{\beta}{2m} \sum_{n=1}^{3N} p_n^2}\text{Factorial is controversial}\\
=& \frac{V^N}{N!} \big[\int dp. e^{-\frac{\beta}{2m}p^2}\big]^{3N}\\
=& \frac{V^N}{N!} \big[\frac{2 m \pi}{\beta}\big]^{\frac{3N}{2}}\numberthis\label{eq:Z_beta}
\end{align*}
We now use Stirling's formula (\ref{eq:stirling}).
\begin{align*}
\frac{V^N}{N!} \approxeq& \big(\frac{eN}{N}\big)^N \\
=& \big(\frac{e}{\rho}\big)^N,\text{, and (\ref{eq:E:beta}) becomes}\\
Z \approxeq& \big(\frac{e}{\rho}\big)^N .  \big[\frac{2 m \pi}{\beta}\big]^{\frac{3N}{2}}
\end{align*}

\begin{align*}
\log(Z) =& - \frac{3N}{2} \log(\beta) + const\\
E =& - \frac{\partial \log(Z)}{\partial \beta}\\
=& \frac{3N}{2} T
\end{align*}
\section{Pressure of an ideal gas and fluctuations}
\section{Weakly interacting gases, heat, and work}
\section{Entropy vs. reversibility}
\section{Entropy, reversibility, and magnetism}
\section{The Ising model}
\section{Liquid-gas phase transition}

\end{document}
