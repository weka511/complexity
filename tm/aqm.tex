\documentclass[]{article}

\usepackage{caption,subcaption,graphicx,float,url,amsmath,amssymb,amsthm,tocloft,cancel,thmtools,gensymb,braket}
\usepackage[toc,nonumberlist]{glossaries}
\usepackage{glossaries-extra}
\newcommand\numberthis{\addtocounter{equation}{1}\tag{\theequation}}
\newtheorem{defn}{Definition}
\newtheorem{thm}{Theorem}
\newtheorem{lemma}[thm]{Lemma}
\graphicspath{{figs/}}
\widowpenalty10000
\clubpenalty10000

%opening
\title{Theoretical Minimum\\Advanced Quantum Mechanics}
\author{Simon Crase}

\begin{document}

\maketitle

\begin{abstract}
These are my notes from the Advanced Quantum Mechanics lectures from Leonard Susskind's Theoretical Minimum series.
\end{abstract}

\tableofcontents

\section{Review and Introduction to Symmetry}

\subsection{Review of Quantum Mechanics}

\begin{itemize}
	\item States: $\ket{\Psi}$ and $\bra{\Psi}$
	\item Observables: $A=A^{\dag}$
	\item Eigenvalues represent values that can be measured: $A\ket{\alpha} = \alpha\ket{\alpha}$
	\item Inner product $\braket{\phi|\psi}$	
	\item Orthogonal $\braket{\phi|\psi}=0$
	\item Independent values $\braket{x|x^{\prime}}=0$
	\item Wave function $\braket{x|\psi} = \psi(x)$
	\item Probability $P(x)=\psi^*(x)\psi(x)$
	\item Momentum $p\psi(x)=- i \hslash \frac{\partial \Psi}{\partial x}$
	\item Evolution $U(t)-- \ket{\psi(0)} = \ket{\psi(t)}$
	\item $\braket{\phi|\psi}=\braket{\phi|U^{\dagger}U|\psi}$ i.e. $U^{\dagger}U=I$
	\item $U(\epsilon)= I + \epsilon G$, so $G=-G^{\dagger}$, $G=- i H$
	\item For general $t$, $U(t)=e^{- i H t}$
\end{itemize}

\begin{align*}
=& e^{-i \hslash \epsilon} \ket{\psi(t)}\\
=& (1- i \hslash \epsilon)\ket{\psi(t)}\\
\frac{\ket{\psi(t+\epsilon)} -\ket{\psi(t)} }{\epsilon}&= -i \hslash \ket{\psi(t)}\\
\frac{\partial \ket{\psi}}{\partial t} =& -i H \ket{\psi(t)} \text{, Time dependent Schr\"odinger Equation}\\
H \ket{\psi} =& E \ket{\psi}\text{, Time independent Schr\"odinger Equation}
\end{align*} 


\subsection{Introduction to Symmetry}

Rotational symmetry is one of the most important symmetries. So is translational symmetry. Transformation is a symmetry that doesn't change equations.

\begin{align*}
V \ket{\psi} =& \ket{\psi^{\prime}} \text{, symmetry operation -- unitary $V$}\\
\ket{\psi_1} \xrightarrow{U}& \ket{\psi_2} \text {, time evolution}\\
\ket{\psi_1^{\prime}} \xrightarrow{U}& \ket{\psi_2^{\prime}} \text {, if $V$ really is a symmetry. Now}\\
V\ket{\psi_2} =& U V \ket{\psi_1} \text{, and}\\
V U\ket {\psi_1} =& U V \ket{\psi_1} \\
V U =& U V \\
V H =& H V\\
[H,V] =& 0
\end{align*}

Symmetry is unitary operator that commutes with Hamiltonian.

\begin{itemize}
	\item Discrete
	\begin{itemize}
		\item Reflection
		\item interchange particles
	\end{itemize}
	\item Continuous
	\begin{itemize}
		\item rotation
		\item translation
	\end{itemize}
\end{itemize}

All continuous symmetries can be generated by $I-i \epsilon G$, for some Hermitean $G$ -- $[H,G]=0$.

E.g.
\begin{align*}
V \psi(x) = & \psi(x-\epsilon)\text{, shift right}\\
=& \psi(x) - \epsilon \frac{\partial \psi}{\partial x}\\
V =& I -  \epsilon \frac{\partial }{\partial x}\\
=& I - \frac{i \epsilon}{\hslash}P\\
G =& \frac{P_x}{\hslash}\text{, Generator of $x$ translation}
\end{align*}


\section{Symmetry groups and degeneracy}

Crystal symmetry: translate one lattice spacing.

Degeneracy of energy levels: more than one state with given energy. Only happens when there is a symmetry.

Rotation.

\begin{align*}
\psi(\theta) \rightarrow & \psi(\theta - \epsilon)\\
\delta\psi =& - \epsilon \frac{\partial \psi}{\partial \theta}\\
=& -i \epsilon \big(-i \frac{\partial \psi}{\partial \theta}\big).\text{ Now defining the Angular Momentum operator $L$}\\
- i \frac{\partial}{\partial \theta} =&\hslash L \\
\delta\psi =& - \frac{i \epsilon}{\hslash} L \psi \text{, generator of rotation}
\end{align*}

Eigenvalues and Eigenvectors.

\begin{align*}
L\ket{\psi} =& l \ket{\psi}\\
-i \hslash \frac{\partial \psi}{\partial \theta} =& \psi\\
\psi(\theta) =& e^{\frac{i m \theta}{\hslash}}\text{Now, we want $\psi$ single valued, so}\\
\frac{m}{\hslash}=&k\text{, some integer. But, by convention, redefine $m$}\\
L=& m \hslash
\end{align*}
This is the quantization of angular momentum.

$n\ne 0 \implies E(m)=E(-m)$, so we have degeneracy. A magnetic field breaks this; symmetry isn't enough for degeneracy, but adding reflection symmetry is sufficient (but need two non-commuting symmetries).

\begin{thm}[Reflection and rotation don't commute]
	If $M\psi(\theta) = \psi(-theta)$, $ML \ne LM$
\end{thm}
\begin{proof}
	\begin{align*}
	MLe^{i m \theta} =& M m e^{i m \theta}\\
	=&m e^{- i m \theta}\\
	LMe^{i m \theta} =& L e^{- i m \theta}\\
	=& -m e^{- i m \theta}
	\end{align*}
\end{proof}

\begin{align*}
[L,H]=& 0\\
[L_x,L_y] =& L_z\\
[L_y,L_z] =& L_x\\
[L_z,L_x] =& L_y\\
L_{\pm} \triangleq& L_x \pm i L_y\\
[L_{\pm},L_Z] =& \mp L_{\pm}
\end{align*}
Suppose we have found one eigenvector:
\begin{align*}
L_z \ket{m} =& m \ket{m}\\
[L_+,L_z] \ket{m} =& \big(L_+L_z - L_z,L_+\big) \ket{m}\\
=&- L_+ \ket{m}\\
 m L_+\ket{m} + L_+\ket{m} =& L_z L_+ \ket{m} \text{, whence}\\
 \big(m + 1 \big)L_+\ket{m}  =& L_z L_+ \ket{m} \text{, i.e. $L_+ \ket{m}$ is an eigenvector, eigenvalue $m+1$.}\numberthis \label{eq:create_m}
\end{align*}

Similarly, $L_- \ket{m}$ is an eigenvector, eigenvalue $m-1$. This terminates if  $L_{\pm} \ket{m}=0$. From symmetry we can show $m$ integral or half integral.

\begin{thm}[Eigenvectors have same energy.]
	 $H \ket{m} = E \ket{m} \implies H \ket{m \pm 1} = E \ket{m \pm 1}$
\end{thm} 
\begin{proof}
	\begin{align*}
	H \ket{m} =& E \ket{m}\\
	H L_+ \ket{m} =& L_+ H \ket{m}\\
	=& L_+ E \ket{m}\\
	H \ket{m+1} =& E \ket{m+1}
	\end{align*}
\end{proof}

Since symmetries commute, we have degeneracy.

\section{Atomic orbits and harmonic oscillators}

\subsection{Atomic orbits}

A particle moving in a central force field: angular momentum and orbital plane preserved. State is $\psi(r,\theta,\phi)= \psi(r,\theta)$, and $L = -i \frac{\partial}{\partial \theta}$.

\begin{align*}
-i \frac{\partial \psi(r,\theta)}{\partial \theta} =& l \psi(r,\theta) \text{, eigenvector}\\
\psi(r,\theta) =& e^{i l \theta} \chi(r) \text{, for some $\chi$}
\end{align*}
More generally, $\psi(r,\theta,\phi)= Y(\theta,\phi) \chi(r)$


From (\ref{eq:create_m}) we have a spectrum of angular momenta, integral or half integral, say $\set{-l,...,l:L_+\ket{l}=0}$. There are $2l+1$ states with constant $L^2 = L_x^2 + L_y^2 + L_x^2$. Classically $L^2 = L_z^2 +(L_x-iL_y)(L_x+iL_y)$, but this fails in quantum mechanics as they operators don't commute.

\begin{align*}
L^2 =& L_z^2 +(L_x-iL_y)(L_x+iL_y) -i [L_x,L_y]\\
=& L_z^2 + L_z + L^-L^+\\
L^2\ket{l}=& L_z^2\ket{l} + L_z\ket{l} + L^-L^+\ket{l}\\
=& l^2\ket{l} + l\ket{l} + 0\\
L^2\ket{l}=&l (l+1) \ket{l}
\end{align*}

\begin{thm}[All eigenvectors of $L_z$ are degenerate eigenvectors of $L^2$]
	\begin{enumerate}
		\item $[L^2, L_i] =0$
		\item All eigenvectors of $L_z$ are eigenvectors of $L^2$, eigenvalue $l(L+1)$
	\end{enumerate}
\end{thm}
\begin{proof}
	Exercise
\end{proof}

Classically:
\begin{align*}
H =& \frac{p^2}{2m} + V(r) \text{, conserved} \numberthis \label{eq:classical:Hamiltonian}\\
L =& r \times P  \text{, conserved, so use xy plane}\\
H =& \frac{p_r^2+p_{\theta}^2}{2m} +V(r)\\
=& \frac{P_r^2}{2m} + \frac{L^2}{2m r^2} + V(r) 
\end{align*}

Schr\"odinger Equation:
\begin{align*}
-\frac{\hslash^2}{2m}\frac{\partial^2 \psi(r)}{\partial r^2} + \frac{l(l+1)\hslash^2}{r^2}\psi(r)+V(r)\psi(r) =& E\psi(r)\numberthis \label{eq:schroedinger:central}
\end{align*}

Number of nodes (zeroes) characterize energy levels.

Classical mechanics inspires choice of Hamiltonian(\ref{eq:classical:Hamiltonian}), but Schr\"odinger equation (\ref{eq:schroedinger:central}) is quantum.

\subsection{Harmonic oscillators}

Everything in physic that disturbs equilibrium by a small amount can be approximated by a simple harmonic oscillator. 

Start with suspended mass $(m=1)$, spring constant $(k=\omega^2)$

\begin{align*}
H =& \frac{P^2}{2m} + \frac{\omega^2 x^2}{2}\\
=& \frac{\omega}{2 \omega}\big(P + i \omega x \big)\big(P - i \omega x \big) - \frac{i \omega}{2} [x,P] \text{, Dirac!}\\
=& \frac{P^2}{2m} + \frac{\omega^2 x^2}{2}\\
=& \frac{\omega}{2 \omega}\big(P + i \omega x \big)\big(P - i \omega x \big) + \underbrace{\frac{\hslash \omega}{2}}_\text{ground state energy}\\
=&\omega \frac{\big(P + i \omega x \big)}{\sqrt{2 \omega}}\frac{\big(P - i \omega x \big)}{\sqrt{2 \omega}}\text{, we'll drop the ground state energy fot the time being}
\end{align*}

We'll introduce raising and lowering operators:

\begin{align*}
a^+ \triangleq & \frac{P + i \omega x } {\sqrt{2 \omega}} \\
a^- \triangleq & \frac{P - i \omega x } {\sqrt{2 \omega}}\text{, Hermitean conjugate} \\
H =& \omega a^+ a^-\text{. We'll take the commutator:}\numberthis \label{eq:shM}\\
[a^-,a^+] =& \frac{1}{2\omega}\big[P - i \omega x,P + i \omega x\big]\\
=& 1\text{. We also define} \numberthis \label{eq:a:comm}\\
N \triangleq& a^+a^-\text{, so (\ref{eq:shM}) becomes}\\
H =& \omega N
\end{align*}
$N$ is Hermitean, so it has a complete set of eigenvalues and eigenvectors.

\begin{align*}
N\ket{n} =& n\ket{n}\\
a^+a^-\ket{n} =& n\ket{n}\\
a^+(a^-a^+ - a^+a^-)\ket{n} =& a^+\ket{n}\text{, using (\ref{eq:a:comm})}\\
a^+a^-a^+ \ket{n} =& a^+a^+a^-\ket{n} + a^+\ket{n}\\
N a^+ \ket{n} =& (n+1) a^+ \ket{n}
\end{align*}

So $a^+$ acts as raising operator, and we can show $a^-$ is lowering. Also $a^-$ eventually leads to $0$.

\begin{align*}
a^+\ket{n} =& \sqrt{n+1}\ket{n+1}\\
a^-\ket{n} =& \sqrt{n}\ket{n-1}
\end{align*}
 
\section{Spin}


\section{Fermions: a tale of two minus signs}



\section{Quantum Field Theory}

\section{Quantum Field Theory II}

\section {Second Quantization}

\section{Quantum field Hamiltonian}

\section{Fermions and the Dirac equation}

\end{document}
