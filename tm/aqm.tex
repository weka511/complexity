\documentclass[]{article}

\usepackage{caption,subcaption,graphicx,float,url,amsmath,amssymb,amsthm,tocloft,cancel,thmtools,gensymb,braket}
\usepackage[toc,nonumberlist]{glossaries}
\usepackage{glossaries-extra}
\newcommand\numberthis{\addtocounter{equation}{1}\tag{\theequation}}
\newtheorem{defn}{Definition}
\newtheorem{thm}{Theorem}
\newtheorem{lemma}[thm]{Lemma}
\graphicspath{{figs/}}
\widowpenalty10000
\clubpenalty10000

%opening
\title{Theoretical Minimum\\Advanced Quantum Mechanics}
\author{Simon Crase}

\begin{document}

\maketitle

\begin{abstract}
These are my notes from the Advanced Quantum Mechanics lectures from Leonard Susskind's Theoretical Minimum series.
\end{abstract}

\tableofcontents

\section{Review and Introduction to Symmetry}

\subsection{Review of Quantum Mechanics}

\begin{itemize}
	\item States: $\ket{\Psi}$ and $\bra{\Psi}$
	\item Observables: $A=A^{\dag}$
	\item Eigenvalues represent values that can be measured: $A\ket{\alpha} = \alpha\ket{\alpha}$
	\item Inner product $\braket{\phi|\psi}$	
	\item Orthogonal $\braket{\phi|\psi}=0$
	\item Independent values $\braket{x|x^{\prime}}=0$
	\item Wave function $\braket{x|\psi} = \psi(x)$
	\item Probability $P(x)=\psi^*(x)\psi(x)$
	\item Momentum $p\psi(x)=- i \hslash \frac{\partial \Psi}{\partial x}$
	\item Evolution $U(t)-- \ket{\psi(0)} = \ket{\psi(t)}$
	\item $\braket{\phi|\psi}=\braket{\phi|U^{\dagger}U|\psi}$ i.e. $U^{\dagger}U=I$
	\item $U(\epsilon)= I + \epsilon G$, so $G=-G^{\dagger}$, $G=- i H$
	\item For general $t$, $U(t)=e^{- i H t}$
\end{itemize}

\begin{align*}
=& e^{-i \hslash \epsilon} \ket{\psi(t)}\\
=& (1- i \hslash \epsilon)\ket{\psi(t)}\\
\frac{\ket{\psi(t+\epsilon)} -\ket{\psi(t)} }{\epsilon}&= -i \hslash \ket{\psi(t)}\\
\frac{\partial \ket{\psi}}{\partial t} =& -i H \ket{\psi(t)} \text{, Time dependent Schr\"odinger Equation}\\
H \ket{\psi} =& E \ket{\psi}\text{, Time independent Schr\"odinger Equation}
\end{align*} 


\subsection{Introduction to Symmetry}

Rotational symmetry is one of the most important symmetries. So is translational symmetry. Transformation is a symmetry that doesn't change equations.

\begin{align*}
V \ket{\psi} =& \ket{\psi^{\prime}} \text{, symmetry operation -- unitary $V$}\\
\ket{\psi_1} \xrightarrow{U}& \ket{\psi_2} \text {, time evolution}\\
\ket{\psi_1^{\prime}} \xrightarrow{U}& \ket{\psi_2^{\prime}} \text {, if $V$ really is a symmetry. Now}\\
V\ket{\psi_2} =& U V \ket{\psi_1} \text{, and}\\
V U\ket {\psi_1} =& U V \ket{\psi_1} \\
V U =& U V \\
V H =& H V\\
[H,V] =& 0
\end{align*}

Symmetry is unitary operator that commutes with Hamiltonian.

\begin{itemize}
	\item Discrete
	\begin{itemize}
		\item Reflection
		\item interchange particles
	\end{itemize}
	\item Continuous
	\begin{itemize}
		\item rotation
		\item translation
	\end{itemize}
\end{itemize}

All continuous symmetries can be generated by $I-i \epsilon G$, for some Hermitean $G$ -- $[H,G]=0$.

E.g.
\begin{align*}
V \psi(x) = & \psi(x-\epsilon)\text{, shift right}\\
=& \psi(x) - \epsilon \frac{\partial \psi}{\partial x}\\
V =& I -  \epsilon \frac{\partial }{\partial x}\\
=& I - \frac{i \epsilon}{\hslash}P\\
G =& \frac{P_x}{\hslash}\text{, Generator of $x$ translation}
\end{align*}


\section{Symmetry groups and degeneracy}

Crystal symmetry: translate one lattice spacing.

Degeneracy of energy levels: more than one state with given energy. Only happens when there is a symmetry.

Rotation.

\begin{align*}
\psi(\theta) \rightarrow & \psi(\theta - \epsilon)\\
\delta\psi =& - \epsilon \frac{\partial \psi}{\partial \theta}\\
=& -i \epsilon \big(-i \frac{\partial \psi}{\partial \theta}\big).\text{ Now defining the Angular Momentum operator $L$}\\
- i \frac{\partial}{\partial \theta} =&\hslash L \\
\delta\psi =& - \frac{i \epsilon}{\hslash} L \psi \text{, generator of rotation}
\end{align*}

Eigenvalues and Eigenvectors.

\begin{align*}
L\ket{\psi} =& l \ket{\psi}\\
-i \hslash \frac{\partial \psi}{\partial \theta} =& \psi\\
\psi(\theta) =& e^{\frac{i m \theta}{\hslash}}\text{Now, we want $\psi$ single valued, so}\\
\frac{m}{\hslash}=&k\text{, some integer. But, by convention, redefine $m$}\\
L=& m \hslash
\end{align*}
This is the quantization of angular momentum.

$n\ne 0 \implies E(m)=E(-m)$, so we have degeneracy. A magnetic field breaks this; symmetry isn't enough for degeneracy, but adding reflection symmetry is sufficient (but need two non-commuting symmetries).

\begin{thm}[Reflection and rotation don't commute]
	If $M\psi(\theta) = \psi(-theta)$, $ML \ne LM$
\end{thm}
\begin{proof}
	\begin{align*}
	MLe^{i m \theta} =& M m e^{i m \theta}\\
	=&m e^{- i m \theta}\\
	LMe^{i m \theta} =& L e^{- i m \theta}\\
	=& -m e^{- i m \theta}
	\end{align*}
\end{proof}

\begin{align*}
[L,H]=& 0\\
[L_x,L_y] =& L_z\\
[L_y,L_z] =& L_x\\
[L_z,L_x] =& L_y\\
L_{\pm} \triangleq& L_x \pm i L_y\\
[L_{\pm},L_Z] =& \mp L_{\pm}
\end{align*}
Suppose we have found one eigenvector:
\begin{align*}
L_z \ket{m} =& m \ket{m}\\
[L_+,L_z] \ket{m} =& \big(L_+L_z - L_z,L_+\big) \ket{m}\\
=&- L_+ \ket{m}\\
 m L_+\ket{m} + L_+\ket{m} =& L_z L_+ \ket{m} \text{, whence}\\
 \big(m + 1 \big)L_+\ket{m}  =& L_z L_+ \ket{m} \text{, i.e. $L_+ \ket{m}$ is an eigenvector, eigenvalue $m+1$.}
\end{align*}

Similarly, $L_- \ket{m}$ is an eigenvector, eigenvalue $m-1$. This terminates if  $L_{\pm} \ket{m}=0$. From symmetry we can show $m$ integral of half integral.

\begin{thm}[Eigenvectors have same energy.]
	 $H \ket{m} = E \ket{m} \implies H \ket{m \pm 1} = E \ket{m \pm 1}$
\end{thm} 
\begin{proof}
	\begin{align*}
	H \ket{m} =& E \ket{m}\\
	H L_+ \ket{m} =& L_+ H \ket{m}\\
	=& L_+ E \ket{m}\\
	H \ket{m+1} =& E \ket{m+1}
	\end{align*}
\end{proof}

Since symmetries don't commute, we have degeneracy.

\section{Atomic orbits and harmonic oscillators}

\section{Spin}

\section{Fermions: a tale of two minus signs}



\section{Quantum Field Theory}

\section{Quantum Field Theory II}

\section {Second Quantization}

\section{Quantum field Hamiltonian}

\section{Fermions and the Dirac equation}

\end{document}
