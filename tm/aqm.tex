\documentclass[]{article}

\usepackage{caption,subcaption,graphicx,float,url,amsmath,amssymb,amsthm,tocloft,cancel,thmtools,gensymb,braket}
\usepackage[toc,nonumberlist]{glossaries}
\usepackage{glossaries-extra}
\newcommand\numberthis{\addtocounter{equation}{1}\tag{\theequation}}
\newtheorem{defn}{Definition}
\newtheorem{thm}{Theorem}
\newtheorem{lemma}[thm]{Lemma}
\graphicspath{{figs/}}
\widowpenalty10000
\clubpenalty10000

%opening
\title{Theoretical Minimum\\Advanced Quantum Mechanics}
\author{Simon Crase}

\begin{document}

\maketitle

\begin{abstract}
These are my notes from the Advanced Quantum Mechanics lectures from Leonard Susskind's Theoretical Minimum series.
\end{abstract}

\tableofcontents

\section{Review and Introduction to Symmetry}

\subsection{Review of Quantum Mechanics}

\begin{itemize}
	\item States: $\ket{\Psi}$ and $\bra{\Psi}$
	\item Observables: $A=A^{\dag}$
	\item Eigenvalues represent values that can be measured: $A\ket{\alpha} = \alpha\ket{\alpha}$
	\item Inner product $\braket{\phi|\psi}$	
	\item Orthogonal $\braket{\phi|\psi}=0$
	\item Independent values $\braket{x|x^{\prime}}=0$
	\item Wave function $\braket{x|\psi} = \psi(x)$
	\item Probability $P(x)=\psi^*(x)\psi(x)$
	\item Momentum $p\psi(x)=- i \hslash \frac{\partial \Psi}{\partial x}$
	\item Evolution $U(t)-- \ket{\psi(0)} = \ket{\psi(t)}$
	\item $\braket{\phi|\psi}=\braket{\phi|U^{\dagger}U|\psi}$ i.e. $U^{\dagger}U=I$
	\item $U(\epsilon)= I + \epsilon G$, so $G=-G^{\dagger}$, $G=- i H$
	\item For general $t$, $U(t)=e^{- i H t}$
\end{itemize}

\begin{align*}
=& e^{-i \hslash \epsilon} \ket{\psi(t)}\\
=& (1- i \hslash \epsilon)\ket{\psi(t)}\\
\frac{\ket{\psi(t+\epsilon)} -\ket{\psi(t)} }{\epsilon}&= -i \hslash \ket{\psi(t)}\\
\frac{\partial \ket{\psi}}{\partial t} =& -i H \ket{\psi(t)} \text{, Time dependent Schr\"odinger Equation}\\
H \ket{\psi} =& E \ket{\psi}\text{, Time independent Schr\"odinger Equation}
\end{align*} 
\subsection{Introduction to Symmetry}

\section{Symmetry groups and degeneracy}

\section{Atomic orbits and harmonic oscillators}

\section{Spin}

\section{Fermions: a tale of two minus signs}



\section{Quantum Field Theory}

\section{Quantum Field Theory II}

\section {Second Quantization}

\section{Quantum field Hamiltonian}

\section{Fermions and the Dirac equation}

\end{document}
