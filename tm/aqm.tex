\documentclass[]{article}

\usepackage{caption,subcaption,graphicx,float,url,amsmath,amssymb,amsthm,tocloft,cancel,thmtools,gensymb,braket}
\usepackage[toc,nonumberlist]{glossaries}
\usepackage{glossaries-extra}
\newcommand\numberthis{\addtocounter{equation}{1}\tag{\theequation}}

\newtheorem{thm}{Theorem}
\newtheorem{defn}[thm]{Definition}
\newtheorem{cor}[thm]{Corollary}
\newtheorem{lemma}[thm]{Lemma}
\graphicspath{{figs/}}
\widowpenalty10000
\clubpenalty10000
\setcounter{tocdepth}{1}

%opening
\title{Theoretical Minimum\\Advanced Quantum Mechanics}
\author{Simon Crase}

\begin{document}

\maketitle

\begin{abstract}
These are my notes from the Advanced Quantum Mechanics lectures from Leonard Susskind's Theoretical Minimum series.
\end{abstract}

\tableofcontents
\listoffigures
\listoftheorems


\section{Review and Introduction to Symmetry}

\subsection{Review of Quantum Mechanics}

\begin{itemize}
	\item States: $\ket{\psi}$ and $\bra{\psi}$
	\item Observables: $A=A^{\dag}$. An observable is represented by a Hermitian matrix.
	\item Eigenvalues represent values that can be measured: $A\ket{\alpha} = \alpha\ket{\alpha}$
	\item Inner product $\braket{\phi|\psi}$	
	\item Orthogonal $\braket{\phi|\psi}=0$
	\item Independent values $\braket{x|x^{\prime}}=0$
	\item Wave function $\braket{x|\psi} = \psi(x)$
	\item Probability $P(x)=\psi^*(x)\psi(x)$
	\item Momentum $p\psi(x)=- i \hslash \frac{\partial \Psi}{\partial x}$
	\item Evolution operator is Unitary $U(t)$: $\ket{\psi(0)} = \ket{\psi(t)}$
	\item $\braket{\phi|\psi}=\braket{\phi|U^{\dagger}U|\psi}$ i.e. $U^{\dagger}U=I$
	\item $U(\epsilon)= I + \epsilon G$, so $G=-G^{\dagger}$, $G=- i H$
	\item For general $t$, $U(t)=e^{- i H t}$
\end{itemize}

\begin{align*}
\psi(t+\epsilon) =& e^{-i \hslash \epsilon} \ket{\psi(t)}\\
=& (1- i \hslash \epsilon)\ket{\psi(t)}\\
\frac{\ket{\psi(t+\epsilon)} -\ket{\psi(t)} }{\epsilon}&= -i \hslash \ket{\psi(t)}\\
\frac{\partial \ket{\psi}}{\partial t} =& -i H \ket{\psi(t)} \text{, Time dependent Schr\"odinger Equation}\\
H \ket{\psi} =& E \ket{\psi}\text{, Time independent Schr\"odinger Equation}
\end{align*} 


\subsection{Introduction to Symmetry}

Rotational symmetry is one of the most important symmetries. So is translational symmetry. A symmetry is a transformation  that doesn't change equations.

\begin{thm}[A symmetry is a unitary operator that commutes with Hamiltonian]
	\begin{align*}
	V \ket{\psi} =& \ket{\psi^{\prime}} \text{is a symmetry}\numberthis \label{eq:symmetry}\\
	\iff&\\
	V H =& H V\numberthis \label{eq:symmetry:commute}
	\end{align*}
\end{thm}

\begin{proof}
	Suppose $\psi_1$ evolves to $\psi_2$:
	\begin{align*}
	\ket{\psi_1} \xrightarrow{U}& \ket{\psi_2} \text {or, equivalently}\\
	\psi_2 =& U \psi_1 \text{. Now, if $V$ really is a symmetry:} \numberthis \label{eq:psi_1}\\
	\ket{\psi_1^{\prime}} \xrightarrow{U}& \ket{\psi_2^{\prime}} \text {,  or}\\
	\psi_2^{\prime} =& U \psi_1^{\prime} \text{, so from (\ref{eq:symmetry}):}\\
	V\ket{\psi_2} =& U V \ket{\psi_1} \text{, or, using (\ref{eq:psi_1})}\\
	V U\ket {\psi_1} =& U V \ket{\psi_1} \text{. But $\ket{\psi_1}$ is an arbitrary state, so} \\
	V U =& U V \text{ i.e. so small time}\\
	V (I - i \epsilon H) =& (I - i \epsilon H) V \text{, whence}\\
	V H =& H V \text{, or}\\
	[H,V] =& 0 \text{, which is (\ref{eq:symmetry:commute})}
	\end{align*}
	Moreover the symmetry $V$ should transform mutually exclusive states into mutually exclusive states, whence it should preserve orthogonality, hence $V$ should be unitary. 
\end{proof}



Symmetries can be discrete or continuous.

\begin{itemize}
	\item Discrete
	\begin{itemize}
		\item Reflection
		\item interchange particles
	\end{itemize}
	\item Continuous
	\begin{itemize}
		\item rotation
		\item translation
	\end{itemize}
\end{itemize}

All continuous symmetries can be generated by $I-i \epsilon G$, for some Hermitean $G$ such that $[H,G]=0$.

E.g.
\begin{align*}
V \psi(x) = & \psi(x-\epsilon)\text{, shift right}\\
=& \psi(x) - \epsilon \frac{\partial \psi}{\partial x}\\
V =& I -  \epsilon \frac{\partial }{\partial x}\\
=& I - \frac{i \epsilon}{\hslash}P\\
G =& \frac{P_x}{\hslash}\text{, Generator of $x$ translation}
\end{align*}


\section{Symmetry groups and degeneracy}

Crystal symmetry: translate one lattice spacing.

\begin{defn}[Degeneracy of energy levels]
	If there is more than one state with given energy level, that energy level is called degenerate. We will see that it only happens when there is a symmetry: symmetries sometimes imply degeneracy, but not always.
\end{defn}
 
Rotation.

\begin{align*}
\psi(\theta) \rightarrow & \psi(\theta - \epsilon)\\
\delta\psi =& - \epsilon \frac{\partial \psi}{\partial \theta}\\
=& -i \epsilon \big(-i \frac{\partial \psi}{\partial \theta}\big).\text{ Now defining the Angular Momentum operator $L$}\\
- i \frac{\partial}{\partial \theta} =&\hslash L \\
\delta\psi =& - \frac{i \epsilon}{\hslash} L \psi \text{, generator of rotation}
\end{align*}

Eigenvalues and Eigenvectors.

\begin{align*}
L\ket{\psi} =& l \ket{\psi}\\
-i \hslash \frac{\partial \psi}{\partial \theta} =& \psi\\
\psi(\theta) =& e^{\frac{i m \theta}{\hslash}}\text{Now, we want $\psi$ single valued, so}\\
\frac{m}{\hslash}=&k\text{, some integer. But, by convention, redefine $m$}\\
L=& m \hslash
\end{align*}
This is the quantization of angular momentum.

$m\ne 0 \implies E(m)=E(-m)$, so we have degeneracy. A magnetic field breaks this; symmetry isn't enough for degeneracy, but adding reflection symmetry is sufficient (but need two non-commuting symmetries).

\begin{thm}[Reflection and rotation don't commute]
	If $M\psi(\theta) = \psi(-\theta)$, $ML \ne LM$
\end{thm}
\begin{proof}
	\begin{align*}
	MLe^{i m \theta} =& M m e^{i m \theta}\\
	=&m e^{- i m \theta}\\
	LMe^{i m \theta} =& L e^{- i m \theta}\\
	=& -m e^{- i m \theta}
	\end{align*}
\end{proof}

\begin{align*}
[L,H]=& 0\\
[L_x,L_y] =& i L_z \numberthis \label{eq:lx_ly}\\
[L_y,L_z] =& i L_x\numberthis \label{eq:ly_lz}\\
[L_z,L_x] =& i L_y\numberthis \label{eq:lz_lx}\\
L_{\pm} \triangleq& L_x \pm i L_y\\
[L_{\pm},L_Z] =& \mp L_{\pm} \numberthis \label{eq:comm:LpmLz}
\end{align*}
Suppose we have found one eigenvector:
\begin{align*}
L_z \ket{m} =& m \ket{m} \numberthis \label{eq:ev:Lz}\\
[L_+,L_z] \ket{m} =& \big(L_+L_z - L_zL_+\big) \ket{m}\\
=&- L_+ \ket{m} \text{, from (\ref{eq:comm:LpmLz}). Rearranging  and using (\ref{eq:ev:Lz})}\\
 m L_+\ket{m} + L_+\ket{m} =& L_z L_+ \ket{m} \text{, whence}\\
 \big(m + 1 \big)L_+\ket{m}  =& L_z L_+ \ket{m} \text{; $L_+ \ket{m}$ is an eigenvector, eigenvalue $m+1$}\numberthis \label{eq:create_m}
\end{align*}

Similarly, $L_- \ket{m}$ is an eigenvector, eigenvalue $m-1$. This terminates if  $L_{\pm} \ket{m}=0$. From symmetry we can show $m$ integral or half integral.

\begin{thm}[Eigenvectors of $L_z$ are degenerate eigenvectors of $H$]
	\begin{align*}
	L_z \ket{m} = m \ket{m} \land& H \ket{m} = E \ket{m}\\
	 \implies&\\
	  H \ket{m \pm 1} =& E \ket{m \pm 1}
	\end{align*}
\end{thm} 
\begin{proof}
	\begin{align*}
	H \ket{m} =& E \ket{m} \numberthis \label{eq:assume_ev}\\
	H L_+ \ket{m} =& L_+ H \ket{m}\\
	=& L_+ E \ket{m} \text{, from (\ref{eq:assume_ev})}\\
	=& E L_+  \ket{m}\\
	H \ket{m+1} =& E \ket{m+1} \text{, from (\ref{eq:create_m})}
	\end{align*}
\end{proof}

Since symmetries commute, we have degeneracy.

\section{Atomic orbits and harmonic oscillators}

\subsection{Atomic orbits}

If a particle moves in a central force field, angular momentum and orbital plane are preserved. State is $\psi(r,\theta,\phi)= \psi(r,\theta)$, and $L = -i \frac{\partial}{\partial \theta}$.

\begin{align*}
-i \frac{\partial \psi(r,\theta)}{\partial \theta} =& l \psi(r,\theta) \text{, eigenvector}\\
\psi(r,\theta) =& e^{i l \theta} \chi(r) \text{, for some $\chi$}
\end{align*}
More generally, $\psi(r,\theta,\phi)= Y(\theta,\phi) \chi(r)$


From (\ref{eq:create_m}) we have a spectrum of angular momenta, integral or half integral, say $\set{-l,...,l:L_+\ket{l}=0}$. There are $2l+1$ states with constant $L^2 = L_x^2 + L_y^2 + L_x^2$. Classically $L^2 = L_z^2 +(L_x-iL_y)(L_x+iL_y)$, but this fails in quantum mechanics as they operators don't commute.

\begin{align*}
L^2 =& L_z^2 +(L_x-iL_y)(L_x+iL_y) -i [L_x,L_y]\\
=& L_z^2 + L_z + L^-L^+\\
L^2\ket{l}=& L_z^2\ket{l} + L_z\ket{l} + L^-L^+\ket{l}\\
=& l^2\ket{l} + l\ket{l} + 0\text{, because eigenvectors.}\\
L^2\ket{l}=&l (l+1) \ket{l}
\end{align*}

\begin{thm}[All eigenvectors of $L_z$ are degenerate eigenvectors of $L^2$]
	\begin{enumerate}
		\item $[L^2, L_i] =0$
		\item All eigenvectors of $L_z$ are eigenvectors of $L^2$, eigenvalue $l(L+1)$
	\end{enumerate}
\end{thm}
\begin{proof}
	Exercise
\end{proof}

\begin{figure}[H]
	\caption{Degeneracy of Energy Levels in Hydrogen Atom}\label{fig:degeneracy:hydrogen}
	\includegraphics[width=0.9\textwidth]{hydrogen-degeneracy}
\end{figure}
Classically:
\begin{align*}
H =& \frac{p^2}{2m} + V(r) \text{, conserved} \numberthis \label{eq:classical:Hamiltonian}\\
L =& r \times P  \text{, conserved, so use xy plane}\\
H =& \frac{p_r^2+p_{\theta}^2}{2m} +V(r)\\
=& \frac{P_r^2}{2m} + \frac{L^2}{2m r^2} + V(r) 
\end{align*}

Schr\"odinger Equation:
\begin{align*}
-\frac{\hslash^2}{2m}\frac{\partial^2 \psi(r)}{\partial r^2} + \frac{l(l+1)\hslash^2}{r^2}\psi(r)+V(r)\psi(r) =& E\psi(r)\numberthis \label{eq:schroedinger:central}
\end{align*}

Number of nodes (zeroes) characterize energy levels.

Classical mechanics inspires choice of Hamiltonian(\ref{eq:classical:Hamiltonian}), but Schr\"odinger equation (\ref{eq:schroedinger:central}) is quantum.

\subsection{Harmonic oscillators}

Everything in physic that disturbs equilibrium by a small amount can be approximated by a simple harmonic oscillator. 

Start with suspended mass $(m=1)$, spring constant $(k=\omega^2)$

\begin{align*}
H =& \frac{P^2}{2m} + \frac{\omega^2 x^2}{2}\\
=& \frac{\omega}{2 \omega}\big(P + i \omega x \big)\big(P - i \omega x \big) - \frac{i \omega}{2} [x,P] \text{, Dirac!}\\
=& \frac{\omega}{2 \omega}\big(P + i \omega x \big)\big(P - i \omega x \big) + \underbrace{\frac{\hslash \omega}{2}}_\text{ground state energy}\\
=&\omega \frac{\big(P + i \omega x \big)}{\sqrt{2 \omega}}\frac{\big(P - i \omega x \big)}{\sqrt{2 \omega}}\text{, we'll drop the ground state energy fot the time being.}
\end{align*}

We'll introduce raising and lowering operators:

\begin{align*}
a^+ \triangleq & \frac{P + i \omega x } {\sqrt{2 \omega}} \numberthis \label{eq:creation:operator}\\
a^- \triangleq & \frac{P - i \omega x } {\sqrt{2 \omega}}\text{, Hermitean conjugate} \numberthis \label{eq:annihilation:operator}\\
H =& \omega a^+ a^-\text{. We'll take the commutator:}\numberthis \label{eq:shM}\\
[a^-,a^+] =& \frac{1}{2\omega}\big[P - i \omega x,P + i \omega x\big]\\
=& 1\text{. We also define} \numberthis \label{eq:a:comm}\\
N \triangleq& a^+a^-\text{, so (\ref{eq:shM}) becomes} \numberthis \label{eq:number:operator}\\
H =& \omega N
\end{align*}
$N$ is Hermitean, so it has a complete set of eigenvalues and eigenvectors.

\begin{align*}
N\ket{n} =& n\ket{n}\\
a^+a^-\ket{n} =& n\ket{n}\\
a^+(a^-a^+ - a^+a^-)\ket{n} =& a^+\ket{n}\text{, using (\ref{eq:a:comm})}\\
a^+a^-a^+ \ket{n} =& a^+a^+a^-\ket{n} + a^+\ket{n}\\
N a^+ \ket{n} =& (n+1) a^+ \ket{n}
\end{align*}

So $a^+$ acts as raising operator, and we can show $a^-$ is lowering. Also $a^-$ eventually leads to $0$.

We will see (Theorem \ref{thm:norm:harmonic}) the appropriate normalization:
\begin{align*}
a^+\ket{n} =& \sqrt{n+1}\ket{n+1}\\
a^-\ket{n} =& \sqrt{n}\ket{n-1}
\end{align*}
 
\section{Spin}

\subsection{Harmonic oscillators(continued)}
Vacuum is the equilibrium for electromagnetic field. 

We will find ground state from Schr\"odinger equation.

Since $H$ is positive, there can be no negative eigenvalues.

\begin{align*}
a^-\ket{0}=&0\\
N\ket{0}=& 0
\end{align*}
NB $\ket{0} \ne 0$. $\ket{0}$ is a vector which can be normalized; $0$ is a vector of length 0.
\begin{align*}
-\frac{1}{2}\frac{d^2 \psi(x)}{dx^2} + \omega^2 x^2 \psi(x)=& E\psi(x)\text{, Schr\"odinger}\\
\int_{-\infty}^{\infty} dx \psi^*(x) \psi(x) =& 1\\
a^-\ket{0}=& 0\\
\big(-i \frac{d}{dx} - i \omega x \big)\psi_0(x) =& 0\\
\psi_0(x) =& e^{-\frac{\omega x^2}{2}} \text{, not normalized.}
\end{align*}

We can then use $a^+$  to generate excited states. Time dependent looks like classical oscillator, especially at high energy (energy larger that ground state). Figure \ref{fig:wave:harmonic} shows how higher levels are away from 0 most of time.

\begin{figure}[H]
	\caption{Wave functions for harmonic oscillator}\label{fig:wave:harmonic}
	\includegraphics[width=0.9\textwidth]{harmonic_wavefunction}
\end{figure}

\subsection{Spin}

Some particles have half-integral spin, e.g. electron or proton. At rest has no orbital rotation, only spin. Spin is angular momentum attached to particle - see (\ref{eq:lx_ly}), (\ref{eq:ly_lz}), and (\ref{eq:lz_lx}). We introduce the Pauli matrices:

$$
\sigma_z = \begin{pmatrix}
1 & 0 \\
0 & -1
\end{pmatrix}
\quad
\sigma_x = \begin{pmatrix}
0 & 1 \\
1 & 0
\end{pmatrix}
\quad
\sigma_z = \begin{pmatrix}
0 & -i \\
i & 0
\end{pmatrix}
$$

and show they satisfy (\ref{eq:lx_ly}), (\ref{eq:ly_lz}), and (\ref{eq:lz_lx}). Actually need $s_i=\frac{\sigma_i}{2}$.

\begin{align*}
[s_x,s_y] =& i s_z \\
[s_y,s_z] =& i s_x\\
[s_z,s_x] =& i s_y\\
J=&L+s\text{, where $J$ is total angular momentum.}
\end{align*}

Eigenvalues of $s_z$ are $\pm \frac{1}{2}$.

Notice that we have integral and half integral spins.

See Figure \ref{fig:degeneracy:hydrogen}. Number of states for each n, 1, 4, 9, 16, ...

For Helium, we can put 2 electrons into ground state; if we try to create an ion with 3 electrons, one goes into next state! Pauli exclusion principle. New property with two values, up and down. Checked with magnetic field.

Pauli's exclusion is a postulate of non-relativistic QM, but a consequence of relativistic QM.

Two kinds of particles: ones that satisfy Pauli, and those that don't.

Identical quantum particles are indistinguishable.

\begin{align*}
\psi(x) =& \braket{x|\psi}\\
\psi(x_1,x_2) =& \braket{x_1x_2|\psi}\text{, two particles}\\
\psi(x_1,x_2) \rightarrow & \psi(x_2,x_1)\text{, swap using operator $S$}\\
S\ket{x1,x2} =& \ket{x2,x1}\\
S^2 =& 1
\end{align*}

$S$ is unitary, so eigenvalues are $\pm1$.

\section{Fermions: a tale of two minus signs}

Fermions: spin $\frac{1}{2}$ particles, which satisfy Pauli exclusion principle. NB, the 2 electrons in Helium are \textit{entangled}.

Photons don't have an exclusion principle; in fact they have a tendency to congregate.
 
Can we have spin $\frac{1}{2}$ particles that don't obey Pauli exclusion principle? No. Can show this from quantum field theory?

Consider wave function of positions for multiple particles, e.g. $\ket{x_1,x_2,x_3}$. Then $\braket{x_1,...x_n|\psi}=\psi(x_1,...x_n)$

\begin{align*}
\ket{x_1, x_2}=&\ket{x_2, x_1}e^{i\phi}\text{, interchange twice--there are two possibilities}\\
\ket{x_1, x_2}=&+\ket{x_2, x_1}\text{ Bosons}\\
\ket{x_1, x_2}=&-\ket{x_2, x_1}\text{ Fermions}
\end{align*}

NB: sign is not observable.

\begin{align*}
\psi(x_1,x_2)=&-\psi(x_2,x_1) \text{, Fermions}\\
\psi(x_1,x_2)=&\psi(x_2,x_1) \text{, Bosons}\\
\psi_0(x_1)\psi_0(x_2)& \text{--OK for Bosons but not Fermions}\\
\psi_0(x_1)\psi_1(x_2)& \text{--not OK for Bosons, or Fermions}\\
\psi_0(x_1)\psi_1(x_2)+\psi_0(x_2)\psi_1(x_1)& \text{--OK for Bosons}\\
\psi_0(x_1)\psi_1(x_2)-\psi_0(x_2)\psi_1(x_1)& \text{--OK for Fermions}
\end{align*}

Rotate system by $2\pi$: is there a phase?

\begin{align*}
J_z\ket{\psi} =& -i \frac{\partial \ket{\psi}}{\partial \theta}\\
=& m\ket{\psi} \text{, if eigenvector}\\
\ket{\psi(\theta)} =& e^{im\theta}\ket{\psi(0)}\text{, so $m$ integral or half integral.}
\end{align*}

What if $m$ half integral and we rotate by $2\pi$? We have phase of $-1$, so $\psi$ changes sign.

David Finkelstein. Belt, Deep topological connection between rotation and interchange.

\section{Quantum Field Theory}

The description of nature as we know it, with the exception of gravity. In principle we believe that we could explain everything, except gravity, if we only had enough computational power.

\subsection{Review of Harmonic Oscillator}
Imagine many (possibly infinite number) harmonic oscillators.

From (\ref{eq:creation:operator}) and (\ref{eq:annihilation:operator}):
\begin{align*}
a^+_i \triangleq & \frac{P + i \omega x } {\sqrt{2 \omega}} \text{, creation operator} \numberthis \label{eq:creation:operator:i}\\
a^-_i \triangleq & \frac{P - i \omega x } {\sqrt{2 \omega}}\text{, annihilation operator} \numberthis \label{eq:annihilation:operator_i}\\
\end{align*}

Think as each oscillator  as an independent system of degrees of freedom, so operators commute. Using  (\ref{eq:a:comm}), and (\ref{eq:number:operator})

\begin{align*}
[a^+_i,a^+_j] =& 0 \numberthis \label{eq:a:comm_i_plus}\\
[a^-_i,a^-_j] =& 0 \numberthis \label{eq:a:comm_i_minus}\\
[a^-_i,a^+_i] =& \delta_{i,j} \numberthis \label{eq:a:comm_i}\\
H =& \hslash \sum_{i}  \omega_i N_i\text{, where $N_i$ are known as occupation numbers.}
\end{align*}
We're ignoring ground state energy.

One basis of states is $\ket{n_1,n_2,n_3,...}$.

\begin{thm}[Normalization of harmonic oscillator state--creation]\label{thm:norm:harmonic}
	\begin{align*}
	|\ket{n}|=&1\\
	\implies&\\
	a^+\ket{n}=&\sqrt{n+1}\ket{n+1}
	\end{align*}
\end{thm} 

\begin{proof}
	\begin{align*}
	a^+\ket{n}=&c_n\ket{n+1} \text{, assumption}\\
	\bra{n}a^-=&c_n\bra{n+1}\text{, assume real}\\
	\braket{n|a^-a^+|n}=&c_n^2\text{, normalization!}\\
	=& \braket{n|a^+a^-+1|n}\text{, using commutator}\\
	=& \braket{n|N+1|n}\\
	=&n+1 \text{, whence}\\
	c_n=&\sqrt{n+1}
	\end{align*}
\end{proof}

\begin{cor}[Normalization of harmonic oscillator state--annihilation]
	\begin{align*}
	a^-\ket{n}=\sqrt{n}\ket{n-1}
	\end{align*}
\end{cor}


$a_i^+\ket{n_1,n_2,.....}=\sqrt{n_i+1}\ket{n_1,n_2,...n_i+1,...}$

\subsection{Quantum Field Theory of Bosons}

Fields are functions of position. $\psi(x)$ not observable; we observe position, momentum, etc, but not $\psi$. $\psi(x_1,x_2,...x_{15})$ function of many positions.

\begin{itemize}
	\item $\Psi$ is an observable, e.g. magnetic field, so an operator.
	\item $\Psi$ function of one position
	\item $\Psi$ describes any number of particles.
\end{itemize}


Consider one particle in a box, and think about energy eigenstates. Wave function $\psi_1(x)$ is sine (lowest energy level). Next $\psi_2(x)$ has one node, with higher energy, $\psi_3(x)$ two nodes,...

Now many particles (bosons), some in first energy level, some in second,...$\ket{n1,n2,n3,...}$> Now we invent operators as in (\ref{eq:a:comm_i_plus}) - (\ref{eq:a:comm_i}) to create and remove particles. NB $\set{1,2,3,...}$ are states $\set{n_i}$ occupation numbers.

\begin{align*}
(a^+a^-)\ket{n}=&(a^-a^+-1)\ket{n}\\
=&\sqrt(n+1)a^-\ket{n+1}-\ket{n}\\
=&(n+1)\ket{n}-\ket{n}\\
=&n\ket{n}
\end{align*}

NB: we are \text{ defining} creation and annihilation operators. Don't ask \emph{why} for a definition, \emph{ask why it is useful}. E.g. creation lets us create a photon. We want to study system with variable number of photons.

\begin{defn}[Vacuum]
	Vacuum is state that is annihilated by $a^-$ -- $\ket{0,0,0,....}$.
\end{defn}

Denote energy by $\omega_i$ for state $i$. 

\begin{align*}
E =& \sum_{i} n_i \omega_i\\
=& \sum_{i} \omega_a a^+_i a^-_i 
\end{align*}

We'll consider free particles--no interactions.

Why can we ignore ground-state energy? Because constants commute with everything. 

Quantum fields theory is a book-keeping device for particles.

$\Psi(x)$ -- an operator that is a function of position--\emph{Fock Space}.

\begin{align*}
\Psi(x)\triangleq&\sum_{i}a^-_i\psi_i(x)\text{, not Hermitean, so conjugate is}\\
\Psi^\dagger(x)=&\sum_{i}a^+_i\psi_i^*(x)\\
\Psi(x)+\Psi^\dagger \text{ is}&\text{ Hermitean}
\end{align*}

What if we apply to vacuum? $\Psi(x)$ annihilates it--what about $\Psi(x)^\dagger$.

\begin{align*}
\sum_{i} \ket{i} \bra{i} =& I \text{, sum over states with $\psi_i$}\\
\sum_{i} \ket{i} \braket{i|X} =& \ket{X}\\
\psi^*_i(x)\underbrace{\ket{i}}_\text{one particle with state $i$} =& \ket{X}\text{, but}\\
\ket{i} =& a^+_i\ket{0} \text{, whence}\\
\psi^*_i(x) a^+_i\ket{0}=&\ket{x}\\
\Psi^\dagger(x) \ket{0} =& \ket{x}\text{ so $\Psi^\dagger(x)$ creates a particle at $x$.}
\end{align*}

\begin{itemize}
	\item $a^+_i$ creates a particle in state $i$.
	\item $\Psi^\dagger(x)$  creates a particle at position $x$.
	\item $a^-_i$ annihilates a particle in state $i$, or gives zero.
	\item $\Psi(x)$  annihilates a particle at position $x$, or gives zero.
\end{itemize}

There is a separate field for each Boson.

Add vacuum to a state, we get probability of 0.5 of vacuum, 0.5 of original state. It is not the same as zero. Vacuum has length of 1!

Let's add two particles, at $x$ and $y$. 
 
\begin{align*}
\Psi(y)\Psi(x)\ket{0} =& \ket{y,x}\\
\Psi(x)\Psi(y)\ket{0} =& \ket{x,y} \text{, but from (\ref {eq:a:comm_i_plus})}\\
\Psi(y)\Psi(x) =& \Psi(x)\Psi(y) \text{, so}\\
\ket{x,y} =& \ket{y,x} \text{, which is what we expect for Bosons.} 
\end{align*}

We can handle situation where number of particles is a random variable: a laser wave is a superposition.

\section{Quantum Field Theory II}

Can start with particles and see why they can be described by fields, or start with fields and quantize. Starting with particles we can use fields to handle variable numbers of particles. Normalization $\int \psi^*(x) \psi(x) dx=1$ gives \textit{number of particles}.

Given an orthonormal basis:
\begin{align*}
\sum_{i} \ket{i} \bra{i} =& I \text{, "Resolution of the Identity"}\numberthis \label{eq:resolution:identity}\\
\braket{y|x}=& \sum_{i} \braket{y|i} \braket{i|x}\\
\delta(x-y)=& \sum_{i} \psi_i(y) \psi^*_i(x)\text{, for any set of eigenvectors of Hermitan operator.}
\end{align*}

Can characterize multi-particle system by $\ket{n_1, n_2,...n_i,...}$. We want to increase and decrease occupation numbers, so introduce creation and annihilation operators.

\begin{itemize}
	\item $a^+_i=a^\dagger_i$
	\item $a^-_i=a_i$
\end{itemize} 

Introduce field operator
\begin{align*}
\Psi(x) \triangleq & \sum_{i} a_i \psi_i(x) \text{, would be Fourier if sine/cosine waves}\\
\Psi^\dagger(x) =& \sum_{i} a^\dagger \psi^*_i(x)\\
\Psi(x) + \Psi^\dagger(x) \text{ is}& \text{ Hermitean (observable)}\\
\frac{\Psi(x) - \Psi^\dagger(x)}{i} \text{ is}& \text{ Hermitean (observable)}
\end{align*}

Vacuum: $\ket{0} = \ket{0,0,0,.....}$.

\begin{align*}
\ket{x}=&\sum_{i}\ket{i}\braket{i|x}\text{, from (\ref{eq:resolution:identity})}\\
=& \sum_i \psi^*_i(x)a^\dagger_i\ket{0}\\
=& \Psi^\dagger(x) \ket{0}\text{, creation operator at a point.}
\end{align*} 

Consider the following operator:
\begin{align*}
\int dx \Psi^\dagger(x) \Psi(x)=&\int dx \sum_{i,j}a^+_i\psi_i^*(x) a_j\psi_j(x)\\
=&\sum_{i,j} a^+_i a_j \int dx \psi_i^*(x) \psi_j(x)\\
=&\sum_{i,j} a^+_i a_j \delta_{i,j}\\
=&\sum_i a^+_i a_i\\
=&\sum_i N_i\text{, operator representing total number of particles.}
\end{align*}

To make energy finite, total number of particles must be finite. $\Psi^\dagger(x) \Psi(x)$ represents particle density.

We now calculate the total energy.

\begin{align*}
E =& \sum_{i} N_i \omega_i\\
=& \sum_{i} a^\dagger_i a_i \omega_i
\end{align*}

We determine $\omega_i$ from eigenvalues of Schr\"odinger equation (ignoring interactions):
\begin{align*}
H \psi_i =& \omega_i \psi_i\\
\big[\frac{p^2}{2m} + V(x)\big]\psi_i(x)=& \omega_i \psi_i(x)\\
\big[-\frac{\nabla^2}{2m} + V(x)\big]\psi_i(x)=& \omega_i \psi_i(x)
\end{align*}

Guess solution
\begin{align*}
E=&\int dx \underbrace{\Psi^\dagger(x) \big[-\frac{\nabla^2}{2m} + V(x)\big] \Psi(x)}_\text{energy density} \\
=& \int dx \sum_{i,j}a^+_i\psi_i^*(x)\big[-\frac{\nabla^2}{2m} + V(x)\big]\psi_j(x)a^-_j\\
=& \int dx \sum_{i,j} a^+_i a^-_j \psi_i^*(x) \omega_j \psi_j(x)\\
=&  \sum_{i,j} a^+_i a^-_j \delta_{i,j} \omega_j
\end{align*}

Behaves like classical theory if the number of particles is very large. E.g. expectations follow classical laws.

If we don't split too many hairs, density from electromagnetic field is density of photons.

\section {Second Quantization}

\subsection{Digression on neutrinos}

If neutrinos have mass, and they mix, do they have the same mass? No.

How to mix and conserve energy?

Neutrinos - mass is energy. Eigenstates of energy are linear superpositions of eigenstates of something else, the type of neutrino. Eigenstates of energy don't mix with each other. Here is an analogous situation. Imagine a particle trapped in the potential of Figure \ref{fig:particle_mixed_left}, and assume middle barrier very high, so particle is trapped on one side or t'other (ignore tunnelling). What is ground state? But there must be a second ground state, from symmetry--Figure \ref{fig:double:well}.

\begin{figure}[H]
	\caption{Double well potential}\label{fig:particle_mixed_left}
	\includegraphics[width=0.8\textwidth]{particle_mixed_left}
\end{figure}

\begin{figure}[H]
	\caption{Particle trapped in double well}\label{fig:double:well}
	\includegraphics[width=0.8\textwidth]{particle_mixed}
\end{figure}

\begin{thm}[If $V$ symmetric,$\psi$ either symmetric or anti-symmetric]
	If  $\psi$  is a continuously differentiable solution to the time independent Schr\"odinger equation
	\begin{align*}
	-\frac{\hslash^2}{2m}\frac{d^2 \psi(x)}{d x^2} + V(x)\psi(x) =& E\psi(x) \text{ and}\numberthis \label{eq:ti:schroedinger}\\
	\forall x V(-x) =& V(x)\text{, then either}\\
	\psi(x) =& \psi(x)\text{ or}\\
	\psi(x) =& \psi(-x)
	\end{align*}
\end{thm}

\begin{proof}
	Define an operator $R$ (reflection in time) such that:
	\begin{align*}
		R \ket{\psi} =& \ket{\psi^{\prime}} \text{, where} \numberthis \label{eq:reflection}\\
		\psi^{\prime}(x) =& \psi(x)\text{. Since $V(x) = V(-x)$, $RV=V$, and $\psi^{\prime}$ satisfies (\ref{eq:ti:schroedinger}), whence}\\
		\psi^{\prime}(x) =& \rho \psi^(x) \text{, for some constant $\rho$.} \numberthis \label{eq:R:rho}
	\end{align*}
	WLOG we can assume that $\psi$ and $\psi^{\prime}$ are both normalized, whence:
	\begin{align*}
		\lvert \rho \rvert = 1& \text{, so (\ref{eq:reflection},\ref{eq:R:rho}) give:}\\
		R \ket{\psi} =& \rho \ket{\psi} \text{. But (\ref{eq:reflection}) implies}\\
		R^2    =& I \text{, whence}\\
		\rho^2 =& 1 \text{, i.e.}\\
		\rho \pm& 1 \text{, so, in any interval where $\psi(x)\ne 0$ \emph{either}}\\
		\psi(x) =& \psi(-x) \text{\emph{or}}\\
		\psi(x) =& -\psi(-x)
	\end{align*}
	We still need to rule out the possibility that $\rho$ changes sign. Let $x_0$ be a point such that $\psi(x_0)=0$, and, WLOG, assume that $\psi(x)=\psi^{\prime}(x)$ in some interval up to $x_0-$ and $\psi(x)=-\psi^{\prime}(x)$ in some interval starting $x_0+$. Since $\psi$ is continuously differentiable: then $\frac{\psi^{\prime}(x_0-)}{dx}=\frac{\psi^(x_0-)}{dx}$. Now $\psi^{\prime\prime}=\big(\psi^{\prime}-\psi\big)$ satisfies (\ref{eq:ti:schroedinger}), $\psi^{\prime\prime}(x_0)=0$, and $\frac{\psi^{\prime\prime}(x_0-)}{dx}=0$, whence $\psi^{\prime\prime}=0 \forall x$. 
\end{proof}

But neither LHS or RHS wave function of Figure \ref{fig:double:well} symmetric or antisymmetric. Can make symmetric or antisymmetric combinations. Figure \ref{fig:double:well:symmetrized} has slightly lower energy than Figure \ref{fig:double:well}. Figure \ref{fig:double:well:1st} shows first excited state.

\begin{figure}[H]
	\caption[Wave function symmetrized]{Wave function symmetrized(minimum in middle is slightly greater than zero-allow tunnelling)}\label{fig:double:well:symmetrized}
	\includegraphics[width=0.8\textwidth]{particle_mixed_symmetrized}
\end{figure}

\begin{figure}[H]
	\caption{First excited state}\label{fig:double:well:1st}
	\includegraphics[width=0.8\textwidth]{particle_mixed_1st_excited}
\end{figure}

So we mix the pure states from Figures \ref{fig:double:well:symmetrized} and \ref{fig:double:well:1st}. Let's start with electron on left, and evolve.
\begin{align*}
\frac{\psi_L+\psi_R}{\sqrt{2}}& \text{--symmetric, with energy }& E_1-\epsilon\\
\frac{\psi_L-\psi_R}{\sqrt{2}}& \text{--antisymmetric, with energy }& E_1+\epsilon
\end{align*}
Each is an eigenvector, so they evolve as follows:
\begin{align*}
\frac{\psi_L+\psi_R}{\sqrt{2}} e^{(E_1-\epsilon)t}=&\psi^+ \text{, say}\\
\frac{\psi_L-\psi_R}{\sqrt{2}}e^{(E_1+\epsilon)t}=&\psi^-
\end{align*}

Now what if we start with pure $\psi_L$?
\begin{align*}
	\psi_L =& \frac{\psi^+ + \psi^-}{\sqrt{2}}\text{, then the denominator evolves to}\\
	\psi_L^\prime =&\psi^+ e^{(E_1-\epsilon)t} + \psi^- e^{(E_1+\epsilon)t}\\
	=&e^{E_1 t}\big[\psi^+ e^{-\epsilon t} + \psi^- e^{+\epsilon t}\big]
\end{align*}
Signs change with time, so eventually they will be opposite.
\begin{align*}
e^{-\epsilon t} =& - e^{\epsilon t}\\
e^{2 \epsilon t}=&-1\\
2 \epsilon t = \pi \text{, so $\psi_L$ becomes $\psi_R$!}
\end{align*}

Mixing goes with oscillation. Neutrinos go like that--electron neutrino changes to muon neutrino. Masses are energy levels. Coupling in Hamiltonian.

Ammonia - $N$ tunnels through $H_3$ plane. 

Spins.

\url{https://youtu.be/7G4C7scQX3A?t=1027}
\url{https://youtu.be/7G4C7scQX3A?t=1467}
\url{https://youtu.be/7G4C7scQX3A?t=1961}

\subsection{Second Quantization of Bosons}

Fourier transforms let us move between the position and momentum representations.

\begin{align*}
\psi(x) \rightarrow& \psi^*(x)\psi(x)=&P(x) \text{ in position representation}\\
\widetilde{\psi}(p) \rightarrow& \widetilde{\psi}^*(p) \widetilde{\psi}(p) =&P(p) \text{ in momentum representation, where}\\
\widetilde{\psi}(p) =& \int \frac{dx}{\sqrt{2\pi}} \psi(x) e^{-i p x}\\
\psi(x) =& \int \frac{dp}{\sqrt{2\pi}} \widetilde{\psi}(p) e^{+i p x}
\end{align*}

Now consider this in field theory.

\begin{align*}
\Psi(x) =& \sum_{i} a^-_i \psi_i(x)\\
\psi_i(x) =& e^{ipx} \text{ for a free particle (different $i$!), so} \\
\Psi(x)=& \int \frac{dp}{\underbrace{\sqrt{2\pi}}_\text{by convention}} a^-(p) e^{ipx} \text{, where annihilation operator}\\
a^-(p)& \text{ removes 1 particle with momentum p: plays same role as $\widetilde{\psi}(p)$ }\\
\Psi^\dagger(x)=& \int \frac{dp}{\sqrt{2\pi}} a^+(p) e^{ipx} \text{ creates particle at position $x$}
\end{align*}

There is a similarity between wave functions in position and momentum space and creation and annihilation operators for particles and given position or momentum.
\begin{align*}
a^-(p) = \int \frac{dx}{\sqrt{2\pi}} \Psi(x) e^{-ipx}\\
a^+(p) = \int \frac{dx}{\sqrt{2\pi}} \Psi^\dagger(x) e^{ipx}
\end{align*}

Remember there is a field operator for each kind of particle.

Can prove following.
 
\begin{align*}
[\Psi^+(x),\Psi^-(y)] =& \delta(x-y) \text{, introduces Heisenberg uncertainty!}\\
[\Psi^+(x),\Psi^+(y)] =&0\\
[\Psi^-(x),\Psi^-(y)] =&0\\
[\Psi^+_R(x),\Psi^-_I(y)] =& \delta(x-y) \text{, can not measure simultaneously at same point!}
\end{align*}

Makes sense in relativity: would violate causality. Spacelike! Can measure at different points.

Fermions don't work like this! One measurement can kick another arbitrarily far away.

\section{Quantum field Hamiltonian}

\subsection{Particle Field Interactions}

Hamiltonian for very simple quantum field,  particles satisfying Schr\"odinger equation.
\begin{align*}
H =& \int dx \big[ \Psi^\dagger(x) \frac{- \nabla^2}{2m} \Psi(x) + V(x) \Psi^\dagger(x) \Psi(x) \big] \text{, count particles; give each energy V(x)}\\
=&\int dx \big[\Psi^\dagger(x) \frac{- \nabla^2}{2m} \Psi(x) + mc^2 \Psi^\dagger(x) \Psi(x)\big] \text{, special case of constant potential energy} \numberthis\label{eq:psi_mc2}
\end{align*}

Hamiltonian updates state.

\begin{align*}
\ket{\phi(t+\epsilon)} =& (1-i \epsilon H) \ket{\phi(t)} \text{State}\\
=& \ket{\phi(t)} -i \epsilon H \ket{\phi(t)} \numberthis \label{eq:update:state}
\end{align*}

What would it mean to say momentum is conserved? It means Hamiltonian doesn't change momentum. Let's see how this works with (\ref{eq:psi_mc2}.) Rewrite 2nd term  using momentum variables

\begin{align*}
\Psi(x) =& \int \frac{dp}{\sqrt{2\pi}} \widetilde{\Psi}(p) e^{ipx}\\
\Psi^\dagger (x) =& \int \frac{dq}{\sqrt{2\pi}} \widetilde{\Psi^\dagger}(q) e^{-iqx} \numberthis \label{eq:psi:dagger}\\
\int dx mc^2 \Psi^\dagger(x) \Psi(x) =& \frac{mc^2}{2 \pi} \int \widetilde{\Psi^\dagger}(q) \widetilde{\Psi}(p) e^{i(p-q)x} dx dq dp\\
=& \frac{mc^2}{2 \pi} \int \widetilde{\Psi^\dagger}(q) \widetilde{\Psi}(p) \delta(p-q) dq dp\\
=& \frac{mc^2}{2 \pi} \int \widetilde{\Psi^\dagger}(p) \widetilde{\Psi}(p) dp
\end{align*}
This just removes a particle with momentum $p$ and puts it back; momentum isn't changed. Let's repeat with the 1st term of (\ref{eq:psi_mc2}.)

(Digression: calculation with multiple particles shows momentum still conserved.)

\begin{align*}
- \nabla^2 \Psi^\dagger (x) =& \int \frac{dq}{\sqrt{2\pi}} q^2 \widetilde{\Psi^\dagger}(q) e^{-iqx}\text{, from (\ref{eq:psi:dagger}), so}\\
\int dx \Psi^\dagger(x) \frac{- \nabla^2}{2m} \Psi(x)=& \int \frac{ p^2}{2m} \widetilde{\Psi^\dagger}(p) \widetilde{\Psi}(p) dp \text{, adds up kinetic energy.}
\end{align*}

\url{https://youtu.be/7a1hon7NJ1M?t=1605}

Imagine two species of particles: fake electrons and protons that happen to be protons.

\begin{align*}
H =& \int dx  \Psi^\dagger_e(x) \frac{- \nabla^2}{2m_e} \Psi_e(x) + \int dx \Psi^\dagger_p(x) \frac{- \nabla^2}{2m_p} \Psi_p(x)  + g \int dx \underbrace{\Psi^\dagger_e(x) \Psi^\dagger_e(x) \Psi_e(x) \Psi_p(x)}_\text{Figure \ref{fig:scatter:e:p}} \text{, where}\\
g=& \textit{ coupling constant--proportional to probability of interaction.}
\end{align*}

\begin{figure}[H]
	\caption[$\Psi^\dagger_e(x) \Psi^\dagger_e(x) \Psi_e(x) \Psi_p(x)$--scattering]{$\Psi^\dagger_e(x) \Psi^\dagger_e(x) \Psi_e(x) \Psi_p(x)$: annihilate and create an electron and proton if they are at the same place, then create again --i.e. scatter}\label{fig:scatter:e:p}
	\includegraphics[width=0.8\textwidth]{scatter_e_p}
\end{figure}

\begin{itemize}
	\item Total momentum is conserved, not that of individual particles.
	\item Coulomb potential has been ignored.
	\item If experiment shows that particles interact, then there must be a term in the Hamiltonian.
\end{itemize}

Imagine particle decay, as shown in Figure \ref{fig:particle:decay}. We might try a term: $\Psi^\dagger_b(x) \Psi^\dagger_c(x) \Psi_a(x)$. Since space it isotropic, the interaction can happen anywhere: $g \int \Psi^\dagger_b(x) \Psi^\dagger_c(x) \Psi_a(x) dx$. We can show that momentum is conserved.

\begin{figure}[H]
	\caption{Particle Decay: $\Psi^\dagger_b(x) \Psi^\dagger_c(x) \Psi_a(x)$}\label{fig:particle:decay}
	\includegraphics[width=0.8\textwidth]{particle_decay}
\end{figure}

Now what goes in the Hamiltonian must be Hermitian, so we need:
\begin{align*}
g \int \Psi^\dagger_b(x) \Psi^\dagger_c(x) \Psi_a(x) dx + g \int \Psi^\dagger_a(x) \Psi_c(x) \Psi_a(x)
\end{align*}
which reverses the process of  Figure \ref{fig:particle:decay}: $b+c\rightarrow a$.
We now have a simple version of quantum field theory:
\begin{itemize}
	\item Creation and annihilation operators;
	\item Fields made out of Creation and annihilation operators;
	\item Fields are functions of position;
	\item Each type of particle has its own field;
	\item Write down Hamiltonian with kinetic energies of particles and other concoctions. 
\end{itemize}

Imagine that we made $\epsilon$ a little larger in (\ref{eq:update:state}):

\begin{align*}
\ket{\phi(t+\epsilon)} =& \ket{\phi(t)} -i \epsilon H \ket{\phi(t)} - \frac{\epsilon^2}{2} H^2 \ket{\phi(t)}
\end{align*}

What is the effect of $H^2$? E.g., Figure \ref{fig:particle:decay} becomes a scatter: $b+c \rightarrow a \rightarrow b + c$. Or Figure \ref{fig:scatter:annihilate} shows a more complex situation. You can have a forest of interactions that are mediated by a single term in the Hamiltonian.

\begin{figure}[H]
	\caption{Exchange diagram}\label{fig:scatter:annihilate}
	\includegraphics[width=0.8\textwidth]{scatter_annihilate}
\end{figure}

Figure \ref{fig:scatter:annihilate} has a coupling constant $g^2$.


\begin{figure}[H]
	\caption[Scatter electron and emit photon(field $A$):$A\Psi^+_e\Psi^-_e$]{Scatter electron and emit photon(field $A$):$A\Psi^+_e\Psi^-_e$. Sometimes what you thought was an electron behaves like an electron + photon. Low probability: think of as correction to electron, not a new process. If we look, we will screw up electron! Superposition.}
	\includegraphics[width=0.8\textwidth]{electron_scattering}
\end{figure}

\url{https://youtu.be/7a1hon7NJ1M?t=3923}

\subsection{The Dirac Equation}

We will study the relativistic electron. We will start with an electron that moves in 1 direction along a line. Electrons in at atom move at a significant proportion of the speed of light.  

\begin{align*}
E =& \frac{p^2}{2m} \text {, classical}\\
P =& -i \frac{\partial}{\partial x} \text{, quantum}\\
H = & \frac{\partial}{\partial t}\\
E^2 =& p^2 + m^2 \text{, relativistic($c^2=1$). This becomes:}\\
-\frac{\partial^2}{\partial t^2} \phi(x) =& -\frac{\partial^2}{\partial x^2}\phi(x) + m^2 \phi(x) \text{, Klein-Gordon equation. But Dirac wanted this:}\\
i \frac{\partial psi}{\partial t} =& H\psi\\
=& \sqrt{- \frac{\partial}{\partial x} + m^2} \psi \text{. What could this mean?}  
\end{align*}

Dirac decided to start again. Imagine a particle moving along the x axis with the speed of light.

\begin{align*}
P =& E \text{, moving to right}\\
i \frac{\partial \psi}{\partial t} =& -i \frac{\partial \psi}{\partial x}\\
\implies&\\
\frac{\partial \psi}{\partial t} + \frac{\partial \psi}{\partial x}=&0\text{. Any function of x-t will satisfy equation}
\end{align*}

Two things are wrong with this equation:
\begin{enumerate}
	\item Allows particles with negative energy with negative momentum.
	\item Only has particles that move to the right.
	\item Electron has no mass: speed of light.
\end{enumerate}

Dirac solved second problem by imagining two species of electrons, one moving to the right, t'other to the left, labelled $\psi_1$ and $\psi_2$. Or a particles can be one or t-other, or a superposition.

\begin{align*}
\frac{\partial \psi_1}{\partial t} + \frac{\partial \psi_1}{\partial x}=&0\\
\frac{\partial \psi_2}{\partial t} - \frac{\partial \psi_2}{\partial x}=&0
\end{align*}

Imagine particle with two degrees of freedom, $x$ and $1/2$. Make second degree into matrix index. Introduce an observable to tell is whether particle is 1 or 2.

\begin{align*}
\underbrace{
	\begin{pmatrix}
		1 & 0\\
		0 & -1
	\end{pmatrix}
}_\text{Dirac's $\alpha$}
\begin{pmatrix}
	\psi_1\\
	\psi_2
\end{pmatrix}\\
H =& \alpha P\\
i\frac{\partial \psi}{\partial t} =& -i \alpha \frac{\partial \psi}{\partial x} \text{, matrix equation!}
\end{align*}

Dirac added a term to give electron mass.

\begin{align*}
H =& \alpha P + \beta m \text{, where $\beta$ is some matrix. Now}\\
E^2 =& P^2 + m^2 \text{, so}\\
E^2 =& \big(\alpha P + \beta m \big) \big(\alpha P + \beta m \big)\\
=& \underbrace{\alpha^2 P^2}_\text{(=1)} + \underbrace{\beta^2 m^2}_\text{set  $=1$} + (\alpha \beta + \beta \alpha) P m\\
=&  P^2 + m^2 + (\alpha \beta + \beta \alpha) P m \text{, so we want $\alpha \beta + \beta \alpha=0$}
\end{align*}

Now $\alpha$ is a Pauli matrix, so we can use any other Pauli matrix for $\beta$. We will use
$
\begin{pmatrix}
0&1\\
1&0
\end {pmatrix}
$.

We have the 1-dimensional version of Dirac's equation.
\begin{align*}
i \frac{\partial \psi}{\partial t}=& a p \psi + \beta m \psi \text{, where $\psi$ is a two component spinor.}\\
i \frac{\partial}{\partial t}\begin{pmatrix}
\psi_1\\
\psi_2
\end{pmatrix}=&-i \begin{pmatrix}
1&0\\
0&-1
\end{pmatrix} \frac{\partial}{\partial x}\begin{pmatrix}
\psi_1\\
\psi_2
\end{pmatrix} + \begin{pmatrix}
0&m\\
m&0
\end{pmatrix}\begin{pmatrix}
\psi_1\\
\psi_2
\end{pmatrix}
\end{align*}

So
\begin{align*}
i\frac{\partial \psi_1}{\partial t} + i\frac{\partial \psi_1}{\partial x}=&m \psi_2\\
i\frac{\partial \psi_2}{\partial t} - i\frac{\partial \psi_2}{\partial x}=&m \psi_1
\end{align*}

So the mass term couples the two equations, and makes particle move slower then light. Dirac knew that mass is coupled to ability to move left or right.

\section{Fermions and the Dirac equation}

\subsection{Second Quantization of Fermions}

If you have two bosons, can out them into same state, and creation an annihilation operators commute. Now, imagine two particles whose wave function changes sign when we interchange. Let $\ket{x,y}$ represent particles at x and y. $\Psi^\dagger(x)\Psi^\dagger(y)\ket{0}$ will create. If operators commute $\ket{x,y}=\ket{y,x}$. For Fermions, $\ket{x,y}=-\ket{y,x}$, so we need  $\Psi^\dagger(x)\Psi^\dagger(y)= -\Psi^\dagger(y)\Psi^\dagger(x)$
 
\begin{align*}
\Psi^\dagger(x)\Psi^\dagger(y)+\Psi^\dagger(y)\Psi^\dagger(x)=&0 \text{ for Fermions. This is usually written:}\\
\{\Psi^\dagger(x)\Psi^\dagger(y)\}=&0 \text{, where}\\
\{A,B\}\triangleq& AB + BA \text{, the anticommutator.}
\end{align*}

NB  $\Psi^\dagger(x)\Psi^\dagger(x)=0$,  so we can't have two identical particles at same position.

We can't confuse Boson creation and annihilation operators, since $[a,a^\dagger]=1$, and $[a^\dagger,a]=-1$. Spectrum of energies from $0$ to $\infty$ can't be turned upside down. For Fermions $\{a\dagger,a\}= \{a,a\dagger\}=1$: from the algebra there is no difference between creation and annihilation operators.

Consider occupation numbers of Fermion with only one state: creation operator takes $0$ to $1$, then annihilates, leaving 0. Complete symmetry between empty and filled, creation and annihilation (mathematically!). $a^2=(a^\dagger)^2=0$.

\subsection{The Dirac Equation in 3 dimensions}

Returning to the 1D Dirac equation for zero mass:
\begin{align*}
i \frac{\partial \psi}{\partial t}=& -i \frac{\partial \psi}{\partial x} \\
H \Psi =& i \Psi \text{, this was a failure as a theory of 1D electrons.}
\end{align*}

We want a 2nd species that propagates to the left.
\begin{align*}
i \frac{\partial \psi_1}{\partial t}=& -i \frac{\partial \psi_1}{\partial x} \\
H \Psi_1 =& i \Psi_1\\
i \frac{\partial \psi_2}{\partial t}=& i \frac{\partial \psi_2}{\partial x} \\
H \Psi_2 =& -i \Psi_2
\end{align*}
We combined into a vector, adding handedness, $\begin{pmatrix} \psi_1\\ \psi_2 \end{pmatrix}$, and defining the matrix $\alpha = \begin{pmatrix}
1&0\\
0&-11
\end{pmatrix}$

\begin{align*}
H \psi =& \alpha P \psi
\end{align*}

That didn't get us a mass, but it got stuff moving in both directions. We then invented another matrix $\beta=\begin{pmatrix}
0&1\\
1&0
\end{pmatrix}$ and then wrote $H = \alpha P + \beta m$. Note $\alpha^2=\beta^2=2$, and $\alpha\beta + \beta\alpha=1$, whence $H^2=P^2+m^2$. Combined left moving and right moving, then added coupling to give mass. How do we do this in 3D?
 Let's  drop $m$ temporarily, make $\alpha$ a vector, so we can dot product with P.
\begin{align*}
H =& \alpha \boldsymbol{\cdot} P\\
=& \alpha_x P_x + \alpha_y P_y + \alpha_z P_z \text{, we want}\\
H^2 =& P_x^2 + P_y^2 + P_z^2 \text{, whence}\\
\alpha_x^2 =&1\\
\alpha_y^2 = &1\\
\alpha_z^2 =& 1 \\
\alpha_y\alpha_x + \alpha_x\alpha_y=&1\\
\alpha_z\alpha_x + \alpha_x\alpha_z=&1\\
\alpha_x\alpha_y + \alpha_y\alpha_z=&1
\end{align*}

\begin{align*}
	\alpha =& \begin{pmatrix}
		\sigma&0 \\
		0&\sigma
	\end{pmatrix} \text{, spin}\\ 
	\alpha_x =& \begin{pmatrix}
		\sigma_x&0\\
		0&\sigma_x
	\end{pmatrix}=&\begin{pmatrix}
		0&1&0&0\\
		1&0&0&0\\
		0&0&0&-1\\
		0&0&-1&0
	\end{pmatrix}\\
	\beta =& \begin{pmatrix}
		0&1\\
		1&0
	\end{pmatrix} =& \begin{pmatrix}
	0&0&1&0\\
	0&0&1&0\\
	1&0&0&0\\
	0&1&0&0
	\end{pmatrix}
\end{align*}

In Dirac theory, mass couple left handed chirality with right handed. Massless particles conserve chirality.

\begin{align*}
\dot L =& i [H,L] \text{, for any L. We'll consider}\\
\dot X =& i[\alpha P + \beta m,X]\\
=& i \alpha [P,X]\\
=&\alpha_x \text{, velocity of particle!}
\end{align*}

\begin{align*}
[H,\alpha_x] \ne 0 \text{, so $\alpha_x$ is not conserved.}
\end{align*}

\url{https://youtu.be/DQUJpT_iz8U?t=3239}

Multiple states --chirality and spin.

Dirac equation can have negative energy, even in 1 D case.

\begin{align*}
H=P \text{, right moving electron, What if $P<0$?}
\end{align*} 

Ground state has lowest energy (not fewest particles). With negative energy can lower indefinitely! What does -ve energy mean?

Dirac defined vacuum to have lowest energy. Dirac filled with negative energies (Fermions!) --only one in each state. Vacuum must have every negative state occupied. Cap add positive energy electron, or move a negative energy electron to positive--Figure \ref{fig:dirac:sea}.

\begin{figure}[H]
	\begin{center}
		\caption[Dirac Sea]{Dirac Sea. Pluck out negative energy electron and get +ve charge and +ve energy hole (positron)}\label{fig:dirac:sea}
		\includegraphics[width=0.4\textwidth]{DiracSea}
	\end{center}
\end{figure}

Imagine photon coming along and hitting -ve energy electron in vacuum: we get an electron and a hole--Figure \ref{photon:dirac}.

\begin{figure}
	\begin{center}
		\caption[Photon$\rightarrow$ electons + hole]{Photon$\rightarrow$ electons + hole. By convention positron points down. Can slice in time: upwards electron, down positrons.}\label{photon:dirac}
		\includegraphics[width=0.5\textwidth]{photon-electrons}
	\end{center}
\end{figure}

The wave function needs fermion operators, which anticommute.
\begin{align*}
\Psi =& \int_{-\infty}^{\infty} dp a^-(p) e^{-ipx}\\
=& \underbrace{\int_{-\infty}^0 dp a^-(p) e^{-ipx}}_\text{negative energies}+ \underbrace{\int_0^{\infty} dp a^-(p) e^{-ipx}}_\text{positive energies}\\
=& \int_{-\infty}^0 dp a^-(p) e^{-ipx}+ \underbrace{\int_0^{\infty} dp b^+(p) e^{-ipx}}_\text{Creation operator for positron}
\end{align*}

Since Fermion creation and annihilation operators the same.


\begin{align*}
a^+a^-A\\
\Psi^\dagger(x)\Psi(x)A\text{, describes Figures \ref{fig:electron_scattering_dirac}, \ref{fig:positron annihilating_dirac}, and \ref{fig:all3-created}}
\end{align*}


\begin{figure}[H]
	\caption{Possible outcomes of $\Psi^\dagger(x)\Psi(x)A$}
	\begin{subfigure}{0.3\textwidth}
		\caption{Electron scattering}\label{fig:electron_scattering_dirac}
		\includegraphics[width=0.9\textwidth]{electron_scattering_dirac}
	\end{subfigure}
	\begin{subfigure}{0.3\textwidth}
		\caption{Positron annihilation}\label{fig:positron annihilating_dirac}
		\includegraphics[width=0.9\textwidth]{positron-annihilation}
	\end{subfigure}
	\begin{subfigure}{0.3\textwidth}
		\caption{Electron, photon, positron created}\label{fig:all3-created}
		\includegraphics[width=0.9\textwidth]{all3-created}
	\end{subfigure}
\end{figure}


\end{document}