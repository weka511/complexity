\documentclass[]{article}
\usepackage{caption,subcaption,graphicx,float,url,amsmath,amssymb,tocloft}
\usepackage[hidelinks]{hyperref}
\usepackage[toc,acronym,nonumberlist]{glossaries}
\usepackage[toc,page]{appendix}
\setacronymstyle{long-short}
\usepackage{glossaries-extra}
\graphicspath{{figs/}} 

\setlength{\cftsubsecindent}{0em}
\setlength{\cftsecnumwidth}{3em}
\setlength{\cftsubsecnumwidth}{3em}

%opening
\title{
	Notes from Origins of Life\\
	Week 1: Introduction
}
\author{Simon Crase\\simon@greenweaves.nz}

\makeglossaries

\loadglsentries{glossary-entries}

\renewcommand{\thesection}{1.\arabic{section}}
\renewcommand{\glstextformat}[1]{\textbf{\em #1}}

\begin{document}

\maketitle

\begin{abstract}
    These are my notes from the $1^{st}$ week of the Santa Fe Institute Origins of Life Course\cite{sfi2019}.\\
    The content and images contained herein are the intellectual property of the Santa Fe Institute, with the exception of any errors in transcription, which are my own.
    These notes are distributed in the hope that they will be useful,
    but without any warranty, and without even the implied warranty of
     merchantability or fitness for a particular purpose. All feedback is welcome,
    but I don't necessarily undertake to do anything with it.
\end{abstract}

\setcounter{tocdepth}{2}
\tableofcontents
\listoffigures

\section{Welcome To The Course}

Lecturer: Chris Kempes

How life originated is one of the most interesting, complicated, difficult, and still open questions in modern science. Throughout this course, we're going to
introduce to you to why it's such a tricky question, and all the different ways that people are trying to solve this unanswered question.	

Since the time of Darwin, we've come to understand how life
progresses, changes, diversifies, finds new niches, evolves greater complexity
And yet we still don't understand how to take that back to
the original formation and initiation and origin of life.


So, in saying that it's important to recognize that life lives on a spectrum--Figure \ref{fig:lifesTransitions}.

\begin{figure}[H]
	\caption[Life lives on a spectrum]{Life lives on a spectrum, which goes from a completely abiotic world
		all the way up to multicellular creatures like us, and entire societies.}\label{fig:lifesTransitions}
	\includegraphics[width=0.9\textwidth]{lifesTransitions}
\end{figure}
At some point along this trajectory, we have this transition into something
that we would call life. And so how is it that we go from this
abiotic world to a biotic world understanding that this is all part of one
evolutionary continuum. So why is that question so challenging?

Well, one of the things that makes it
challenging is that it requires
a lot of diverse types of knowledge to
begin to understand
this process and to answer this question.
And so those types of knowledge draw from
a variety of different disciplines--Figure \ref{fig:tradional:disciplines}.
\begin{figure}[H]
	\begin{center}
		\caption[Traditional disciplines needed for Origin of Life]{We need to understand something about Earth science, biology, chemistry, and physics and how all of these different concept areas intersect with one another and help us to understand exactly what happened when life first formed.}\label{fig:tradional:disciplines}
		\includegraphics[width=0.9\textwidth]{4mainAreas}
	\end{center}
\end{figure}

Throughout this course, we will introduce you to the main questions in each of these concept areas and how they relate to the origins of life. And we will also start to begin to paint a picture of where the intersection and synthesis of these different concepts live, hopefully moving towards a better theory of how life started and evolved into
greater complexity.

So within these concept areas, what are the main questions? 

\subsection{Earth Science}

\begin{itemize}
	\item What was the environment like during the time when life originated?
	We understand that the early Earth was 	very different than the modern Earth.
	\item What chemistry was possible on the early Earth? We need to understand something about what chemistry was possible then and how that might come together to help 	life form in the first place.
	\item What was the diversity and complexity of various micro-environments on the early Earth? So if we think about the early Earth, not
	only as a different average chemical
	space than what we currently have, but
	also lots of different unique types of
	micro-environments existing in this very
	different planetary environment, which of
	those micro-environments were most likely
	to give us life in the first place or to
	have the right type of chemical complexity
	to at least start down that trajectory
	towards a living organism.
	
	\item How did early life and the geosphere co-evolve? 
	So we understand in the modern Earth, how
	life interacts with the overall planetary
	system, we understand through lots of the
	history of life how life interacts,
	has a large feedback width, and modifies
	the geological environment and the
	geosphere. And the question is when life
	first started, "how did that--what did
	that feedback look like and what did this
	coevolution--how did this coevolution
	happen and how important was that for the
	specific trajectory that life took once
	it was formed?"
\end{itemize}


\subsection{Biology}
\begin{itemize}
	\item How do we wind the clock back from modern life to early life? From a biological perspective, the questions we're interested in are mostly 	how do we take everything we know about
	modern life and try to work as far back in
	time as we can? So how do we wind back the
	clock on modern life to understand what
	early life might have looked like?
	This involves taking a lot of
	phylogenetic or genetic perspectives to
	look back in time.
	\item  What does the composition, structure, and function of modern life tell us about the origin?
	\item  Which aspects of extant life are general and which are arbitrary?
	\item How do we apply modern evolutionary theory to the proto-life? Also in that vein, how do we take
	everything we know about modern
	evolutionary theory and start to apply that to things like protolife which
	might have a much less formal version of
	inheritence or genetics,
	how can we use evolutionary theory to
	think about the simplest early life which
	might be radically different but also is 
	almost certainly undergoing some sort of
	evolutionary process. So, what are the features that we see in
	modern life that are really essential to
	life through any origin and trajectory and
	which are contingent on the particular
	evolutionary history that we happened to
	have seen in--for more recent life.
\end{itemize}	


\subsection{Chemistry}
\begin{itemize}
	\item How does life arise from the huge space of chemical reactions and compounds? 
	So we understand that chemistry gives us
	this really vast set of possibilities,
	this really rich high-dimensional space,
	which is great for allowing something
	like life to form but it makes it very
	complicated to understand exactly what
	combinations and processes and
	trajectories are actually necessary or
	were the ones that led to the life that we
	have.
	\item What was early “living” chemistry like? What was that-- how was
	that possible in early Earth? And how do
	we start to define sort of living
	chemistry that isn't quite life?
	\item How do we go from complicated chemistry in an environment to the amazing chemical complexity of even the simplest cells? So even the simplest cells have a set of 	chemistry and feedback and dynamics and 	interconnections that is much more 
	complicated than things that we see in the 	environment and so how do we make that
	transition?
\end{itemize}

	
\subsection{Physics}

\begin{itemize}
	\item How do physical constraints, such as the laws of thermodynamics, bound the possibilities for life? How do we take everything we know about
	physics and use that to say what is and
	isn't possible for life, especially as it 
	first forms.
	\item What properties and processes are “easy” to obtain through physical dynamics alone?	So for example, we understand that in
	purely physical systems, it's possible to
	get very rich pattern information and
	dynamics and so how much-- how important
	are those sorts of naturally emergent--
	occurring phenomena to thinking about the
	first formation of life and how we start
	to apply some of those concepts of
	emergence and simple pattern formation to
	interact with
	how life began in the first place.
	\item How do we generalize physical concepts to understand life’s formation? So how do we port physical concepts over
	to biology in general and in particular,
	how do we port those ideas over to the
	origins of life?
\end{itemize}	

So, as you can see this is a very rich
set of questions coming out of a variety
of concept spaces across a huge range of
science and in this course we've gathered
people from this wide diversity of
scientific perspectives to really layout
the details of each of these questions in
specific ways and then bring those details
together in a way that we have a broader
and more complete understanding of how
life might have formed in the first place.



\section{Life}
Lecturer: David Baum
\subsection{Easy or hard?}
A key initial consideration is how easy, or hard, is it for life to emerge?
The extremes:
\begin{itemize}
	\item We need to wait billions of years for specific conditions: even hospitable planets lack life--Figure \ref{fig:luca}.
	\item New life can evolve quickly whenever conditions are ripe:  New life could emerge in the lab--Figure \ref{fig:zircons}.
\end{itemize}

Of course this isn't really a dichotomy, as there is a continuum from easy to hard.

\begin{figure}[H]
	\caption[Life is hard.]{Life is hard; Time is long. A key piece of evidence comes from looking at cellular life and tracing it back to a single ancestor--\gls{gls:LUCA}. If life were easy one might expect multiple independent trees.}\label{fig:luca} 
	\includegraphics[width=0.9\textwidth]{Luca}
\end{figure}

\begin{figure}[H]
	\caption[Life emerged quickly]{Life emerged quickly: The oldest rocks that could have retained evidence of life do have evidence of life. We now don't know any rock formation that doesn't have evidence of life. This figure depicts some including in zircons\cite{bell2015potentially}, and the isotopic ratios suggest the carbon was from life.}\label{fig:zircons} 
	\includegraphics[width=0.9\textwidth]{Zircons}
\end{figure}

 Even if life arose only once, it nevertheless happened remarkably early, since Figure \ref{fig:zircons} dates from shortly after the planet had cooled enough to provide liquid water. To help us think about that we can turn to Charles Darwin.
\begin{itemize}
	\item Life preempts Life: ''But if (and oh! what a big if!) we could conceive in some warm little pond with all sorts of ammonia and phosphoric salts,—light, heat, electricity etc present, that a protein compound was chemically formed, ready to undergo still more complex changes, at the present day such matter would be instantly devoured, or absorbed, which would not have been the case before living creatures were formed''--Charles Darwin\cite{darwin1871letter}. So the fact that life only emerged once does not necessarily mean that life is hard.
	
	\item Do we know that all life on this planet is cellular? Would we recognize alt-life?
	
	\item Some theories allow life to originate easily\cite{wachtershauser1988before}
\end{itemize}

\subsection{The meaning of ''life''}
In everyday discourse the term "life" is not ambiguous.
When we point at something and label it "alive,"
or an "instance of life," we usually know what we mean.
But, when it comes to the science and the philosophy of life,
this is not so trivial.
And, in fact,
scientists and philosophers
have debated for a long time
what we really mean by the term "life."

And, of course, we have
a very clear reference point.
The reference point is cellular life -
life as we know it on this planet,
which shares a number
of distinctive features.
And, we can make a very long list
of the features
that all life as we know it share,
and some of them are very specific.

\gls{gls:LAWKI} shares many features
\begin{itemize}
	\item They all use a nucleic acid code 	with the same bases-- (A,T/U,C, G)
	\item they all use the same 20 amino acids
	to make their proteins;
	\item and they're always
	the left-handed variants.
	\item They will have a very similar
	genetic code,
	\item similar machinery - ribosomes -
	for making proteins,
	\item and they have similar biochemistry, (e.g. \gls{gls:ATP})
	\item and even they share particular genes.
\end{itemize}

So, on the one hand, it's kind of exciting to realize that all life has these unique
and shared properties,because it tells us something fundamental about life as we know it,
which is that life as we know it traces back to a single common ancestor--Figure \ref{fig:LUCA_common}.
\begin{figure}[H]
	\caption{Life as we know it
		traces back to a single common ancestor}\label{fig:LUCA_common} 
	\includegraphics[width=\textwidth]{LUCA_common}
\end{figure}
So, there was, at some point in the past, an ancestral lineage that gave rise to
all three branches of life that we know of today - bacteria, archaea and eukaryotes.

And, the traits that we see in all life as we know it -
that list I just gave you and more -all predate the last universal
common ancestor,
the last organism that was ancestral
to all three of those lineages.
So, that's an exciting
and important insight,
but it also poses a slight problem,
which is -
if life as we know it
has a very long list of shared traits.
But, it doesn't necessarily guide us
as to ask the question - suppose you found
some other lifelike system,maybe on another planet
or in some strange environment,you wanted to decide -
"Is this thing alive? Is it an instance of life?" -you presumably wouldn't care about
this full laundry list.These particular traits are the result
of the historical factors that occurred in the origins of this life.
and we want some more general understanding of life
to answer the question - is something else
another instance of life?

So, people have tried to approach this by whittling down those specific traits
to generalities - general features - that life seems to have
that distinguish these instances of life from instances of non-life -
inanimate matter. And if, you open any sort
of biology textbook,
you'll have a list
of the features of life,
and they vary in number
from five, seven, eight, whatever--Figure {fig:lawki-focus}.

\begin{figure}[H]
	\caption{Focus on generic features...but which?}\label{fig:lawki-focus} 
	\includegraphics[width=\textwidth]{lawki-focus}
\end{figure}
And, this has been the starting point
for a process in the origin of life field
of thinking about - how low can we go?
What, at core, are the essential features
of something to be considered to be
an instance of life in the general sense?

Now, there might be some
difference of opinion,
but, by and large,
the field has converged
on two key features,
which are captured
in this definition I've given here.
\gls{gls:life} is a self-propagating chemical system capable of undergoing adaptive evolution--adapted from \cite{deamer1994origins}.

This definition is based on one that was kind of established by NASA to help guide them in the question of looking on other planets and deciding whether there is life there.

\begin{itemize}
	\item life is a self-propagating 	chemical system,	meaning it's a system that makes more	of itself over time - or least can;
	\item  we expect 	that this system 	is capable of undergoing
	adaptive evolution.--Figure \ref{fig:AdaptiveEvolution}
	\item Self-propagation and evolution are key
\end{itemize}

So, let's look at those two pieces in turn.


The idea of self-propagation is that you have some kind of chemical system that can make more of itself and occupy additional areas of space. It doesn't much matter
whether we're thinking about growth, which is where we have a single,
living protoplasm that expands to encompass more space, 
or if we're thinking about
the process of making multiple cells
from a single parent cell.
The only difference
between these phenomena
is whether or not the newly formed
protoplasm - living material -
either is not sort of
packaged up into cells.
And so, what we expect of anything
that we would label to be "alive" -
is that it has some capacity
to propagate itself spatially.
But, of course,
that isn't really enough
because we know of quite a few
systems that can self-propagate,
but we usually
wouldn't call them "alive" -
for example, a crystallization process.
\begin{figure}[H]
	\caption[Adaptive Evolution]{Adaptive evolution entails the potential to complexify and become more out-of-equilibrium}\label{fig:AdaptiveEvolution} 
	\includegraphics[width=0.9\textwidth]{AdaptiveEvolution}
\end{figure}

And, that's where the second part
of the definition comes in -
and that is this idea that things
that we want to consider to be alive
also have the capacity
for adaptive evolution...
which means that
they have the capacity
to get better at
self-propagating over time.
This is an important idea
and something that we don't see in,
for example, crystallization processes,
because it explains
how it is that life as we know it
became so complicated
and out of equilibrium
with their chemical environments.
Cells and living systems that we know
are quite sophisticated
and certainly far from being typical of
the chemical environment around them.
And, this is possible because,
during the adaptive process,
the variants arise,
and the fate of those variants
is independent of whether they are
of raised or lower complexity.
Natural selection,
the driving force of adaptive evolution,
will select the trait that
confers higher fitness
whether or not
it's a more complicated trait.
Furthermore, there are many instances
in which we know
that the more complicated variants
have an advantage.
As a result,
over long, long periods of time,
adaptive evolution can explain
how a simple self-propagating system
can become more and more complex.

So, putting these
two pieces together,
it's fair to say that
the origin of life field,
in general, has a great deal
of attention...
that it pays to these two properties -
self-propagation and adaptive evolution.
We try and understand
how they come about,
and we try and understand the kinds
of planetary or chemical conditions
in which we might expect
self-propagation and evolution
to emerge spontaneously.



\section{Constraining Chemical Complexity to Form Life}

Zach Adam
University of Arizona

We'll be talking about huge chemical complexity, and constraining the complexity to form life. When we look at the deep past we find some interesting challenges regarding how life could have originated from a non-living environment. Most matter isn't alive: how to bridge the two states of matter?

\begin{figure}[H]
	\caption{Chemical Complexity}\label{fig:ChemicalComplexity} 
	
	\begin{subfigure}[b]{0.45\textwidth}
		\centering
		\caption{When we first learn chemistry, we use stick and ball model}
		\includegraphics[width=\textwidth]{ChemicalComplexity}
	\end{subfigure}
	\begin{subfigure}[b]{0.45\textwidth}
		\centering
		\caption{But atoms really have many levels, all interacting. The outer layers of electrons give rise to the "sticks" in the na\"ive model.}
		\includegraphics[width=\textwidth]{ChemicalComplexity1}
	\end{subfigure}
\end{figure}

Let's start with a canonical type of molecule--Figure \ref{fig:NucleotideMolecule}. When we zoom in a bead we see that it represents three molecules. If we change even one atom, or link the atoms differently, the molecule may not function.
\begin{figure}[H]
	\caption{Nucleotide Molecule }\label{fig:NucleotideMolecule}
	
	\begin{subfigure}[b]{0.45\textwidth}
		\centering
		\caption{RNA Molecule: one bead expands to 3 molecules. Change one atom, likely to change behaviour of entire molecule!}
		\includegraphics[width=\textwidth]{NucleotideMolecule}
	\end{subfigure}
	\begin{subfigure}[b]{0.45\textwidth}
		\centering
		\caption{Zoom into Adenine Molecule: change one thing, not adenine any more!}\label{fig:adenine}
		\includegraphics[width=0.6\textwidth]{AdenineMolecule}
	\end{subfigure}
\end{figure}

Let's zoom in further on the Adenine molecule, Figure \ref{fig:adenine}: change any atom and it wouldn't be adenine; now zoom in even further, Figure \ref{fig:MolecularInteractions}, and we see more interactions between atoms and with solvent.

\begin{figure}[H]
	\caption{Zoom in further, and atoms interact with each other, and with solvents!}\label{fig:MolecularInteractions} 
	\includegraphics[width=0.9\textwidth]{MolecularInteractions}
\end{figure}

If you want to think about generating these molecules from scratch, you need to think about all these interactions and their effects.

There are a number of ways origins of life researchers try to navigate this complectity and design experiments that get to the root of how life could have originated from a non-living state.

\begin{itemize}
	\item Constrain Location 
	\begin{itemize}
		\item Advantages
		\begin{itemize}
			\item Many possibilities for complex interactions
			\item Experiments can be designed by analogy with modern environments.
		\end{itemize}
		\item Disadvantages: mostly implausible to assume pure reactants in natural settings
	\end{itemize}
	\item Constrain Reactants
	\begin{itemize}
		\item Advantages: limit to needed molecules and needed amounts.
		\item Disadvantages: resulting network is not very complex or robust
	\end{itemize}	
	\item Constrain Energy Sources
		\begin{itemize}
		\item Advantages: processes are not location- or reactant-specific; many 		outcomes are possible
		\item Disadvantages: Difficult or impossible to predict outcomes of
		processes that cross multiple object levels
	\end{itemize}
\end{itemize}

Examples
\begin{itemize}
	\item Constrained Location Examples \begin{itemize}
		\item Hydrothermal Vents\cite{martin2006origin}
		\item Atmosphere (Impactor/Shock Synthesis, Lightning,
		Insolation)\cite{chyba1992endogenous} and \cite{miller1959organic}
	\end{itemize}
	\item Constrained Reactant Examples 
	\begin{itemize}
		\item RNA World\cite{powner2009synthesis},
		\item Pyrite-mediated synthesis\cite{wachtershauser1993cradle}
		\item Borate-mediated synthesis\cite{grew2011borate}
	\end{itemize}
	\item Constrained Energy Examples
	\begin{itemize}
		\item \gls{gls:serpentinization}\cite{schrenk2013serpentinization}
		\item Solar flares\cite{airapetian2016prebiotic}
		\item Radioactivity\cite{yi2018radiolytic} and \cite{adam2018estimating}
	\end{itemize}
\end{itemize}
\section{Geological Conditions, Change, and Chaos}

Zach Adam, University of Arizona\cite{spencer2017growth}

What is the source of our understanding of conditions when we thought life originated?
\begin{figure}[H]
	\caption{Earth's Structure: includes everything in Figure, plus minerals, etc. Can only sample surface, plus a few km down.}\label{fig:EarthStructure} 
	\includegraphics[width=0.9\textwidth]{EarthStructure}
\end{figure}

\begin{figure}[H]
	\caption{Geologic Record is inseparable from history of life}\label{fig:GeologicRecord} 
	
	\begin{subfigure}[b]{0.45\textwidth}
		\centering
		\caption{Great oxidation event changed all sorts of stuff. We need oxygen, but other critters had to adapt or die. Oxidation changed minerals.}
		\includegraphics[width=\textwidth]{GeologicRecord}
	\end{subfigure}
	\begin{subfigure}[b]{0.45\textwidth}
		\centering
		\caption{Different models for crustal accumulation.}
		\includegraphics[width=\textwidth]{GeologicRecord1}
	\end{subfigure}
\end{figure}

\begin{figure}[H]
	\caption{Another Level Down: different models for accumulation of crust. Not simple accumulation!}\label{fig:AnotherLevelDown} 
	\includegraphics[width=0.9\textwidth]{AnotherLevelDown}
\end{figure}

Only direct record for events that sow chaos and processes of prolonged gradual change

What does this mean for origins of life research?
\begin{itemize}
	\item Limited direct rock data
	\begin{itemize}
		\item 	Sampling and preservation
		biases abound!
			\item No rock packages to
		provide environmental
		context or views into surface
		conditions.
		\item 	No rocks and few minerals
		from time surrounding life’s
		origins.
	\end{itemize}
	\item Complementary lab data
	\begin{itemize}
		\item Experiments are critical for
		filling in where the rock
		record is not available.
			\item Unclear whether powerful
		events/energy are more
		constructive than destructive.
	\end{itemize}
\end{itemize}
\section{Pattern Formation in Chemical Systems}
Chris Kempes

\begin{itemize}
	\item What properties and processes are easy to obtain through physical dynamics alone?
	
	\item How might this make it easy for lifelike things to begin?
	
	\item Reaction Diffusion Equations\cite{sfi_grayscott2018}
\end{itemize}

Describe reaction of $U$ everywhere in space.
\begin{align*}
\frac{\partial U}{\partial t} =&   \underbrace{\overbrace{D_V}^\text{diffusivity} \underbrace{ \nabla^2 U }_\text{difference in fluxes}}_\text{diffusion term} + \underbrace{ F(U)}_\text{reaction term}
\end{align*}

With two chemicals...
\begin{align*}
\frac{\partial U}{\partial t} =&   D_U \nabla^2 U + F(U,V)\\
\frac{\partial V}{\partial t} =&   D_V \nabla^2 V + G(U,V)
\end{align*}

Consider two reacting species, $U$ and $V$, and an inert product $P$.
\begin{align*}
U + 2V \Longrightarrow& 3V\\
V \Longrightarrow& P\\
\frac{\partial U}{\partial t} =& -U V^2 + \overbrace{F}^\text{Fixed flow into system (chemostat)} (1 - U) + D_u \nabla^2 U \\
\frac{\partial V}{\partial t} =& U V^2 - (F +\underbrace{f}_\text{loss to make $P$}) V + D_v \nabla^2 V 
\end{align*}

\begin{figure}[H]
	\caption{Grey Scott Reaction}
	
	\begin{subfigure}[b]{0.45\textwidth}
		\centering
		\caption{Starting configuration}
		\includegraphics[width=\textwidth]{GreyScottInitial}
	\end{subfigure}
	\begin{subfigure}[b]{0.45\textwidth}
		\centering
		\caption{Evolves toward steady state,  where high concentration close to low concentration. Segregation.}
		\includegraphics[width=\textwidth]{GreyScottFinal}
	\end{subfigure}
\end{figure}

Other starting values give cell-like behaviour (including division), so maybe get some ''lifelike" behaviour from physics and simple chemistry.

\section{The Central Dogma of Biology}


\subsection{The Central Dogma of Biology}

How do we unwind present day life to learn about origins? Focus on most common properties of all life, especially the Central Dogma.\cite{crick1958biological} \cite{crick1970central}

\begin{figure}[H]
	\caption{The Central Dogma after \cite{crick1970central}}\label{fig:CentralDogma} 
	\includegraphics[width=0.9\textwidth]{CentralDogma}
\end{figure}


\begin{figure}[H]
	\caption{Ribosome is conserved across life, but appears to have onion structure, built up by evolution. Ribosome started with common core, and more added.\cite{hsiao2009peeling}.}\label{fig:Ribosome} 
	\includegraphics[width=0.9\textwidth]{Ribosome}
\end{figure}
Williams argues that we can use the onion, Figure \ref{fig:Ribosome}, to unwind evolution.

\begin{figure}[H]
	\caption{Ribosome Phases after \cite{petrov2015history}}\label{fig:RibosomePhases} 
	\includegraphics[width=0.5\textwidth]{RibosomePhases}
\end{figure}

Carl Woese used this to build new Tree of Life.

\begin{figure}[H]
	\caption{\gls{gls:TOL}}\label{fig:TOL} 
	\includegraphics[width=0.9\textwidth]{TOL}
\end{figure}



\subsection{The Efficiency of the Central Dogma of Biology}

\gls{gls:landauer_bound} \glsdesc{gls:landauer_bound}\cite{bennett2003notes}\cite{landauer1961irreversibility}. 

\begin{figure}[H]
	\caption{Take unordered set of letters and try to write as specific string.}\label{fig:LandauerRibosome} 
	\includegraphics[width=0.9\textwidth]{LandauerRibosome}
\end{figure}

\begin{itemize}
	\item Life is only 20 times less efficient that physical limit\cite{kempes2017thermodynamic}
	\item Computers are 20 times worse than Landauer's bound!
\end{itemize}

\begin{figure}[H]
	\caption{For small cell, \gls{gls:DNA} and proteins take up (nearly) entire volume; for large cells, Ribosome does same thing! So we have bounds on allowable volume.\cite{kempes2016evolutionary}}\label{fig:RibosomeTradeoffs} 
	\includegraphics[width=0.9\textwidth]{RibosomeTradeoffs}
\end{figure}

\begin{figure}[H]
	\caption{A more efficient ribosome would allow larger cells; less efficient might not work at all.}\label{fig:RibosomeTradeoffsEarly} 
	\includegraphics[width=0.9\textwidth]{RibosomeTradeoffsEarly}
\end{figure}

\section{Biological Similarity}

Sarah Mauer: ''By examining and deconstructing the commonalities between modern living organisms, we can understand the requirements of first life.''

\begin{itemize}
	\item L-amino acids for all proteins;
	\item D-sugars for all sugars and nucleic acids.
	\item Use membranes: amphiphilic molecules with hydrophobic tail and hydrophilic end.
	\item All cells use proteins, same 20 amino acids.
\end{itemize}

\begin{figure}[H]
	\caption{Comparison of similarity allows for the
		construction of a tree, outlining evolution. Large part of conserved set involved in Central Dogma, and also metabolic.}\label{fig:Phylogeny} 
	\includegraphics[width=0.9\textwidth]{Phylogeny}
\end{figure}

\begin{figure}[H]
	\caption{Horizontal gene transfer complicates things!}\label{fig:PhylogenyHorizontal} 
	\includegraphics[width=0.9\textwidth]{PhylogenyHorizontal}
\end{figure}

\section{What is Life?}

\subsection{Constraining the Definition of Life}

Lecturer:  Sara Imari Walker

Schrödinger wondered whether we could explain life using physocs as currently known.

''... living matter, while not eluding the laws of physics as established up to date, is likely to involve other laws of physics hitherto unknown''--Erwin Schrödinger\cite{schrodinger1944life}.

WE don't really know what life looks like, but we might ask critically what are the examples of life on Earth, and how we can use them to build a unified theory of what life is--a real predictive theory that allows us to understand nit just life on this planet, but also life on other worlds. In physics we have this wonderful hierarchy of theories--Figure \ref{fig:physics:unifications}, but it doesn't have anything to say about complex systems or about us: ''The theory of everything is a theory of everything except of those things that theorize''--David Krakauer.

So the challenge is how can we think about how we can approach an explanatory theory for life, and whether we can draw inspiration from the history of physics. So it is constructive to look at examples of life on earth. Is life really a natural kind? Are some systems more alive than others?

\begin{figure}[H]
	\caption{History of Unifications in Physics}
	\includegraphics[width=0.9\textwidth]{Unifications}\label{fig:physics:unifications}
\end{figure}


Examples of "Life"
\begin{itemize}
	\item The Cell as a unit of Life--Figure \ref{fig:cell}
	\item Metabolism--Figure \ref{fig:metabolism}
	\item Tardigrade--an extremophile that can live in space--Figure \ref{fig:tardigrade}. Maybe we should think about the widest set of conditions under which life \textit{can} exist.
	\item Two-headed planarian worms--Figure \ref{fig:2headed:planaria}. What is the limit for viable life?\cite{levin2019planarian}
	\item What if we replace RNA/DNA with XNA?
	\item We can consider scales of organization, as with Social Insects--Figure \ref{fig:social:insects}. Is the super-organism, the colony, "alive"? Is life something that emerges in chemistry, but can exist at higher levels?\cite{pratt2015psychology}
	\item Is a City alive--Figure \ref{fig:city}?
	\item Is there life on the scale of the Planet\cite{lovelock1974atmospheric}--Figure \ref{fig:gaia}?
\end{itemize}

\begin{figure}[H]
	\caption{The Cell as a unit of Life}\label{fig:cell}
	\includegraphics[width=0.9\textwidth]{Cell}
\end{figure}

\begin{figure}[H]
	\caption{Metabolism}\label{fig:metabolism}
	\includegraphics[width=0.9\textwidth]{Metabolism}
\end{figure}

\begin{figure}[H]
	\caption{Tardigrade--an extremophile that can live in space}\label{fig:tardigrade}
	\includegraphics[width=0.9\textwidth]{Tardigrade}
\end{figure}


\begin{figure}[H]
	\caption{Two Headed Planaria}\label{fig:2headed:planaria}
	\includegraphics[width=0.9\textwidth]{TwoHeadedPlanaria}
\end{figure}

\begin{figure}[H]
	\caption{Social Insects: is the super-organism "alive"?}\label{fig:social:insects}
	\includegraphics[width=0.9\textwidth]{SocialInsects}
\end{figure}

\begin{figure}[H]
	\caption{City}\label{fig:city}
	\includegraphics[width=0.9\textwidth]{City}
\end{figure}

\begin{figure}[H]
	\caption{Is there Life on the scale of a Planet?}\label{fig:gaia}
	\begin{subfigure}[b]{0.45\textwidth}
		\caption{Gaia}
		\includegraphics[width=\textwidth]{Globe1}
	\end{subfigure}
	\begin{subfigure}[b]{0.45\textwidth}
		\caption{Network representation of the global inventory of
			enzymatically catalyzed biochemical reactions}
		\includegraphics[width=\textwidth]{Globe2}
	\end{subfigure}
\end{figure}


\subsection{Weird Life}

Lecturer: Sarah Maurer

The material in this section is entirely speculative.

\begin{itemize}
	\item No Cells--diffusion systems. But it is very restrained: have to get the right values for parameters.
	\item Membrane-less Cells? Organization of 	chemical gradients\cite{hollants2011life}\cite{kim2001life}
	\begin{itemize}
		\item Mineral surfaces
		\item Coacervates
		\item Oil droplets
		\item Aerosols
	\end{itemize}
	\item No water? 
	\begin{itemize}
		\item Polar solvents\cite[Chapter 6]{board2007limits}--Figure \ref{fig:no:water}. Water, ammonia, and sulphuric acid can:
		\begin{itemize}
			\item drive formation of carbon-carbon bonds;
			\item hydrogen bond.
		\end{itemize}
		\item Non-polar solvents\cite{cejkova2014dynamics}--Figure \ref{fig:non:polar}.
	\end{itemize}
	\item No liquid?\cite[Chapter 6]{board2007limits}
	\begin{itemize}
		\item Solids
		\begin{itemize}
			\item Ices?
			\item Very slow metabolic rates (longer time scales)
		\end{itemize}
		\item Gases
		\begin{itemize}
			\item Higher temperatures
			\item Less stable large molecules
			\item Much larger (galaxy level?)
		\end{itemize}
	\end{itemize}
\end{itemize}


\begin{figure}[H]
	\caption{No Water? Polar solvents}\label{fig:no:water}
	\includegraphics[width=0.9\textwidth]{NoWater}
\end{figure}

\begin{figure}[H]
	\caption{No water? Non-polar solvents}\label{fig:non:polar}
	\begin{subfigure}[t]{0.5\textwidth}
		\caption{Titan--liquid ethane/methane--very cold}
		\includegraphics[width=\textwidth]{Titan}
	\end{subfigure}
	\begin{subfigure}[t]{0.5\textwidth}
		\caption{Io--liquid sulfur--Hotter than Earth}
		\includegraphics[width=\textwidth]{Io}
	\end{subfigure}
\end{figure}

Open Questions

\begin{itemize}
	\item What other substrates could life exist in?
	\item Would we recognize it?
\end{itemize}


\begin{appendices}
\section{Selection Theory}
Chris Kempes

\subsection{Quasispecies equation}
\begin{itemize}
	\item How do we think about evolution in a simple, mathematical way?
	\item Start with binary Genome. One string can change into another via point mutations.
\end{itemize}


\begin{figure}[H]
	\caption{Topology of genome -- point mutations represent edges}\label{fig:GenomeTopology} 
	\includegraphics[width=0.9\textwidth]{GenomeTopology}
\end{figure}

\begin{figure}[H]
	\caption{Fitness Landscape, showing maximum fitness.}\label{fig:FitnessLandscape} 
	\includegraphics[width=0.9\textwidth]{FitnessLandscape}
\end{figure}

Introduce \emph{Quasispecies equation}.
\begin{align*}
\dot x_i =& f_i x_i - \phi x_i \text{, where $x_i$ is frequency of sequence $i$, $f_i$ is fitness, $\phi$ is death term.}\\
\sum_{i=1}^{n} \dot x_i =& 0 \text{, assume constant population}\\
\sum_{i=1}^{n} f_i - \phi \sum_{i=1}^{n} x_i =& 0\\
\phi =& \sum_{i=1}^{n}x_i f_i\text{, since $\sum_{i=1}^{n}x_i = 1$}
\end{align*}
Hence $\phi$ is \emph{average fitness}. But this leaves out \emph{mutation!} Include $q_{ij}$, the probability of mutation from $i$ to $j$.
\begin{align*}
\dot x_i =& \sum_{j=1}^{n} f_j q_{ij}x_j - \phi x_i\text{, full quasispecies equation} 
\end{align*}

NB. See Figure \ref{fig:GenomeTopology} - point mutations much more likely than 2-point, etc.

\subsection{What are the basic ways in which we think about evolution?}

\begin{itemize}
	\item How do we generalize evolution?
	\item How do we use this to think about origins of life? 
\end{itemize}

\begin{figure}[H]
	\caption{Simplify to \emph{Master Sequence}, $f_0$ with all less fit sequences lumped as $f_1$.}\label{fig:SteadyStateSolution} 
	\includegraphics[width=0.9\textwidth]{SteadyStateSolution}
\end{figure}

\begin{align*}
\dot x_i =& \sum_{j=1}^{n} f_j q_{ij}x_j - \phi x_i\\
f_0 >&1, f_1 = 1\\
q_{00} =& (1-u)^L \triangleq q\text{, where $u$ is mutation rate, $L$ = length of genome}\\
q_{01} =& 1-q\\
q_{11} =& 1\text{, $L$ is large, so 1$\rightarrow$1} \\
q_{10} =& 0\text{, $1\nrightarrow0$}\\
\phi =& \sum_{i=1}^{n}x_i f_i\\
\text{Whence,}\\
\dot{x_0}=&f_0 q x_0 - \phi x_0\text{,}\\
\dot{x_1}=&f_0 (1-q) x_0 + x_1 -\phi x_1\text{, and}\\
\phi =& f_0 x_0 + x_1 \text{. But} \\
x_0 + x_1 =& 1\text{, so we have}\\
\dot{x_0}=& x_0 \big[f_0 q -1 - x_0 (f_0 - 1)\big]
\end{align*}
Now we can look for a steady state solution:
\begin{align*}
x_0^* =& \frac{f_0 q - 1}{f_0 - 1}\text{, se we need}\\
f_0q >& 1 \text{, if the master sequence is to persist. I.e.}\\
u <& log(f_0)\\
log(f_0) >& -L log(1-u)\text{. Now we assume $L<<1$, and find}\\
u <& \frac{log(f_0)}{L}\text{, which is known as the \emph{Error Threshold.}} 
\end{align*}
The Error Threshold is the \emph{maximum mutation rate that allows for adaptive evolution!} NB: it is more sensitive to $L$ than to $f_0$.

\subsection{Quasispecies and Error Catastrophe}

Michael Lachmann

We look at growth of types, e.g.:
\begin{itemize}
	\item Genome in population
	\item Molecule in test-tube.
\end{itemize}

\begin{align*}
\vec{x}^{\,t} =& W M \vec{x}\text{, where M represents mutations, W growth, e.g.}\\
M =& \begin{bmatrix}
1-\mu & 0\\
\mu& 1
\end{bmatrix}&
\end{align*}
Let us diagonalize WM
\begin{align*}
WM =& VDV^{-1}\\
(WM)^t =& V D^t V^{-1}
\end{align*}
So $D^t$ is dominated by its largest eigenvalue.

Figures \ref{fig:ErrorCatastrophe1} and \ref{fig:ErrorCatastrophe2} show that master sequence lost if mutation rate is too high. 
\begin{figure}[H]
	\caption{Hamming Distance.}\label{fig:ErrorCatastrophe1} 
	\includegraphics[width=0.9\textwidth]{ErrorCatastrophe1}
\end{figure}
\begin{figure}[H]
	\caption{Error Catastrophe.}\label{fig:ErrorCatastrophe2} 
	\includegraphics[width=0.9\textwidth]{ErrorCatastrophe2}
\end{figure}


See \cite{eigen1978hypercycle}, \cite{eigen1988molecular}, \cite{eigen2002error}, \cite{crotty2001rna}, and \cite{stadtler2002fitness_landscapes}.

See also \cite{wessner2010origins}


	\section{Nonequilibrium Physics}

Lecturer: Eric Smith, External Professor, Santa Fe Institute.\\
\\
Topics covered:
\begin{itemize}
	\item The equilibrium concept of phase transition
	
	\item How phase transitions explain robust patterns
	
	\item Why equilibrium isn’t enough to understand life
	\item Phase transitions in dynamical systems ''frozen motion''
\end{itemize}

The ''ordinary'' response of thermodynamic systems to controls. E.g. lava: Viscosity increases smoothly as temperature is lowered. Phase transitions are different
\begin{itemize}
	\item Water does not become harder as it is cooled
	\item It turns to ice suddenly (critical temperature)
	\item The average direction of pointing of the ice sharply increases from zero at the freezing temperature
	\item The direction and strength of the crystal is called the ''Order 	Parameter'' of the transition
	\item Change is sudden because ''you can’t have half a symmetry''
	\begin{itemize}
		\item A direction either exists or it doesn’t
		\item All frozen directions are equivalent
	\end{itemize}
	\item Phase transitions, cooperatively-maintained states, and robustness
	\begin{itemize}
		\item Diamonds are hard because many atoms lock each other in place
		\item The order of the crystal is a ''robust'' property of freezing
	\end{itemize}
\end{itemize}

Evolution happens on a background of robust architectures
\begin{itemize}
	\item Universal small metabolites
	\item RNA and proteins
	\item Cellular and genomic individuality
\end{itemize}

Equilibrium ideas are not enough to explain the robust order of life (Chicken vs. chicken soup)


The Miller-Urey synthesis of amino acids\cite{miller1959organic}

Life is made of interlocking structures and processes: can phase transition ideas
be applied to these?

What might be the order parameters of life?

\begin{itemize}
	\item They would be chemical and energetic
	\item They would involve interdependent	structure and process
\end{itemize}

The characteristic molecules
\begin{itemize}
	\item Unchanging universal roles 	for small metabolites
	\item Key macromolecules such as RNA
\end{itemize}

The great biogeochemical cycles
\begin{itemize}
	\item Life alters cycling of Carbon, Nitrogen, Sulfur, and more
	\item New compounds are also formed of 	these elements
\end{itemize}

\begin{figure}[H]
	\caption[The great biogeochemical cycles]{The great biogeochemical cycles\cite{falkowski2008microbial}}\label{fig:biogeochemical} 
	\includegraphics[width=0.9\textwidth]{biogeochemical}
\end{figure}



\begin{figure}[H]
	\caption[Earth’s energy throughput]{Earth’s energy throughput: the Biosphere changes the way a planet converts sunlight into heat.\cite{meadows2005modelling}}\label{fig:EnergyThroughput} 
	\includegraphics[width=0.9\textwidth]{EnergyThroughput}
\end{figure}

The emergences of individualities
\begin{itemize}
	\item Individuality takes many 	forms
	\item The order parameters in individual-based systems 	are proper names
\end{itemize}

Take-home messages from the lecture:
\begin{itemize}
	\item Phase transitions are one way natural systems  spontaneously form order
	\item The order is robust due to mutual reinforcement
	\item Phase transitions can also lead to spontaneous order in processes like fractures
	\item Candidates for living order parameters include chemical cycles and individuality
\end{itemize}


\end{appendices}
% end of text 

% glossary: may need command makeglossaries.exe origins1
\printglossaries

% bibliography goes here
 
\bibliographystyle{unsrt}
\addcontentsline{toc}{section}{Bibliography}
\bibliography{origins,wikipedia}


\end{document}
