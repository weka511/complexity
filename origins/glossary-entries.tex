% Glossary entries

\newacronym{gls:ATP}{ATP}{Adenosine triphosphate}

\newacronym{gls:CSS}{CSS}{Cellular Signature Structure}

\newacronym{gls:DNA}{DNA}{Deoxyribonucleinc Acid}

\newacronym{gls:ESP}{ESP}{Eukaryotic Signature Protein}

\newacronym{gls:LAWKI}{LAWKI}{Life as we know it}

\newacronym{gls:LBA}{LBA}{Long Branch Attraction}

\newacronym{fls:LGT}{LGT}{Lateral Gene Transfer}

\newacronym{gls:LECA}{LECA}{Last Eukaryotic Common Ancestor}

\newacronym{gls:LUCA}{LUCA}{Last Universal Common Ancestor}

\newacronym{gls:ML}{ML}{Maximum Likelihood}

\newacronym{gls:MP}{MP}{Maximum Parsimony}

\newacronym{gls:RNA}{RNA}{Ribonucleinc Acid}

\newacronym{gls:SFI}{SFI}{Santa Fe Institute}

\newacronym{gls:TOL}{TOL}{Tree of Life}

\newglossaryentry{gls:16S-ribosomal-RNA}{
	name={16S ribosomal RNA},
	description={16S ribosomal RNA (or 16S rRNA) is the component of the 30S small subunit of a prokaryotic ribosome that binds to the Shine-Dalgarno sequence. The genes coding for it are referred to as 16S rRNA gene and are used in reconstructing phylogenies, due to the slow rates of evolution of this region of the gene. Carl Woese and George E. Fox were two of the people who pioneered the use of 16S rRNA in phylogenetics in 1977--\cite{wiki:16SrRNA}.}}

\newglossaryentry{gls:aldose} {
	name=aldose,
	description={An aldose is a monosaccharide (a simple sugar) with a carbon backbone chain with a carbonyl group on the endmost carbon atom, making it an aldehyde, and hydroxyl groups connected to all the other carbon atoms. Aldoses can be distinguished from \glspl{gls:ketose}, which have the carbonyl group away from the end of the molecule, and are therefore ketones--\cite{ wiki:aldose}}
}

\newglossaryentry{gls:amphiphile}{
	name={amphiphile},
	description={An amphiphile is a chemical compound possessing both hydrophilic (water-loving, polar) and lipophilic (fat-loving) properties. Such a compound is called amphiphilic or amphipathic--\cite{wiki:amphiphile}}}

\newglossaryentry{gls:arrheniusequation}{
	name={Arrhenius Equation},
	description={In physical chemistry, the Arrhenius equation is a formula for the temperature dependence of reaction rates. The equation was proposed by Svante Arrhenius in 1889, based on the work of Dutch chemist Jacobus Henricus van 't Hoff who had noted in 1884 that van 't Hoff equation for the temperature dependence of equilibrium constants suggests such a formula for the rates of both forward and reverse reactions--\cite{wiki:arrheniusequation}
	\begin{align*}
	k =& e^{\frac{-E_a}{k_B T}}\text{, where}\\
	k =& \text{rate constant,}\\
	T =& \text{the absolute temperature (in Kelvin),}\\
	A =& \text{the pre-exponential factor, a constant for each chemical reaction. According to collision theory, A is the frequency of collisions in the correct orientation,}\\
	E_a =& \text{the activation energy for the reaction,}\\
	k_B =& \text{the Boltzmann constant.}
	\end{align*}
}}

\newglossaryentry{gls:autocatalysis}{
	name={autocatalysis},
	description={Chemical self production: the ability of a set of chemical species to make more of that same set, by undergoing a series of chemical 	reactions--\cite{sfi2019}}}

\newglossaryentry{gls:autocatalyst}{
	name={autocatalyst},
	description={Something that acts as a \gls{gls:catalyst} for its own production}}

\newglossaryentry{gls:catalyst}{
	name={catalyst},
	description={Something that gets returned after a sequence 	of reactions}}

\newglossaryentry{gls:cellular:signature:structures}{
	name={Cellular Signature Structures},
	description={A cellular compartment that  distinguished  eukaryotes  from  prokaryotes.\cite{kurland2006genomics}}}

\newglossaryentry{gls:clathrate}{
	name={clathrate},
	description={Inclusion compound in which the guest molecule is in a cage
		formed by the host molecule or by a lattice of host molecules--\cite{book2014compendium}}}

\newglossaryentry{gls:coacervate}{
	name={coacervate},
	description={Polar bulk phase separated from the water phase}}

\newglossaryentry{gls:deamination}{
	name={deamination},
	description={Deamination is the removal of an amino group from a molecule. Enzymes that catalyse this reaction are called deaminases--\cite{wiki:deamination}}}

\newglossaryentry{gls:DNAmethyltransferase}
	{name={DNAmethyltransferase},
	description={In biochemistry, the DNA methyltransferase (DNA MTase) family of enzymes catalyze the transfer of a methyl group to DNA. DNA methylation serves a wide variety of biological functions. All the known DNA methyltransferases use S-adenosyl methionine (SAM) as the methyl donor--\cite{wiki:DNAmethyltransferase}}}

\newglossaryentry{gls:endomembrane}{
	name={endomembrane},
	description={The endomembrane system is composed of the different membranes that are suspended in the cytoplasm within a eukaryotic cell. These membranes divide the cell into functional and structural compartments, or organelles. In eukaryotes the organelles of the endomembrane system include: the nuclear membrane, the endoplasmic reticulum, the Golgi apparatus, lysosomes, vesicles, endosomes, and plasma (cell) membrane among others. The system is defined more accurately as the set of membranes that form a single functional and developmental unit, either being connected directly, or exchanging material through vesicle transport. Importantly, the endomembrane system does not include the membranes of chloroplasts or mitochondria, but might have evolved from the latter--\cite{wiki:endomembrane}}}

\newglossaryentry{gls:enthalpy}{
	name={enthalpy},
	description={A property of a thermodynamic system, is equal to the system's internal energy plus the product of its pressure and volume. In a system enclosed so as to prevent matter transfer, for processes at constant pressure, the heat absorbed or released equals the change in enthalpy--\cite{ wiki:Enthalpy}}}

\newglossaryentry{gls:eukaryotic:signature:protein}{
	name={Eukaryotic Signature Protein},
	description={A protein that is found in eukaryotic cells but has no significant homology to proteins in Archaea and Bacteria--cite{hartman2002origin}}}

\newglossaryentry{gls:furan}{
	name={furan},
	description={Furan is a heterocyclic organic compound, consisting of a five-membered aromatic ring with four carbon atoms and one oxygen. Chemical compounds containing such rings are also referred to as furans--\cite{wiki:furan}}
}

\newglossaryentry{gls:gibbs-free}{
	name={Gibbs free energy},
	description={A thermodynamic potential that can be used to calculate the maximum of reversible work that may be performed by a thermodynamic system at a constant temperature and pressure (isothermal, isobaric). The Gibbs free energy is the maximum amount of non-expansion work that can be extracted from a thermodynamically closed system (one that can exchange heat and work with its surroundings, but not matter); this maximum can be attained only in a completely reversible process--\cite{wiki:gibbs-free}}}

\newglossaryentry{gls:heterotachy}{
	name={heterotachy},
	description={Heterotachy refers to variations in lineage-specific evolutionary rates over time. In the field of molecular evolution, the principle of heterotachy states that the substitution rate of sites in a gene can change through time. It has been proposed that the positions that show switches in substitution rate over time (that is, heterotachous sites) are good indicators of functional divergence. However, it appears that heterotachy is a much more general process, since most variable sites of homologous proteins with no evidence of functional shift are heterotachous--\cite{wiki:heterotachy}}}

\newglossaryentry{gls:hydrogenosome }{
	name={hydrogenosome },
	description={A hydrogenosome is a membrane-enclosed organelle of some anaerobic ciliates, trichomonads, fungi, and animals. The hydrogenosomes of trichomonads (the most studied of the hydrogenosome-containing microorganisms) produce molecular hydrogen, acetate, carbon dioxide and ATP by the combined actions of pyruvate:ferredoxin oxido-reductase, hydrogenase, acetate:succinate CoA transferase and succinate thiokinase. Superoxide dismutase, malate dehydrogenase (decarboxylating), ferredoxin, adenylate kinase and NADH:ferredoxin oxido-reductase are also localized in the hydrogenosome. It is nearly universally accepted that hydrogenosomes evolved from mitochondria--\cite{wiki:hydrogenosome}.}}

\newglossaryentry{gls:ketose}{
	name={ketose},
	description={A ketose is a monosaccharide containing one ketone group per molecule. The simplest ketose is dihydroxyacetone, which has only three carbon atoms, and it is the only one with no optical activity. All monosaccharide ketoses are reducing sugars, because they can tautomerize into \glspl{gls:aldose} via an aldol intermediate, and the resulting aldehyde group can be oxidised, for example in the Tollens' test or Benedict's test. Ketoses that are bound into glycosides, for example in the case of the fructose moiety of sucrose, are nonreducing sugars--\cite{wiki:ketose}}
}

\newglossaryentry{gls:kinetic:isotope:effect}{
	name={kinetic isotope effect},
	description={In physical organic chemistry, a kinetic isotope effect (KIE) is the change in the reaction rate of a chemical reaction when one of the atoms in the reactants is replaced by one of its isotopes. Formally, it is the ratio of rate constants for the reactions involving the light $k_L$ and the heavy $k_H$ (isotopically substituted reactants (isotopologues) $\frac{k_L}{k_H}$\cite{wiki:kinetic:isotope:effect}}}


\newglossaryentry{gls:kinetics}{
	name={kinetics},
	description={Concerning the rate of transitioning between states.
    Kinetic stability refers to being in a state which transitions slowly.}}

\newglossaryentry{gls:landauer_bound}{
	name={Landauer Bound},
	description={Landauer Bound: any logically irreversible manipulation of information, such as the erasure of a bit or the merging of two computation paths, must be accompanied by a corresponding entropy increase in non-information-bearing degrees of freedom of the information-processing apparatus or its environment--$kT(H_i - H_f)$, where $H_i$ represents the entropy of the initial state, $H_i$ of the final\cite{wiki:landauer}}}

\newglossaryentry{gls:life}{
	name={Life},
	description={Life is a self-propagating chemical system capable of undergoing adaptive evolution--adapted from \cite{deamer1994origins}}}

\newglossaryentry{gls:mitosome}{
	name={mitosome},
	description={A mitosome is an organelle found in some unicellular eukaryotic organisms. The mitosome has only recently been found and named,[1] and its function has not yet been well characterized. It was termed a crypton by one group, but that name is no longer in use--\cite{wiki:mitosome}}}


\newglossaryentry{gls:oligotroph}{
	name={oligotroph},
	description={An oligotroph is an organism that can live in an environment that offers very low levels of nutrients. They may be contrasted with copiotrophs, which prefer nutritionally rich environments. Oligotrophs are characterized by slow growth, low rates of metabolism, and generally low population density --\cite{wiki:oligotroph}}}

\newglossaryentry{gls:orthology}{
	name={orthology},
	description={Homologous sequences are orthologous if they are inferred to be descended from the same ancestral sequence separated by a speciation event: when a species diverges into two separate species, the copies of a single gene in the two resulting species are said to be orthologous. Orthologs, or orthologous genes, are genes in different species that originated by vertical descent from a single gene of the last common ancestor. The term "ortholog" was coined in 1970 by the molecular evolutionist Walter Fitch--cite{wiki:homology}e}}

\newglossaryentry{gls:outgroup}{
	name={outgroup},
	description={In cladistics or phylogenetics, an outgroup is a more distantly related group of organisms that serves as a reference group when determining the evolutionary relationships of the ingroup, the set of organisms under study, and is distinct from sociological outgroups. The outgroup is used as a point of comparison for the ingroup and specifically allows for the phylogeny to be rooted. Because the polarity (direction) of character change can be determined only on a rooted phylogeny, the choice of outgroup is essential for understanding the evolution of traits along a phylogeny\cite{wiki:outgroup}}}

\newglossaryentry{gls:phylogenetics}{
	name={phylogenetics},
	description={In biology, phylogenetics  is the study of the evolutionary history and relationships among individuals or groups of organisms (e.g. species, or populations). These relationships are discovered through phylogenetic inference methods that evaluate observed heritable traits, such as DNA sequences or morphology under a model of evolution of these traits. The result of these analyses is a phylogeny (also known as a phylogenetic tree) – a diagrammatic hypothesis about the history of the evolutionary relationships of a group of organisms. The tips of a phylogenetic tree can be living organisms or fossils, and represent the "end", or the present, in an evolutionary lineage. A phylogenetic tree can be rooted or unrooted. A rooted tree indicates the common ancestor, or ancestral lineage, of the tree. An unrooted tree makes no assumption about the ancestral line, and does not show the origin or root of the gene or organism in question. Phylogenetic analyses have become central to understanding biodiversity, evolution, ecology, and genomes\cite{wiki:phylogenetics}}}


\newglossaryentry{gls:poza}{
	name={poza},
	description={Earth bordered basins for water (commonly used for irrigation in the Andes)--\cite{denevan2003cultivated}}}

\newglossaryentry{gls:price:equation}{
	name={Price equation},
	description={In the theory of evolution and natural selection, the Price equation (also known as Price's equation or Price's theorem) describes how a trait or gene changes in frequency over time. The equation uses a covariance between a trait and fitness, to give a mathematical description of evolution and natural selection. It provides a way to understand the effects that gene transmission and natural selection have on the proportion of genes within each new generation of a population. The Price equation was derived by George R. Price, working in London to re-derive W.D. Hamilton's work on kin selection. Examples of the Price equation have been constructed for various evolutionary cases. The Price equation also has applications in economics--\cite{ wiki:price:equation}}}

\newglossaryentry{gls:pyran}{
	name={pyran},
	description={In chemistry, pyran, or oxine, is a six-membered heterocyclic, non-aromatic ring, consisting of five carbon atoms and one oxygen atom and containing two double bonds. The molecular formula is C5H6O. There are two isomers of pyran that differ by the location of the double bonds. In 2H-pyran, the saturated carbon is at position 2, whereas, in 4H-pyran, the saturated carbon is at position 4--\cite{wiki:pyran} }
}

\newglossaryentry{gls:r-K-selection}{
	name={r/K selection},
	description={In ecology, r/K selection theory relates to the selection of combinations of traits in an organism that trade off between quantity and quality of offspring. The focus on either an increased quantity of offspring at the expense of individual parental investment of r-strategists, or on a reduced quantity of offspring with a corresponding increased parental investment of K-strategists, varies widely, seemingly to promote success in particular environments--\cite{wiki:k-r-selection}}}


\newglossaryentry{gls:reaction-network}{
	name={reaction network},
	description={multiple chemical reactions that interact}}

\newglossaryentry{gls:ribozyme}{
	name={ribozyme},
	description={Ribozymes (ribonucleic acid enzymes) are RNA molecules that are capable of catalyzing specific biochemical reactions, similar to the action of protein enzymes. The 1982 discovery of ribozymes demonstrated that RNA can be both genetic material (like DNA) and a biological catalyst (like protein enzymes), and contributed to the RNA world hypothesis, which suggests that RNA may have been important in the evolution of prebiotic self-replicating systems. The most common activities of natural or in vitro-evolved ribozymes are the cleavage or ligation of RNA and DNA and peptide bond formation. Within the ribosome, ribozymes function as part of the large subunit ribosomal RNA to link amino acids during protein synthesis. They also participate in a variety of RNA processing reactions, including RNA splicing, viral replication, and transfer RNA biosynthesis. Examples of ribozymes include the hammerhead ribozyme, the VS ribozyme, Leadzyme and the hairpin ribozyme--\cite{wiki:ribozyme} }}

\newglossaryentry{gls:serpentinization}{
	name={serpentinization},
	description={A hydration and metamorphic transformation of ultramafic rock from the Earth's mantle--\cite{wiki:serpentinization}}
}

\newglossaryentry{gls:sympatric-speciation}{
	name={sympatric speciation},
	description={the evolution of a new species from a surviving ancestral species while both continue to inhabit the same geographic region--\cite{wiki:Sympatric:speciation}}}

\newglossaryentry{gls:thermodynamics}{
	name={thermodynamics},
	description={Concerning the flow of energy between states. Thermodynamic stability refers to being in a low energy state}}

\newglossaryentry{gls:zwitterion}{
	name={zwitterion},
	description={In chemistry, a zwitterion is a molecule with two or more functional groups, of which at least one has a positive and one has a negative electrical charge and the net charge of the entire molecule is zero--\cite{wiki:zwitterion}}}