% Glossary entries

\newacronym{gls:ATP}{ATP}{Adenosine triphosphate}

\newacronym{gls:DNA}{DNA}{Deoxyribonucleinc Acid}

\newacronym{gls:LAWKI}{LAWKI}{Life as we know it}

\newacronym{gls:LUCA}{LUCA}{Last Universal Common Ancestor}

\newacronym{gls:RNA}{RNA}{Ribonucleinc Acid}

\newacronym{gls:SFI}{SFI}{Santa Fe Institute}

\newacronym{gls:TOL}{TOL}{Tree of Life}

\newglossaryentry{gls:aldose} {
	name=aldose,
	description={An aldose is a monosaccharide (a simple sugar) with a carbon backbone chain with a carbonyl group on the endmost carbon atom, making it an aldehyde, and hydroxyl groups connected to all the other carbon atoms. Aldoses can be distinguished from \glspl{gls:ketose}, which have the carbonyl group away from the end of the molecule, and are therefore ketones--\cite{ wiki:aldose}}
}

\newglossaryentry{gls:amphiphile}{
	name={amphiphile},
	description={An amphiphile is a chemical compound possessing both hydrophilic (water-loving, polar) and lipophilic (fat-loving) properties. Such a compound is called amphiphilic or amphipathic--\cite{wiki:amphiphile}}}

\newglossaryentry{gls:clathrate}{
	name={clathrate},
	description={Inclusion compounds in which the guest molecule is in a cage
		formed by the host molecule or by a lattice of host molecules--\cite{book2014compendium}}}

\newglossaryentry{gls:deamination}{
	name={deamination},
	description={Deamination is the removal of an amino group from a molecule. Enzymes that catalyse this reaction are called deaminases--\cite{wiki:deamination}}}

\newglossaryentry{gls:DNAmethyltransferase}
	{name={DNAmethyltransferase},
	description={In biochemistry, the DNA methyltransferase (DNA MTase) family of enzymes catalyze the transfer of a methyl group to DNA. DNA methylation serves a wide variety of biological functions. All the known DNA methyltransferases use S-adenosyl methionine (SAM) as the methyl donor--\cite{wiki:DNAmethyltransferase}}}

\newglossaryentry{gls:enthalpy}{
	name={enthalpy},
	description={A property of a thermodynamic system, is equal to the system's internal energy plus the product of its pressure and volume. In a system enclosed so as to prevent matter transfer, for processes at constant pressure, the heat absorbed or released equals the change in enthalpy--\cite{ wiki:Enthalpy}}}

\newglossaryentry{gls:furan}{
	name={furan},
	description={Furan is a heterocyclic organic compound, consisting of a five-membered aromatic ring with four carbon atoms and one oxygen. Chemical compounds containing such rings are also referred to as furans--\cite{wiki:furan}}
}

\newglossaryentry{gls:gibbs-free}{
	name={Gibbs free energy},
	description={A thermodynamic potential that can be used to calculate the maximum of reversible work that may be performed by a thermodynamic system at a constant temperature and pressure (isothermal, isobaric). The Gibbs free energy is the maximum amount of non-expansion work that can be extracted from a thermodynamically closed system (one that can exchange heat and work with its surroundings, but not matter); this maximum can be attained only in a completely reversible process--\cite{wiki:gibbs-free}}}

\newglossaryentry{gls:ketose}{
	name={ketose},
	description={A ketose is a monosaccharide containing one ketone group per molecule. The simplest ketose is dihydroxyacetone, which has only three carbon atoms, and it is the only one with no optical activity. All monosaccharide ketoses are reducing sugars, because they can tautomerize into \glspl{gls:aldose} via an aldol intermediate, and the resulting aldehyde group can be oxidised, for example in the Tollens' test or Benedict's test. Ketoses that are bound into glycosides, for example in the case of the fructose moiety of sucrose, are nonreducing sugars--\cite{wiki:ketose}}
}

\newglossaryentry{gls:kinetics}{
	name={kinetics},
	description={Concerning the rate of transitioning between states.
    Kinetic stability refers to being in a state which transitions slowly.}}

\newglossaryentry{gls:life}{
	name={Life},
	description={Life is a self-propagating chemical system capable of undergoing adaptive evolution--adapted from \cite{deamer1994origins}}}

\newglossaryentry{gls:r-K-selection}{
	name={r/K selection},
	description={In ecology, r/K selection theory relates to the selection of combinations of traits in an organism that trade off between quantity and quality of offspring. The focus on either an increased quantity of offspring at the expense of individual parental investment of r-strategists, or on a reduced quantity of offspring with a corresponding increased parental investment of K-strategists, varies widely, seemingly to promote success in particular environments--\cite{wiki:k-r-selection}}}

\newglossaryentry{gls:oligotroph}{
	name={oligotroph},
	description={An oligotroph is an organism that can live in an environment that offers very low levels of nutrients. They may be contrasted with copiotrophs, which prefer nutritionally rich environments. Oligotrophs are characterized by slow growth, low rates of metabolism, and generally low population density --\cite{wiki:oligotroph}}}

\newglossaryentry{gls:poza}{
	name={poza},
	description={Earth bordered basins for water (commonly used for irrigation in the Andes)--\cite{denevan2003cultivated}}}

\newglossaryentry{gls:pyran}{
	name={pyran},
	description={In chemistry, pyran, or oxine, is a six-membered heterocyclic, non-aromatic ring, consisting of five carbon atoms and one oxygen atom and containing two double bonds. The molecular formula is C5H6O. There are two isomers of pyran that differ by the location of the double bonds. In 2H-pyran, the saturated carbon is at position 2, whereas, in 4H-pyran, the saturated carbon is at position 4--\cite{wiki:pyran} }
}

\newglossaryentry{gls:serpentinization}{
	name={serpentinization},
	description={A hydration and metamorphic transformation of ultramafic rock from the Earth's mantle--\cite{wiki:serpentinization}}
}

\newglossaryentry{gls:thermodynamics}{
	name={thermodynamics},
	description={Concerning the flow of energy between states. Thermodynamic stability refers to being in a low energy state}}

\newglossaryentry{gls:zwitterion}{
	name={zwitterion},
	description={In chemistry, a zwitterion is a molecule with two or more functional groups, of which at least one has a positive and one has a negative electrical charge and the net charge of the entire molecule is zero--\cite{wiki:zwitterion}}}