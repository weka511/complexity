% Glossary entries

\newacronym{gls:ATP}{ATP}{Adenosine triphosphate}

\newacronym{gls:DNA}{DNA}{Deoxyribonucleinc Acid}

\newacronym{gls:LAWKI}{LAWKI}{Life as we know it}

\newacronym{gls:LUCA}{LUCA}{Last Universal Common Ancestor}

\newacronym{gls:RNA}{RNA}{Ribonucleinc Acid}

\newacronym{gls:SFI}{SFI}{Santa Fe Institute}

\newacronym{gls:TOL}{TOL}{Tree of Life}

\newglossaryentry{gls:aldose} {
	name=aldose,
	description={An aldose is a monosaccharide (a simple sugar) with a carbon backbone chain with a carbonyl group on the endmost carbon atom, making it an aldehyde, and hydroxyl groups connected to all the other carbon atoms. Aldoses can be distinguished from \glspl{gls:ketose}, which have the carbonyl group away from the end of the molecule, and are therefore ketones--\cite{wiki:main-page}.}
}

\newglossaryentry{gls:amphiphile}{
	name={amphiphile},
	description={An amphiphile is a chemical compound possessing both hydrophilic (water-loving, polar) and lipophilic (fat-loving) properties. Such a compound is called amphiphilic or amphipathic--\cite{wiki:main-page}.}}

\newglossaryentry{gls:furan}{
	name={furan},
	description={Furan is a heterocyclic organic compound, consisting of a five-membered aromatic ring with four carbon atoms and one oxygen. Chemical compounds containing such rings are also referred to as furans--\cite{wiki:main-page}. }
}

\newglossaryentry{gls:ketose}{
	name={ketose},
	description={A ketose is a monosaccharide containing one ketone group per molecule. The simplest ketose is dihydroxyacetone, which has only three carbon atoms, and it is the only one with no optical activity. All monosaccharide ketoses are reducing sugars, because they can tautomerize into \glspl{gls:aldose} via an aldol intermediate, and the resulting aldehyde group can be oxidised, for example in the Tollens' test or Benedict's test. Ketoses that are bound into glycosides, for example in the case of the fructose moiety of sucrose, are nonreducing sugars--\cite{wiki:main-page}.}
}

\newglossaryentry{gls:life}{
	name={Life},
	description={Life is a self-propagating chemical system capable of undergoing adaptive evolution--adapted from \cite{deamer1994origins}}}

\newglossaryentry{gls:oligotroph}{
	name={oligotroph},
	description={An oligotroph is an organism that can live in an environment that offers very low levels of nutrients. They may be contrasted with copiotrophs, which prefer nutritionally rich environments. Oligotrophs are characterized by slow growth, low rates of metabolism, and generally low population density --\cite{wiki:main-page}.}}

\newglossaryentry{gls:poza}{
	name={poza},
	description={Earth bordered basins for water (commonly used for irrigation in the Andes).\cite{denevan2003cultivated}}}

\newglossaryentry{gls:pyran}{
	name={pyran},
	description={In chemistry, pyran, or oxine, is a six-membered heterocyclic, non-aromatic ring, consisting of five carbon atoms and one oxygen atom and containing two double bonds. The molecular formula is C5H6O. There are two isomers of pyran that differ by the location of the double bonds. In 2H-pyran, the saturated carbon is at position 2, whereas, in 4H-pyran, the saturated carbon is at position 4--\cite{wiki:main-page}. }
}

\newglossaryentry{gls:serpentinization}{
	name={serpentinization},
	description={A hydration and metamorphic transformation of ultramafic rock from the Earth's mantle--\cite{wiki:main-page}.}
}

\newglossaryentry{gls:zwitterion}{
	name={zwitterion},
	description={In chemistry, a zwitterion is a molecule with two or more functional groups, of which at least one has a positive and one has a negative electrical charge and the net charge of the entire molecule is zero--\cite{wiki:main-page}.}}