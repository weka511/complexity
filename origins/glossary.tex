\documentclass[]{article}
\usepackage{caption,subcaption,graphicx,float,url,amsmath,amssymb,tocloft}
\usepackage[hidelinks]{hyperref}
\usepackage[toc,acronym,nonumberlist]{glossaries}
\setacronymstyle{long-short}
\usepackage{glossaries-extra}
\graphicspath{{figs/}} 
\setlength{\cftsubsecindent}{0em}
\setlength{\cftsecnumwidth}{3em}
\setlength{\cftsubsecnumwidth}{3em}

%opening
\title{
	Notes from Origins of Life\\
	Glossary
}

\author{Simon Crase}

\makeglossaries

\loadglsentries{glossary-entries}

\begin{document}
	\maketitle

\begin{abstract}
	These is my glossary for the SFI Origins of Life Course. The course aims to push the field of Origins of Life research forward by bringing new and synthetic thinking to the question of how life emerged from an abiotic world.\\
	The content and images contained herein are the intellectual property of the Santa Fe Institute, with the exception of any errors in transcription, which are my own.
	These notes are distributed in the hope that they will be useful,
	but without any warranty, and without even the implied warranty of
	merchantability or fitness for a particular purpose. All feedback is welcome,
	but I don't necessarily undertake to do anything with it.
\end{abstract}
	
	% glossary
	\glsaddall
	\printglossaries
	
	% bibliography goes here
	
	\bibliographystyle{unsrt}
	\addcontentsline{toc}{section}{Bibliography}
	\bibliography{origins}
\end{document}