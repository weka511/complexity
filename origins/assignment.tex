\documentclass[]{article}
\usepackage{caption,subcaption,graphicx,float,url,amsmath,amssymb,tocloft}
\usepackage[hidelinks]{hyperref}
\usepackage[toc,acronym,nonumberlist]{glossaries}
\setacronymstyle{long-short}
\usepackage{glossaries-extra}
\graphicspath{{figs/}} 
\setlength{\cftsubsecindent}{0em}
\setlength{\cftsecnumwidth}{3em}
\setlength{\cftsubsecnumwidth}{3em}
\newcommand\numberthis{\addtocounter{equation}{1}\tag{\theequation}}

%opening
\title{
	\gls{gls:SFI} Origins of Life Course\\
	Peer Review Assignment}
\author{Lawki}

\makeglossaries
\loadglsentries{glossary-entries}

\begin{document}

\maketitle

\tableofcontents

\section{Energetics of metabolism}

\subsection{
	Why is methanogenesis more likely to be an early evolutionary metabolism, while aerobic oxidation was a later development?
}

\begin{enumerate}
	\item We know that oxygen levels were low until some time between 2.5Gya and 2.0Gya--as shown in Figure \ref{fig:nature13068-f1} \cite{lyons2014rise}.  It appears, however, that methanogenesis emerged as far back as 3.8Gya--Figure \ref{fig:nisbet2011evolution} \cite{nisbet2011evolution}. It is worth noting that methanogenesis is virtually confined to archaea\cite{angel2012methanogenic}, and aerobic oxidation is a specialization of eukaryotes. The question therefore boils down to whether the observed prior emergence of methanogenesis seems likely, or is just something than happened.
	\begin{figure}[H]
		\caption{Evolution of Earth’s atmospheric oxygen content through time after \cite{lyons2014rise} }\label{fig:nature13068-f1}
		\includegraphics[width=0.8\textwidth]{nature13068-f1}
	\end{figure}
	
	\begin{figure}[H]
		\caption{Geologic aeons after \cite{nisbet2011evolution} showing emergence of metahogenesis}\label{fig:nisbet2011evolution}
		\includegraphics[width=0.8\textwidth]{nisbet2011evolution}
	\end{figure}
	
	\item Aerobic oxidation is unlikely to have arisen prior to oxygen becoming available. Moreover it requires two inventions, photo synthesis and oxidation itself, whereas methanogenesis requires only one, and is therefore simpler, all things being equal.
	
	\item A hint may be found in \cite{nisbet2011evolution}.\begin{quote}
		Sedimentological evidence implies there were liquid oceans despite the faint young Sun. These oceans may have been sustained by the greenhouse warming effect of biologically-made methane. Oxygenesis in the late Archaean would have released free O2 into the water. This would have created oxic surface waters, challenging the methane greenhouse. After the Great Oxidation Event around 2.3 to 2.4 billion years ago, the atmosphere itself became oxic, perhaps triggering a glacial crisis by cutting methane-caused greenhouse warming.
	\end{quote}
	Perhaps an "oxygen-first world" might have been too cold to progress.
	\item The following suggests that methanogenesis may in some sense be easy, as different organisms have different pathways.
	\begin{quote}
		Although the biochemical machinery in methanogens varies with the pathway used, few functional genes, which encode for key enzymes in the production of methane, are common to all known methanogens--\cite{angel2012methanogenic}
	\end{quote}
	On the other hand, an metabolism that consumes oxygen depends on mitochondria.
\end{enumerate}


\subsection{
	If aerobic oxidation is more energetically favorable, why would methanogenesis still exist today?
}
There are still plenty of anaerobic environments. It seems eminently reasonable that organisms could successfully practice different lifestyles if there is not oxygen.
\begin{quote}
	Methanogens are strict anaerobes, and methanogenesis was shown to be fully suppressed upon exposure to oxygen in both pure culture and in soil--\cite{angel2012methanogenic}.
\end{quote}

\section{Life elsewhere}

You have been asked to participate on a mission to locate extraterrestrial life. What geochemical signatures would you want to look for (provide a minimum of 3 with justification) to help you determine if a planet is or has been inhabited in the past, based on the fossil record of the Earth and geochemistries elsewhere in the solar system?

\section{Phylogenic tree building}

\subsection{ Generate a phylogenic tree
	based on a single protein (or
	nucleotide) sequence}

\subsection{ Generate a phylogenic tree
	based on the known taxonomy}

\subsection{ Compare your protein and
	taxonomic trees. Do you notice
	any differences? (Include at
	least 1 difference, or state that
	they are identical)}

\subsection{ What are the challenges with
	building these trees?}

\section{Metabolic Flux}

\subsection{Make a 2D map of the metabolic rate per unit volume as
	a function of cell size on one axis and on the other
	axis.}

\subsection{Simulate the concentration field around a cell}
% end of text 

% glossary
\printglossaries

% bibliography go here

\bibliographystyle{unsrt}
\addcontentsline{toc}{section}{Bibliography}
\bibliography{origins,wikipedia}

\end{document}
