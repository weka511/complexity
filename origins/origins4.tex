\documentclass[]{article}
\usepackage{caption,subcaption,graphicx,float,url,amsmath,amssymb,tocloft}
\usepackage[hidelinks]{hyperref}
\usepackage[toc,acronym,nonumberlist]{glossaries}
\setacronymstyle{long-short}
\usepackage{glossaries-extra}
\graphicspath{{figs/}}
\setlength{\cftsubsecindent}{0em}
\setlength{\cftsecnumwidth}{3em}
\setlength{\cftsubsecnumwidth}{3em} 
%opening
\title{Notes from Origins of Life\\Week4: Early Life}
\author{Simon Crase}

\makeglossaries

\loadglsentries{glossary-entries}

\renewcommand{\thesection}{4.\arabic{section}}
\renewcommand{\glstextformat}[1]{\textbf{\em #1}}

\begin{document}

\maketitle

\begin{abstract}
   These are my notes from the $4^{th}$ week of the Santa Fe Institute Origins of Life Course\cite{sfi2019}. The course aims to push the field of Origins of Life research forward by bringing new and synthetic thinking to the question of how life emerged from an abiotic world.\\
   The content and images contained herein are the intellectual property of the Santa Fe Institute, with the exception of any errors in transcription, which are my own.
   These notes are distributed in the hope that they will be useful,
   but without any warranty, and without even the implied warranty of
   merchantability or fitness for a particular purpose. All feedback is welcome,
   but I don't necessarily undertake to do anything with it.
\end{abstract}

\setcounter{tocdepth}{2}
\tableofcontents
\listoffigures
\section{Introduction}

Lecturer: Chris Kempes


We'll discuss evolution in a pre-cellular world, chemical signatures of early life, protocells, and what the \gls{gls:LUCA} might have looked like.

\section{Protocells}

Lecturer: Sarah Mauer

We'll look at Protocells as a model for the origins of life. Aggregates are important for the origins of life--Figure \ref{fig:ImportanceOfAggregation}.

\begin{itemize}
	\item  Co-localize metabolic/genetic components
	\item   Define the individual to allow for 	selection
	\item   Direct involvement in metabolism
	\begin{itemize}
		\item 	Chemical gradients
		\item   Electron transfer reactions
		\item   Catalytic
	\end{itemize}
	\item   Protect metabolic/genetic 	components
\end{itemize}

\begin{figure}[H]
	\caption{Importance of aggregates to origins of life}\label{fig:ImportanceOfAggregation}
	\includegraphics[width=0.9\textwidth]{ImportanceOfAggregation}
\end{figure}

See  \cite{deamer2017role},\cite{maurer2011primitive}, \cite{segre2001lipid}

Self-assembled structures

Controlled by the hydrophobic effect (entropy) and non-bonding interactions (e.g., hydrogen bonds)

\begin{table}[H]
	\caption{Aggregates that are used to model the origins of life}
	\begin{tabular}{l|p{4cm}|p{3cm}} \hline
		Type & Definition &Biological example\\ \hline
		Vesicle (Liposome)& Bilayer-enclosed aqueous compartment& Cells, membrane bound organelles,\\ \hline
		Oil droplet&
		Nonpolar bulk phase often stabilized by
		amphiphiles&
		Lipoproteins (LDL)\\ \hline
		Coacervate&
		\glsdesc{gls:coacervate}&
		P-bodies, membrane-less organelles\\ \hline
		Inorganic&&Thin films or crystals\\ \hline
	\end{tabular}
\end{table}


\begin{figure}[H]
	\centering
	\caption{Self-assembled structures}
	\label{fig:self-assembled-structures}
	\begin{subfigure}[b]{0.3\textwidth}
		\centering
		\includegraphics[width=\textwidth]{SelfAssembled1}
		\caption{Water}
		\label{fig:water}
	\end{subfigure}
	\hfill
	\begin{subfigure}[b]{0.3\textwidth}
		\centering
		\includegraphics[width=\textwidth]{SelfAssembled2}
		\caption{Not water}
		\label{fig:not-water}
	\end{subfigure}
	\hfill
	\begin{subfigure}[b]{0.3\textwidth}
		\centering
		\includegraphics[width=\textwidth]{SelfAssembled3}
		\caption{Proton Gradient across thin inorganic barrier}
		\label{fig:proton-gradient}
	\end{subfigure}

\end{figure}

See \cite{sojo2016origin}

Factors that affect aggregation:
\begin{itemize}
	\item concentration;
	\item temperature;
	\item ionic strength;
	\item pH.
\end{itemize}

What chemistries do protocells harbour? Figure \ref{fig:ProtocellsAndReactions1} shows several possibilities:
\begin{itemize}
	\item A molecules interact with surface of aggregate through electrostatic interactions or hydrogen bonding;
	\item B hydrophobic effect sequestering non-polar molecule;
	\item C amphiphillic molecule anchored in membrane; 
	\item D molecule sequestered but still in water phase.
\end{itemize}

\begin{figure}[H]
	\caption{Aggregates interacting with molecules that are going to start life}\label{fig:ProtocellsAndReactions1}
	\includegraphics[width=0.9\textwidth]{ProtocellsAndReactions1}
\end{figure}

Figure \ref{fig:ProtocellsAndReactions2} depicts two locations for reactions.
\begin{itemize}
	\item A Surface associated reaction. Substrate becomes Product + Waste, and Waster drifts away. 
	\item B Catalytic network inside membrane, so we need to get rid of Waste.
\end{itemize}
\begin{figure}[H]
	\caption{Two locations for reactions}\label{fig:ProtocellsAndReactions2}
	\includegraphics[width=0.9\textwidth]{ProtocellsAndReactions2}
\end{figure}

Figure \ref{fig:ProtocellsAndReactions3} shows an example of a reaction:  the Hammerhead ribosome can self cleave at the red arrow. On the right we see that the reaction is not as fast when it is enclosed in a vesicle, because the waste products can't drain away.


\begin{figure}[H]
	\caption{Example of reaction: the Hammerhead ribosome}\label{fig:ProtocellsAndReactions3}
	\includegraphics[width=0.9\textwidth]{ProtocellsAndReactions3}
\end{figure}



See \cite{adamala2016programmable} and \cite{monnard2015current}.

See movie, where we see growing vesicle stealing material from neighbour. 
Figure \ref{fig:ProtocellGrowthDivision} illustrates Protocell Growth and division.  

\begin{figure}[H]
	\caption{Protocell Growth and division}\label{fig:ProtocellGrowthDivision}
	\includegraphics[width=0.9\textwidth]{ProtocellGrowthDivision}
\end{figure}

See  \cite{zhu2012photochemically} and   \cite{chen2004emergence}.

In Figure \ref{fig:TowardsLUCA} we see a prebiotic soup, aggregating, decreasing molecular diversity and increasing functional complexity, and moving towards first life. First life can undergo Darwinian evolution towards \gls{gls:LUCA}.
\begin{figure}[H]
	\caption{Towards LUCA}\label{fig:TowardsLUCA}
	\includegraphics[width=0.9\textwidth]{TowardsLUCA}
\end{figure}

\section{LUCA}

Lecturer: Kate Adamala

Kate discusses the Last Universal Common Ancestor of all Life.

The modern Tree of Life--Figure \ref{fig:TOL4}\cite{hug2016new}-- is very diverse in physiology, morphology, and life strategies. But all the diversity  comes from one single ancestor--Figure \ref{fig:TOL_root}.
\begin{figure}[H]
	\caption{The modern Tree of Life}\label{fig:TOL4}
	\includegraphics[width=\textwidth]{A_Novel_Representation_Of_The_Tree_Of_Life}
\end{figure}

\begin{figure}[H]
	\caption{All the diversity of Figure \ref{fig:TOL4} comes from one single ancestor.}\label{fig:TOL_root}
	\includegraphics[width=0.9\textwidth]{TOL_root}
\end{figure}

We know that all life came from one population of earliest cells; we know this because all modern life is built on the same principles--Figure \ref{fig:ModernCell}. At one stage, all of life went through the stage of a very simple cell, \gls{gls:LUCA}--Figure \ref{fig:EvolCell}.

\begin{figure}[H]
	\caption{All life came from one population of earliest cells}\label{fig:ModernCell}
	\includegraphics[width=0.9\textwidth]{ModernCell}
\end{figure}

\begin{figure}[H]
	\caption{Evolution of Life, showing very simple cell, LUCA}\label{fig:EvolCell}
	\includegraphics[width=0.9\textwidth]{EvolCell}
\end{figure}

\gls{gls:LUCA} was the ancestor to all modern cells, so it must have possessed the basic mechanisms of  modern cells.

\begin{figure}[H]
	\caption{LUCA must have possessed the basic mechanisms of  modern cells}\label{fig:LUCA_Attributes}
	\includegraphics[width=0.9\textwidth]{LUCA_Attributes}
\end{figure}

We know quite a lot about LUCA, but much is still unknown. We can deduce much by studying biochemical evolution and thr properties of modern cells to see what they share.

References:
\begin{itemize}
	\item \cite{penny1999nature}--cytology of PUCA;
	\item \cite{weiss2016physiology}--phylogenomic analysis to determine genome of LUCA;
	\item \cite{torino2013piecing}--reconstruct LUCA in ther lab.
\end{itemize}

\section{Chemical signatures for identifying life in the geological record}

Lecturer: Mayuko Nakagawa

Contents
\begin{itemize}
	\item About Biogeochemistry
	\item Fingerprints of life and environment
	\begin{itemize}
		\item Fossils
		\item Mineral compositions
		\item Isotopic signatures
	\end{itemize}
	\item How to use the signatures for identifying lives from 	Earth geological records?
\end{itemize}

Biogeochemistry--The study of:
\begin{itemize}
	\item How chemical elements flow through living systems and their physical environments--Figure \ref{fig:Biogeochemistry}\cite{linares-pasten_2018}.
	\item Investigate the factors that influence cycles of key 	elements such as bioelements (C, H, N, O, S...).
	
\end{itemize}

\begin{figure}[H]
	\caption{How chemical elements flow through living systems}\label{fig:Biogeochemistry}
	\includegraphics[width=0.9\textwidth]{Biogeochemistry}
\end{figure}

Fingerprints of Life
\begin{itemize}
	\item  DNA information cannot be preserved over geologic time
	scale (thousands ~ million years for eukaryotes' DNA)
	\item Chemical and morphological signatures are utilized
	\begin{itemize}
		\item  Fossils, molecular fossils
		\item Mineral compositions
		\item Isotopic signatures
	\end{itemize}
\end{itemize}

Isotopic signatures: Isotope variants of a particular chemical element
which differ in neutron number
\begin{itemize}
	\item  Stable isotope
	\begin{itemize}
		\item do not decay into other
		elements.
		\item Behavior is slightly different by the
		mass, useful for understanding
		material cycle.
	\end{itemize}
	\item Radioactive isotope: one having an unstable nucleus and
	which emits characteristic radiation during
	its decay to a stable form- e.g. $^3H$, $^14C$
\end{itemize}

\begin{figure}[H]
	\caption{Carbon Isotopes}\label{fig:CarbonIsotopes}
	\includegraphics[width=0.9\textwidth]{CarbonIsotopes}
\end{figure}

Isotopic Signatures

\begin{itemize}
	\item \textit{Kinetic isotopic effect}:
	Isotope ratio is changed by kinetic reactions 	(e.g.) the isotopic ratios are changed between substrates and products reflecting the metabolic processes. \begin{align*}
	^{12}CO_2 + H_2O \rightarrow&^{12}CH_2O + O_2 \text{, runs slightly faster than}\\
	^{13}CO_2 + H_2O \rightarrow&^{13}CH_2O + O_2 
	\end{align*}
	\item \textit{Equilibrium isotopic effect}:
	Isotope ratio is changed by equilibrium reactions.
	e.g.) Temperature effect, phase (gas, liquid, solid)
	$^{18}O$ ( $^{16}O$) is more (less) enriched in liquid than gas phase
\end{itemize}

\begin{figure}[H]
	\caption{Earth environment interacts with origin and evolution of life}\label{fig:Timeline}
	\includegraphics[width=0.9\textwidth]{Timeline}
\end{figure}

\begin{itemize}
	\item O2 level is important for evolution of life:
	\begin{itemize}
		\item the content of oxidized minerals and S isotope ratios are used for
		signatures of O2 level
	\end{itemize}
	\item  Small C isotope ratio ($\delta^{13}C$) of microfossils
	\begin{itemize}
		\item 	The difference of $\delta^{13}C$ values between carbonate and organic carbon ($<13\%$) indicated the possibility of Acetyl-CoA pathway
		and/or Calvin cycle product.
	\end{itemize}
\end{itemize}

Isotopic signature for Methanogens:
\begin{itemize}
	\item The sample rocks ($\approx 3.5Ga$); at the Dresser Formation at the North Pole area in Pilbara craton, Western Australia \cite{ueno2006evidence},
	\item Fluid inclusion; Tiny bubble of liquid or gas trapped inside a solid mineral-phase--Figure \ref{fig:FluidInclusionInTheRock}
	\item Measure C isotope ratio ( $\delta^{13}C$) of $CO_2$ and $CH_4$ in fluid inclusion
\end{itemize}

\begin{figure}[H]
	\caption{Fluid inclusion in the rock}\label{fig:FluidInclusionInTheRock}
	\includegraphics[width=0.9\textwidth]{FluidInclusionInTheRock}
\end{figure}

Isotopic signature for Methanogens. Figure \ref{fig:ueno_comparison} supports the idea that the methane is biogenic.
\begin{itemize}
	\item Methane production processes (Biogenic)
	\begin{itemize}
		\item 	Acetate Fermentation
		\item $CO_2$ reduction
	\end{itemize}
	\item Methane production processes (Abiotic)
	\begin{itemize}
		\item Thermogenic decomposition,
		\item Fischer-Tropsch reaction
	\end{itemize}
\end{itemize}

\begin{figure}[H]
	\caption{Comparison with present-day hydrothermal system}\label{fig:ueno_comparison}
	\includegraphics[width=0.9\textwidth]{ueno_comparison}
\end{figure}

Take Home Message: Signatures of life:
\begin{itemize}
	\item Chemical signatures, especially stable isotope information, are important tool for identifying the biogenic signatures preserved in geological records
	\item To understand how to record and decode the signatures, researches on modern earth material cycle and organisms are 	necessary.
\end{itemize}

Suggested Reading:

\cite{sharp2017principles} and \cite{allegre2008isotope}

References:
\cite{ueno2006evidence},
 \cite{bell2015potentially}, \cite{rosing199913c},  \cite{shen2001isotopic}, \cite{summons19992}, \cite{han1992megascopic}

\section{RNA}

\subsection{The RNA World}
\cite{robertson2012origins}, \cite{joyce2018protocells}, \cite{hud2018searching},  \cite{hoshika2019hachimoji}

\subsection{Molecular Evolution in the Lab}

\cite{joyce2007forty}, \cite{seelig2007selection}, \cite{chen2007ribozyme}, \cite{gold2012aptamers}, \cite{sefah2014vitro}, \cite{pinheiro2012synthetic}, \cite{mansy2007structure}, \cite{bartel1993isolation}, \cite{petrie2014limits}, \cite{pressman2019mapping}

\section{Autocatalysis}

\subsection{Autocatalytic Sets: A Cooperative Origin of Life}
\cite{wim2017origin}, \cite{hordijk2017chasing}, \cite{wim2019wandering}, \cite{patzke2007self}, \cite{vaidya2012spontaneous}, \cite{ashkenasy2004design}, \cite{hordijk2012structure}, \cite{sousa2015autocatalytic}

\subsection{Reaction Networks and Autocatalysis}

\section{Evolutionary Theory}

\subsection{An Introduction}

\subsection{A Recipe for Adaptation}

\subsection{Chance \& Change}

\section{Niche Construction}

% end of text 

% glossary
\printglossaries

% bibliography go here
 
\bibliographystyle{unsrt}
\addcontentsline{toc}{section}{Bibliography}
\bibliography{origins}

\end{document}
