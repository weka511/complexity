\documentclass[]{article}
\usepackage{caption,subcaption,graphicx,float,url,amsmath,amssymb,tocloft}
\usepackage[hidelinks]{hyperref}
\usepackage[toc,acronym,nonumberlist]{glossaries}
\setacronymstyle{long-short}
\usepackage{glossaries-extra}
\graphicspath{{figs/}} 
\setlength{\cftsubsecindent}{0em}
\setlength{\cftsecnumwidth}{3em}
\setlength{\cftsubsecnumwidth}{3em}

%opening
\title{
	Notes from Origins of Life\\
	Week 6: Astrobiology \& General Theories of Life
}
\author{Simon Crase}

\makeglossaries
\renewcommand{\thesection}{6.\arabic{section}}

\loadglsentries{glossary-entries}

\renewcommand{\glstextformat}[1]{\textbf{\em #1}}

\begin{document}

\maketitle

\begin{abstract}
   These are my notes from the $6^{th}$ Week of the Santa Fe Institute Origins of Life Course\cite{sfi2019}. The course aims to push the field of Origins of Life research forward by bringing new and synthetic thinking to the question of how life emerged from an abiotic world.\\
   The content and images contained herein are the intellectual property of the Santa Fe Institute, with the exception of any errors in transcription, which are my own.
   These notes are distributed in the hope that they will be useful,
   but without any warranty, and without even the implied warranty of
   merchantability or fitness for a particular purpose. All feedback is welcome,
   but I don't necessarily undertake to do anything with it.

\end{abstract}

\setcounter{tocdepth}{2}
\tableofcontents

\listoffigures

\section{Introduction}

\section{Origins of Life and Astrobiology}

\section{Exoplanets}

\subsection{The Habitable Zone}

\cite{fujii2018exoplanet}
\cite{villanueva2015unique}
\cite{kasting1993habitable}
\cite{kopparapu2013habitable}
\cite{nasa2019Explonet}

\subsection{Exoplanet Atmospheric Characterization}

\cite{sagan1993search}
\cite{kaltenegger2017characterize}
\cite{fujii2018exoplanet}
\cite{nasa2019Explonet}
\cite{robinson2011earth}
\cite{marois2010images}
\cite{greenbaum2018gpi}
\cite{deming2013infrared}
\cite{knutson2007map}

\section{What is Life?}

\subsection{Constraining the Definition of Life}

\cite{schrodinger1944life}

\subsection{Weird Life}

\cite{hollants2011life}

\cite{kim2001life}

\cite[Chapter 6: Why Water? Toward More Exotic Habitats ]{board2007limits}

\cite{cejkova2014dynamics}

\section{Abstract and general Models for Life}

\cite{trifonov2011vocabulary}

\cite{cronin2016beyond}

\section{The Multiple Origins of Life}

\subsection{The Argument}

\subsection{Reversing the Arrow of Time}

\subsection{The Theory of the Adaptive Arrow of Time}

\cite{rockmore2018cultural}

\subsection{Evolutionary Agents}

\section{Evolutionary Computation}

\cite{mitchell1998introduction}
\cite{eiben2003introduction}
\cite{holland1992adaptation}
\cite{forrest1993genetic}
\cite{ma2014novo}
\cite{marshall2014evolution}


\section{Scaling}

\cite{anderson2013altered}
\cite{damuth1981population}
\cite{enquist1998allometric}
\cite{enquist2012land}
\cite{marquet2005scaling}
\cite{schmidt1984scaling}
\cite{tucker2014evolutionary}
\cite{west1997general}

\section{Energy}

\cite{odum1976energy}
\cite{odum1983systems}
\cite{schmidt1997animal}
\cite{brown2004toward}
\cite{sibly2012metabolic}
\cite{ernest2003thermodynamic}
\cite{savage2004predominance}
\cite{dell2011systematic}
\cite{kempes2017thermodynamic}

% end of text 

% glossary
\printglossaries

% bibliography go here
 
\bibliographystyle{unsrt}
\addcontentsline{toc}{section}{Bibliography}
\bibliography{origins,wikipedia}

\end{document}
