\documentclass[]{article}
\usepackage{caption,subcaption,graphicx,float,url,amsmath,amssymb,tocloft}
\usepackage[hidelinks]{hyperref}
\usepackage[toc,acronym,nonumberlist]{glossaries}
\usepackage{titling}
\setacronymstyle{long-short}
\usepackage{glossaries-extra}
\graphicspath{{figs/}} 
\setlength{\cftsubsecindent}{0em}
\setlength{\cftsecnumwidth}{3em}
\setlength{\cftsubsecnumwidth}{3em}
\newcommand\numberthis{\addtocounter{equation}{1}\tag{\theequation}}
\pretitle{
	\begin{center}
		\includegraphics[width=6cm]{KanjiLife}\\	
	}
	\posttitle{\end{center}}

%opening
\title{
	Notes from Origins of Life\\
	Week 6\\
	 Astrobiology \& General Theories of Life
}
\author{Simon Crase (compiler)\\simon@greenweaves.nz}

\makeglossaries
\renewcommand{\thesection}{6.\arabic{section}}

\loadglsentries{glossary-entries}

\renewcommand{\glstextformat}[1]{\textbf{\em #1}}

\begin{document}

\maketitle

\begin{abstract}
   These are my notes from the $6^{th}$ Week of the Santa Fe Institute Origins of Life Course\cite{sfi2020}. 
   The content and images contained herein are the intellectual property of the Santa Fe Institute, with the exception of any errors in transcription, which are my own.
   These notes are distributed in the hope that they will be useful,
   but without any warranty, and without even the implied warranty of
   merchantability or fitness for a particular purpose. All feedback is welcome,
   but I don't necessarily undertake to do anything with it.

\end{abstract}

\setcounter{tocdepth}{2}
\tableofcontents

\listoffigures

\section[Introduction]{Introduction--Chris Kempes}

Over the course of these lectures we have provided a range of tools and perspectives that would help us understand the origin of life on our own planet. These can be used to develop a general theory of life. Our ultimate goal is to provide a general theory of Life, one capable of uncovering the history we know about, but also bounding the possibilities for other types of life, and helping us recognize other forms of life. In this unit we'll discuss the search for life beyond Earth and how origins of life fits into this effort. We'll also discuss general evolutionary processes and abstract life. 

\section[Origins of Life and Astrobiology]{Origins of Life and Astrobiology-- Sara Imari Walker}

Hi I'm Sarah Walker, a professor at Arizona State University and an astrobiologist.
I study the origin of life.

And so one of the questions you might have as an astrobiologist is: why is origin of life so critical to the study of astrobiology?

Well, astrobiologists are really interested
in whether or not we can identify a
living world so if we can identify
life on another planet.
We want to discover aliens and
ultimately, the question is whether we'll
be able to distinguish planets that have
life from plants that don't have life. So
in our own solar system, we can actually send robotic missions to other planets to look for life on the surface of those worlds but we're thinking about exoplanets and planets and distant solar systems all that we're going to get is a little bit of data about the entire planet.

As astrobiologists we are interested in thinking about life not only at the scale of individual organisms, but also at the scale of entire planets. 
And so just to talk about the magnitude of the problem a lot of people want to talk about looking at life in kind of new ways and maybe trying to use insights from different aspects of known fields of science to try to understand how we identify life.
Figure \ref{fig:Jupiter:Tellus} shows two worlds that are probably familiar to everyone, Jupiter and Earth; we know one of these planets is inhabited and the other isn't.
\begin{figure}[H]
	\caption[What makes worlds with life different?]{What makes worlds with life different? Not just non-equilibrium structures.}\label{fig:Jupiter:Tellus}
	\begin{subfigure}[b]{0.44\textwidth}
		\caption{Jupiter has its Great Red Spot}\label{fig:Jupiter}
		\includegraphics[width=\textwidth]{Jupiter}
	\end{subfigure}
	\;\;\;
	\begin{subfigure}[b]{0.5\textwidth}
		\caption{Earth at night, showing cities.}\label{fig:Earth}
		\includegraphics[width=\textwidth]{Tellus}
	\end{subfigure}
\end{figure}

The inhabited world is obviously our
very own earth, and we can see living
structures on its surface what we see
here is cities at night. When we're
thinking about how to think about the
problem of distinguishing the living
process on Earth from the nonliving
process on Jupiter, it's clear that both
have non-equilibrium structures on
their surfaces. So Jupiter for example
has this great red spot and as I
mentioned, earth has cities. So when we
want to talk about defining the
properties of those planets that are
associated with life, it's not just about
this disequilibra. Clearly cities are
fundamentally different than the great
red spot of Jupiter even though they're
both non-equilibrium structures. So we
have to move a little bit further and
understand the origins of the processes
that led to structure
on the surface of our planet that are
associated with life, and the root of
that question is really to understand
what's the probability of life emerging
on a planet--Figure \ref{fig:P:Life} and how can we actually
understand that as a planetary scale
process. 

\begin{figure}[H]
	\caption[Rate of abiogenesis in a prebiotic environment]{Rate of abiogenesis in a prebiotic environment as a function of its physical and chemical conditions}\label{fig:P:Life}
	\includegraphics[width=0.9\textwidth]{P_Life}
\end{figure}


There's really two ways of
constraining the likelihood of life
emerging on a planet. We don't
think that Jupiter is a living planet. Obviously
based on our observations of Jupiter, it
could be that it might satisfy some
definition of life down the road if we
actually come up with a theory for life
and Jupiter satisfies that theory but
right now we don't think Jupiter's alive.

And so we need some observations of
other living world to constrain the
probability of life right now earth is
the only example we know and so to do
that we actually have to detect alien
life and determine its abundance. And so
this is the way you usually people think
about astrobiology is actually looking
at other worlds trying to identify if
there's aliens on those worlds and then
maybe we would actually be able to
constrain the probability that a planet
like Earth is going to emerge life on
its surface and a planet like Jupiter is
not. 

But we can also think about theory
and experiment to constrain the
probability for life. And from this view
the idea is really to try to uncover
what are the universal principles of
life that might actually allow us to
build predictive models for the
circumstances under which life should
emerge. So, we would have some a priori
theory that would enable us to predict $P(life)$--
the probability of life emerging.
And  that theory should be
able to account for the differences in Figure \ref{fig:Jupiter:Tellus} that
it's not just a non-equilibrium process
on the surface of a planet but in the
case of formation of cities or forests
or any of the kind of rich structure
that we see on earth that's a product of
biology that the theory would be able to
explain what those things are and be
able to predict what kinds of other
examples of life we might be able to see
on other planets and their likelihood.

But really what we're talking about in
order to constrain the probability of
life is not just to think about the
probability of forests or cities on the
surface of planets as opposed to the
probability of great red spots or other
kinds of dissipative structures that
aren't alive. What we really want is to
understand what's the likelihood of life
even emerging on that planet.
So we really need to be able to solve
the origin of life problem in order to do
astrobiology effectively and constrain
the likelihood of life in the universe.

And so in order to do that, we have to
come up with better theories for origins
of life and be able to understand how
life emerges. And so one of the ways I
like to think about it is really that
we're looking for new principles that
would explain life not just on earth but
life on other planets and I really love
this quote from David Deutsch which I
think articulates very nicely the kind
of processes that are happening on
planets that we really need to be able
to understand in order to understand
life. 

\begin{quotation}
	Base metals can be transmuted into gold by stars, and by intelligent beings who understand the processes that power stars, and by nothing else in the universe--David Deutsch\cite{deutsch2011beginning}.
\end{quotation}

We have a physics that explains things like stars or the physics of Jupiter and why Jupiter has a great storm on the surface of the planet at the great red spot.
But we don't have a physics that explains the evolution of a planet like our own, how life emerges on that planet, or how it evolves over time to lead to the kind of diversity of structures that we have on the surface of our planet today, like cities and
thinking human beings.

\begin{figure}[H]
	\begin{center}
		\caption[We don't have a physics that explains the evolution of our planet]{We don't have a physics that explains the evolution of a planet like our own}\label{fig:tellus}
		\includegraphics[width=0.6\textwidth]{Tellus}
	\end{center}
\end{figure}

The origin of life problem is really a problem of how that entire process of life gets started in the first place and it's ultimately critically important to the field of astrobiology that we understand that process because we want to know on how many worlds that occurs.


\section{Exoplanets}

\subsection{The Habitable Zone}

Lecturer: Elizabeth Tasker

Figure \ref{fig:exoplants} shows the number of exoplanets known by discovery technique.\cite{nasa2019Explonet}.

\begin{figure}[H]
	\caption{Exoplanets by discovery technique}\label{fig:exoplants}
	\includegraphics[width=0.9\textwidth]{Exoplanets}
\end{figure}

There are two main techniques:
\begin{itemize}
	\item Radial velocity or Doppler wobble:Orbit with planet causes the
	star to wobble, creating a
	periodic shift in wavelength
	\item Transit: Dip in light as planet crosses
	our line of sight to the star
\end{itemize}

Typically this tells us up to two things about the planet, none of which measurable properties directly relates to surface conditions:
\begin{itemize}
	\item Radius of planet
	\item Mass of planet
\end{itemize}

Our next generation of instruments
aim at atmospheric composition

Rank by most interesting target for habitability

\begin{itemize}
	\item Easiest to recognise Earth-like life 	(water \& carbon-based chemistry)
	\item Needs to be detectable 	(surface water needed)
	\item How much insolation does an 	Earth-like planet need?
\end{itemize}

\begin{figure}[H]
	\caption{Classical Habitable Zone}\label{fig:classical:habitable:zone}
	\includegraphics[width=0.9\textwidth]{ClassicalHabitableZone}
\end{figure}

Figure \ref{fig:optimistic:habitable:zone} depicts the Optimistic Habitable Zone, based on empirical data that Venus \& Mars once had surface liquid water 1 - 3.8 Gyrs ago \cite{kasting1993habitable}.
\cite{kopparapu2013habitable}
\begin{figure}[H]
	\caption{Optimistic Habitable Zone }\label{fig:optimistic:habitable:zone}
	    \includegraphics[width=0.9\textwidth]{OptimisticHabitableZone.jpg}
\end{figure}

The classical habitable zone is only for an Earth-like planet. Different planets might have a habitable zone at a different location, or not at all.

Are the exoplanets in Figure \ref{fig:are:these:earthlike} Earth-like? We don’t know. Can only say: If we found another habitable Earth-like planet, it would be in the habitable zone.

\begin{figure}
	\caption{Are these exoplanets Earth-like?}\label{fig:are:these:earthlike}
	\includegraphics[width=0.9\textwidth]{AreTheseEarthlike}
\end{figure}

Conclusions

\begin{itemize}
	\item We’ve discovered thousands of exoplanets, many of which are similar in
	size to the Earth.
	\item But at the moment, we have no way of knowing what their surfaces are
	like (note that the Earth and Venus are both “Earth-sized planets”.)
	\item Our next generation of telescopes will be able to detect the atmosphere
	of these worlds and tell us something about their surfaces for the first
	time.
	\item The habitable zone is a useful concept for selecting planets for the 	new telescopes, but it offers no guarantee that a planet is habitable.
\end{itemize}

See also \cite{fujii2018exoplanet} and \cite{villanueva2015unique}.

\subsection{Exoplanet Atmospheric Characterization}

Lecturer: Yuka Fujii

Exoplanet Discovery usually come with Size (mass and/or radius) and Orbit--Figure \ref{fig:discovered:exoplanets}.

\begin{figure}[H]
	\caption{Discovered Exoplanets}\label{fig:discovered:exoplanets}
	\includegraphics[width=0.9\textwidth]{ExoplanetCharacteristics}
\end{figure} 

Figure \ref{fig:spectrum:earth:twin} depicts the spectrum that we'd expect from a twin planet for Earth. At shorter wavelengths the planet scatters light--Figure \ref{fig:spectrum:earth:twin1}--and the spectrum depends on the composition of the surface; at longer wavelengths it emits infrared--Figure \ref{fig:spectrum:earth:twin2}-- and the spectrum depends on the temperature structure. Figure \ref{fig:spectrum:earth:twin3} depicts the effect of absorption by atmospheric species. For an Earth twin we'd expect to find biologically important molecules. We can scan the plant's surface as it rotates.

\begin{figure}[H]
	\caption{Spectrum of an earth twin}
	\begin{subfigure}[b]{0.45\textwidth}
		\caption{Spectrum of an earth twin}\label{fig:spectrum:earth:twin}
		\includegraphics[width=\textwidth]{SpectrumEarthTwin}
	\end{subfigure}
	\begin{subfigure}[b]{0.45\textwidth}
		\caption{At shorter wavelengths the planet scatters light}\label{fig:spectrum:earth:twin1}
		\includegraphics[width=\textwidth]{SpectrumEarthTwin1}
	\end{subfigure}
	\begin{subfigure}[b]{0.45\textwidth}
		\caption{At longer wavelengths it emits infrared}\label{fig:spectrum:earth:twin2}
		\includegraphics[width=\textwidth]{SpectrumEarthTwin2}
	\end{subfigure}
	\begin{subfigure}[b]{0.45\textwidth}
		\caption{Absorption by atmospheric species}\label{fig:spectrum:earth:twin3}
		\includegraphics[width=\textwidth]{SpectrumEarthTwin3}
	\end{subfigure}
\end{figure}

Of course it is difficult to disentangle the light of the planet from the star--Figure \ref{fig:StarIsMuchBrighter}. We need to subtract the light of the star.

\begin{figure}[H]
	\caption[the star is many orders of magnitude brighter than its planets]{Unfortunately the star is many orders of magnitude brighter than its planets}\label{fig:StarIsMuchBrighter}
	\includegraphics[width=\textwidth]{StarIsMuchBrighter}
\end{figure}

In the past decade, Direct Imaging has been successful with young hot Jupiters in wide orbits--Figure \ref{fig:young:jupiter}. See  Figures  \ref{fig:young:jupiter1}\cite{marois2010images} and Figure \ref{fig:young:jupiter2}
\cite{greenbaum2018gpi}. These planets, however, are an order of magnitude brighter than earthlike planets.

\begin{figure}[H]
	\caption{Success with Young Jupiter-like Planets in Distant Orbits}\label{fig:young:jupiter}
	\begin{subfigure}[b]{0.45\textwidth}
		\caption{}\label{fig:young:jupiter1}
		\includegraphics[width=\textwidth]{DirectImaging1.jpg}
	\end{subfigure}
	\begin{subfigure}[b]{0.45\textwidth}
		\caption{}\label{fig:young:jupiter2}
		\includegraphics[width=\textwidth]{DirectImaging2.jpg}
	\end{subfigure}
\end{figure}

The discovery of transiting planets has opened up new prospects for studying exoplanet spectra--Figure \ref{fig:transiting:planets} without special instruments. A transiting planet is one that passes in front of the Star because its orbital plane is aligned with the telescope--Figure \ref{fig:transiting:planets2}. A small portion of the light is filtered through the planetary atmosphere--Figure \ref{fig:transiting:planets3}.
\begin{figure}[H]
	\caption{Transiting Planets}\label{fig:transiting:planets}
	\begin{subfigure}[t]{0.3\textwidth}
		\caption{}\label{fig:transiting:planets1}
		\includegraphics[width=\textwidth]{TransitingPlanets1}
	\end{subfigure}
	\begin{subfigure}[t]{0.3\textwidth}
		\caption{Passing in front}\label{fig:transiting:planets2}
		\includegraphics[width=\textwidth]{TransitingPlanets2}
	\end{subfigure}
	\begin{subfigure}[t]{0.3\textwidth}
		\caption{Transmission Spectroscopy}\label{fig:transiting:planets3}
		\includegraphics[width=\textwidth]{TransitingPlanets3}
	\end{subfigure}
\end{figure}
	

Summary: Key observations to characterize atmospheres (and surfaces) of exoplanets
\begin{itemize}
	\item Direct Imaging
	\item  Transmission spectroscopy
	\item  Secondary eclipse
	\item  Phase curves
\end{itemize}
Using these techniques, how would you find life on an Earth-twin?

\cite{sagan1993search}
\cite{kaltenegger2017characterize}
\cite{fujii2018exoplanet}

\cite{robinson2011earth}

\cite{deming2013infrared}
\cite{knutson2007map}



\section{Abstract and general Models for Life}

Lecturer: Sara Imari Walker

See discussion in \cite{trifonov2011vocabulary}. We want a general definition so we can recognize and explain life elsewhere in the Universe.

Theory vs. Model vs. Definition (ad hoc, based on Life On earth). We are limited by a single example of life--Figure \ref{fig:yatol}--descended from the \gls{gls:LUCA}. \gls{gls:LUCA} would have had DNA, translation, proteins, cellular organization: it doesn't take us back to the origin of life on Earth.

\begin{figure}[H]
	\caption{We are limited by a single example of life}\label{fig:yatol}
	\includegraphics[width=\textwidth]{YATOL}
\end{figure}

Genetics-First versus Metabolism-First--Figure \ref{fig:GeneticsVsMetabolism}.

\begin{itemize}
	\item Genetics-First, e.g. RNA World
	\item Metabolism-First, e.g. autocatalytic sets
\end{itemize}

\begin{figure}[H]
	\caption[Genetics-First versus Metabolism-First]{Genetics-First versus Metabolism-First: Two competing hypotheses}\label{fig:GeneticsVsMetabolism}
	\includegraphics[width=0.9\textwidth]{GeneticsVsMetabolism}
\end{figure}

Figure \ref{fig:GeneticsVsMetabolism} puts these two points of view side-by-side. We need Theories, not just Models: the Definition needs to come from Theory, not from a Model.

Life as an information processing system--\cite{nurse2008life}

''Focusing on information… may perhaps provide our best shot at uncovering universal laws of life that work not just for biological systems with known chemistry but also for putative artificial and alien life.''--\cite{cronin2016beyond}--Figure \ref{fig:LifeInformation}.

\begin{figure}[H]
	\caption{Life as Information}\label{fig:LifeInformation}
	\includegraphics[width=0.9\textwidth]{LifeInformation}
\end{figure}


\section{The Multiple Origins of Life}

Lecturer: David Krakauer

\subsection{The Argument}

\begin{itemize}
	\item Origin of Life Research is dominated by Naturalist-reductionists.
	\begin{itemize}
		\item My car and my adding machine 	understand nothing: they are
		not in that line of business--John Searle
	\end{itemize}
	\item an alternative would be Functionalists (like AI before Turing and modern ML)
	\begin{itemize}
		\item We are not interested in the fact that the brain has the consistency of cold porridge--Alan Turing.
	\end{itemize}
	\item Life emerges from an adaptive arrow of time (the reverse of the thermodynamic arrow).
	\item The Adaptive arrow of time is multi-scale and applies as readily to inference as organic evolution (this is not dependent on biological chemistry)
	\item The key to any form of adaptive evolution is the Agent, AKA, the Individual
	\item Individuals have evolved countless times in earth history - emerging from
	forms both ecological and individual: e.g. Virus evolution and the Block Chain.
\end{itemize}

\subsection{Reversing the Arrow of Time}

\begin{itemize}
	\item[Thermodynamic arrow] ''Let us draw an arrow arbitrarily. If as we follow the arrow we find more and more
	of the random element in the state of the world, then the arrow is pointing towards
	the future; if the random element decreases the arrow points towards the past … I
	shall use the phrase “time's arrow” to express this one-way property of time which
	has no analogue in space''--Arthur Eddington\cite{eddington1939philosophy}
	
	\item[Adaptive arrow]''It was Darwin’s chief contribution, not only to Biology but to the whole of natural science, to have brought to light a process by which contingencies a priori
	improbable are given, in the process of time, an increasing probability, until it is their non-occurrence, rather than their occurrence, which becomes highly improbable.''--Ronald Fisher\cite{fisher1930genetical}
\end{itemize}

\begin{figure}[H]
	\caption[Thermodynamic Arrow of Time versus Adaptive ]{Thermodynamic Arrow of Time wants to roll down hill, Adaptive up hill towards lower probability.}\label{fig:NaturalSelection}
	\includegraphics[width=0.9\textwidth]{NaturalSelection}
\end{figure}

\subsection{The Theory of the Adaptive Arrow of Time}

\begin{itemize}
	\item The Fundamental Theorem of Natural Selection
	R.A. Fisher, 1930
	\item ''adaptation is an optimization dynamics
	transferring information
	from the environment into the agent
	- reducing uncertainty about states of the world''\cite{rockmore2018cultural}
	\item A scale-invariant, substrate-neutral, stochastic process
	\begin{itemize}
		\item Evolution
		\item Inference
		\item Learning
	\end{itemize}
\end{itemize}

\subsection{Evolutionary Agents}

Sol Spiegelman wanted to know what was the minimal genome for \textit{Q Beta Phage}. He bred 74 generations in the presence of \textit{QBeta RNA replicase}, which it would normally have to synthesize--Figure \ref{fig:SpiegelmanMonster}. The RNA reduces from 4500 base pairs to 218!\cite{spiegelman1965synthesis}
\begin{figure}[H]
	\caption{The Spiegelman Monster}\label{fig:SpiegelmanMonster}
	\includegraphics[width=0.9\textwidth]{SpiegelmanMonster}
\end{figure}

Figure \ref{fig:SpiegelmanMonsterVenn} depicts the process: $v_i$ represents the information held by the virus only, $h_i$ the information held by the host (environment), and $s_i$ the information that is shared. The virus eliminates the shared information if the information is always there! If we don't know that the information is always there, we get autonomy.

\begin{figure}[H]
	\caption{Elimination of shared Information}\label{fig:SpiegelmanMonsterVenn}
	\includegraphics[width=0.9\textwidth]{SpiegelmanMonsterVenn}
\end{figure}

 Virus is normally considered as non-living, because it depends its environment, but we depend on our environment too, e.g. for vitamins. There is a spectrum of adaptive agency--Figure \ref{fig:SpectrumOfLife}.  ''Life is a mechanism for acquiring adaptive information about the World, that it propagates forward in time''. This includes computer viruses and blockchain!
 
\begin{figure}[H]
	\caption{The Spectrum of Life}\label{fig:SpectrumOfLife}
	\includegraphics[width=0.9\textwidth]{SpectrumOfLife}
\end{figure}
Open Questions

\begin{itemize}
	\item[Fundamentalists] Are we only interested in the initial necessary conditions for all subsequent life (e.g. Evolveability out of abiotic physics as the most fundamental basis for understanding).
	\item[Pluralists] Or are we interested in the multiple origins of life (adaptive agency) and thereby the many analogous processes that can support these (e.g. 	Coarse-Grained effective theories at multiple scales?)
\end{itemize}


\section{Evolutionary Computation}

Lecturer: Stephanie Forrest

Evolution as computation.

Basic Principles of Evolution
\begin{itemize}
	\item Random variation
	\item Selection
	\item Inheritance
\end{itemize}

What are the genetic representations?
\begin{itemize}
	\item  Discrete units (genes)
	\item Genotype vs. phenotype
	\item Information stored in a linear array
\end{itemize}

Figure \ref{fig:GeneticAlgorithm} shows a Genetic Algorithm, as introduced by John Holland\cite{holland1992adaptation}, and Figure \ref{fig:GAfitness} shows a typical performance curve.

\begin{figure}
	\caption{Genetic Algorithm, exhibiting Selection, Crossover, and Mutation}\label{fig:GeneticAlgorithm}
	\includegraphics[width=0.9\textwidth]{GeneticAlgorithm}
\end{figure}

\begin{figure}
	\caption{Example Fitness Curve, with two plateaux}\label{fig:GAfitness}
	\includegraphics[width=0.9\textwidth]{GAfitness}
\end{figure}

Applications
\begin{itemize}
	\item  Engineering
	\begin{itemize}
		\item 	Multi-parameter function optimization\cite{marshall2014evolution}
	\end{itemize}
\end{itemize}

Origins of Evolutionary Computing

\begin{itemize}
	\item John Holland \cite{holland1992adaptation}
	\item Ingo Rechenberg \cite{rechenberg1965cybernetic}
	\item David Fogel et al\cite{fogel1998artificial}
	\item John Koza--evolving computer programs\cite{koza1992genetic}
\end{itemize}

See also \cite{mitchell1998introduction}, \cite{eiben2003introduction}, \cite{forrest1993genetic}, and \cite{ma2014novo}.



\section{Scaling}

Lecturer: Pablo Marquet

Why is scaling important?

\begin{itemize}
	\item Provides a way to deal with the diversity of scales and
	organisms found in ecological systems
	\item Makes apparent the fundamental similarity that
	underlies diversity in nature and how this has been
	molded by evolution
	\item Provides a benchmark against which species,
	populations and ecosystems can be compared
\end{itemize}

Many ecological attributes scale with size--(\ref{eq:PowerLawScaling}) and Figure \ref{fig:PowerLawScaling}.
\begin{align*}
	y \propto& M^{\alpha}\text{, where $M$ represents mass, size, etc.}\numberthis\label{eq:PowerLawScaling}\\
	=& c M^{\alpha}\\
	\implies&\\
	\log(y)=&\log(c) + \alpha \log(M)
\end{align*}

\begin{figure}[H]
	\caption{Many ecological attributes scale with size}\label{fig:PowerLawScaling}
	\includegraphics[width=0.9\textwidth]{PowerLawScaling}
\end{figure}

Figure \ref{fig:ScalingExamples} shows some examples, after \cite{sibly2012metabolic}. Mammals are in grey, marsupials in red. We can use this as a basis for investigating why mammals deviate.
\begin{figure}[H]
	\caption[Scaling of life-history events]{Scaling of life-history events: mammals are in grey, marsupials in red.}\label{fig:ScalingExamples}
	\includegraphics[width=0.9\textwidth]{ScalingExamples}
\end{figure}


Figure \ref{fig:Scaling:individuals:ecosystems}	depicts Mean Carnivore Mass-- Figure \ref{fig:mcm} \cite{tucker2014evolutionary}; Carbon Turnover--Figure \ref {fig:ct} \cite{anderson2013altered}; and Primary Production--Figure \ref{fig:npp} \cite{enquist2012land}.

\begin{figure}[H]
	\caption{Scaling in individuals and	ecosystems}\label{fig:Scaling:individuals:ecosystems}
	\begin{subfigure}[b]{0.45\textwidth}
		\caption{Mean Carnivore Mass}\label{fig:mcm}
		\includegraphics[width=\textwidth]{Scaling1}
	\end{subfigure}
	\begin{subfigure}[b]{0.45\textwidth}
		\caption{Carbon Turnover}\label{fig:ct}
		\includegraphics[width=\textwidth]{Scaling2}
	\end{subfigure}
	\begin{subfigure}[b]{0.45\textwidth}
		\caption{Primary Production}\label{fig:npp}
		\includegraphics[width=\textwidth]{Scaling3}
	\end{subfigure}
\end{figure}

The central role of energy: the size ($M$) of an individual affects the amount
of energy it requires to maintain itself (basal metabolism $B$)--Figure \ref{fig:Kleiber} \cite{schmidt1984scaling}.

\begin{figure}[H]
	\caption{Kleiber's Law: $B \propto M^\frac{3}{4}$}\label{fig:Kleiber}
	\includegraphics[width=0.9\textwidth]{Kleiber}
\end{figure}

Figure \ref{fig:Kleiber} suggest that somehow an elephant is just a large mouse! Natural Selection is not acting randomly, but is subject to constraints. The West, Brown \& Enquist developed a general model for the origin of allometric scaling laws in biology\cite{west1997general}.
\begin{itemize}
	\item The properties of resource delivery networks determine
	the properties of whole-organism metabolic rate.
	\item  Biological systems have evolved under natural
	selection to optimize performance (delivery networks
	minimize energy loss)
\end{itemize}

Some implications 

\begin{itemize}
	\item What is the maximum number of individuals that can be found in a given area--(\ref{eq:max}) and Figure \ref{fig:MammalianHerbivores} \cite{damuth1981population}?
	\begin{align*}
	R=& \text{Energy or resources per unit area}\\
	B=&\text{Individual resource requirements}\\
	N_{max}\propto& R M^{-\frac{3}{4}}\label{eq:max}\numberthis
	\end{align*}
	\begin{figure}[H]
		\caption{Mammalian Herbivores--$N_{max}\propto R M^{-\frac{3}{4}}$}\label{fig:MammalianHerbivores}
		\includegraphics[width=0.9\textwidth]{MammalianHerbivores}
	\end{figure}	
	\item Populations of different organisms tend to use the same amount of energy--(\ref{eq:const}), Figure \ref{fig:PopulationEnergyUse} \cite{enquist1998allometric} 
	\begin{align*}
	E_{pop}\propto&N B\text{, Energy use for population}\\
	\propto& M^{-\frac{3}{4}} M^{\frac{3}{4}}\\
	\propto& 1\label{eq:const}\numberthis
	\end{align*}
	\begin{figure}[H]
		\caption{Population Energy Use}\label{fig:PopulationEnergyUse}
		\includegraphics[width=0.9\textwidth]{PopulationEnergyUse}
	\end{figure}
	\item Humans--the hyper-dense species--Figure \ref{fig:Hyperdense}
	\begin{figure}[H]
		\caption{Humans--the hyper-dense species}\label{fig:Hyperdense}
		\includegraphics[width=0.9\textwidth]{Hyperdense}
	\end{figure}
\end{itemize}

See also \cite{marquet2005scaling}

\section{Energy}

Lecturer: Van Savage

How is energy used in biology?

\begin{itemize}
	\item Used for infinitely many things
	\item Art of science is drawing boxes in useful ways
	\item What main boxes of energy do we need to consider for evolution and	ecological systems?
	\begin{itemize}
		\item Development/Growth
		\item Maintenance
		\item Reproduction
	\end{itemize}
	\item How do organisms obtain and produce energy?
	\begin{itemize}
		\item Obtaining resources
		\item Processing energy
		\item Distributing energy
		\item Converting energy (mitochondria)
	\end{itemize}
\end{itemize}

Metabolic rate depends on optimized networks and body size--Figure \ref {fig:MammalianBasalMetabolicRate} \cite{savage2004predominance}.

\begin{figure}[H]
	\caption{Metabolic rate depends on optimized networks and body size}\label{fig:MammalianBasalMetabolicRate}
	\begin{subfigure}[p]{0.45\textwidth}
		\caption{Mammalian Basal Metabolic Rate}
		\includegraphics[width=0.9\textwidth]{MammalianBasalMetabolicRate}
	\end{subfigure}
	\begin{subfigure}[p]{0.45\textwidth}
		\caption{Whole Plant Xylum Flux}
		\includegraphics[width=0.9\textwidth]{WholePlantXylumFlux}
	\end{subfigure}
\end{figure}

Metabolic rate depends on body temperature--Figure \ref{fig:BodyTemperature} \cite{dell2011systematic}.

\begin{figure}[H]
	\caption{Metabolic rate depends on body temperature}\label{fig:BodyTemperature}
	\includegraphics[width=0.9\textwidth]{BodyTemperature}
\end{figure}

Affects all physiology--Figure \ref{fig:MammalianRestingHeartRate}\cite{savage2004predominance}.

\begin{figure}[H]
	\caption{Mammalian Resting Heart Rate}\label{fig:MammalianRestingHeartRate}
	\includegraphics[width=0.9\textwidth]{MammalianRestingHeartRate}
\end{figure}

And Ecology--Figure \ref{eq:AndEcology} \cite{ernest2003thermodynamic}.

\begin{figure}[H]
	\caption{And Ecology}\label{eq:AndEcology}
	\begin{subfigure}[b]{0.5\textwidth}
		\caption{Temperature Corrected Individual Production}
		\includegraphics[width=\textwidth]{AndEcology1}
	\end{subfigure}
	\begin{subfigure}[b]{0.5\textwidth}
		\caption{Temperature Corrected Population Density}
		\includegraphics[width=\textwidth]{AndEcology2}
	\end{subfigure}
\end{figure}
See also \cite{odum1976energy}, \cite{odum1983systems}, \cite{schmidt1997animal}, \cite{brown2004toward}, and \cite{kempes2017thermodynamic}.

\section{Nonequilibrium Physics}

Lecturer: Eric Smith, External Professor, Santa Fe Institute.\\
\\
Topics covered:
\begin{itemize}
	\item The equilibrium concept of phase transition
	
	\item How phase transitions explain robust patterns
	
	\item Why equilibrium isn’t enough to understand life
	\item Phase transitions in dynamical systems ''frozen motion''
\end{itemize}

The ''ordinary'' response of thermodynamic systems to controls. E.g. lava: Viscosity increases smoothly as temperature is lowered. Phase transitions are different
\begin{itemize}
	\item Water does not become harder as it is cooled
	\item It turns to ice suddenly (critical temperature)
	\item The average direction of pointing of the ice sharply increases from zero at the freezing temperature
	\item The direction and strength of the crystal is called the ''Order 	Parameter'' of the transition
	\item Change is sudden because ''you can’t have half a symmetry''
	\begin{itemize}
		\item A direction either exists or it doesn’t
		\item All frozen directions are equivalent
	\end{itemize}
	\item Phase transitions, cooperatively-maintained states, and robustness
	\begin{itemize}
		\item Diamonds are hard because many atoms lock each other in place
		\item The order of the crystal is a ''robust'' property of freezing
	\end{itemize}
\end{itemize}

Evolution happens on a background of robust architectures
\begin{itemize}
	\item Universal small metabolites
	\item RNA and proteins
	\item Cellular and genomic individuality
\end{itemize}

Equilibrium ideas are not enough to explain the robust order of life (Chicken vs. chicken soup)


The Miller-Urey synthesis of amino acids\cite{miller1959organic}

Life is made of interlocking structures and processes: can phase transition ideas
be applied to these?

What might be the order parameters of life?

\begin{itemize}
	\item They would be chemical and energetic
	\item They would involve interdependent	structure and process
\end{itemize}

The characteristic molecules
\begin{itemize}
	\item Unchanging universal roles 	for small metabolites
	\item Key macromolecules such as RNA
\end{itemize}

The great biogeochemical cycles
\begin{itemize}
	\item Life alters cycling of Carbon, Nitrogen, Sulfur, and more
	\item New compounds are also formed of 	these elements
\end{itemize}

\begin{figure}[H]
	\caption[The great biogeochemical cycles]{The great biogeochemical cycles\cite{falkowski2008microbial}}\label{fig:biogeochemical} 
	\includegraphics[width=0.9\textwidth]{biogeochemical}
\end{figure}



\begin{figure}[H]
	\caption[Earth’s energy throughput]{Earth’s energy throughput: the Biosphere changes the way a planet converts sunlight into heat.\cite{meadows2005modelling}}\label{fig:EnergyThroughput} 
	\includegraphics[width=0.9\textwidth]{EnergyThroughput}
\end{figure}

The emergences of individualities
\begin{itemize}
	\item Individuality takes many 	forms
	\item The order parameters in individual-based systems 	are proper names
\end{itemize}

Take-home messages from the lecture:
\begin{itemize}
	\item Phase transitions are one way natural systems  spontaneously form order
	\item The order is robust due to mutual reinforcement
	\item Phase transitions can also lead to spontaneous order in processes like fractures
	\item Candidates for living order parameters include chemical cycles and individuality
\end{itemize}




% end of text 

% glossary
\printglossaries

% bibliography go here
 
\bibliographystyle{unsrt}
\addcontentsline{toc}{section}{Bibliography}
\bibliography{origins,wikipedia}

\end{document}
