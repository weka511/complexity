\documentclass[]{article}
\usepackage{caption,subcaption,graphicx,float,url,amsmath,amssymb}
\usepackage[hidelinks]{hyperref}
\usepackage[toc,acronym]{glossaries}
\setacronymstyle{long-short}
\usepackage{glossaries-extra}
\graphicspath{{figs/}} 
%opening
\title{
	Notes from Origins of Life\\
	Week 2: Chemical Commonalities
}
\author{Simon Crase}

\makeglossaries

\loadglsentries{glossary}

\renewcommand{\glstextformat}[1]{\textbf{\em #1}}

\begin{document}

\maketitle

\begin{abstract}
    These are my notes from the \gls{gls:SFI} Origins of \gls{gls:life} Course\cite{sfi2019}. The course aims to push the field of Origins of Life research forward by bringing new and synthetic thinking to the question of how life emerged from an abiotic world.

\end{abstract}

\setcounter{tocdepth}{2}
\tableofcontents


\section{Unit 3 Introduction}

\section{DNA as Information}

\cite{kunkel2004dna}

\section{Water as a Driving Force for Organization}

\cite{ball2017water}

\section{Kinetic vs. Thermodynamics – Assembly Constraints}

\cite{pross2017and},\cite{semenov2016autocatalytic},\cite{pross2008can}, \cite{dee2016comparing},\cite{pross2005emergence}

\section{Chemical Configurations: Proteins and DNA}

\section{Early Metabolisms}
\cite{bar2011survey}, \cite{fuchs2011alternative}, \cite{weiss2016physiology}

\section{Energy Harvesting}

\cite{simon2008organisation}

\section{Systematics and Limits of Metabolic Rates}

% end of text 

% glossary
\printglossaries

% bibliography goes here
 
\bibliographystyle{unsrt}
\addcontentsline{toc}{section}{Bibliography}
\bibliography{origins}

\end{document}
