\documentclass[]{article}
\usepackage{caption,subcaption,graphicx,float,url,amsmath,amssymb}
\usepackage[hidelinks]{hyperref}
\usepackage[toc,acronym]{glossaries}
\setacronymstyle{long-short}
\usepackage{glossaries-extra}
\graphicspath{{figs/}} 
%opening
\title{Notes from Origins of \gls{gls:life}\\Chemical Commonalities}
\author{Simon Crase}

\makeglossaries

\loadglsentries{glossary}

\renewcommand{\glstextformat}[1]{\textbf{\em #1}}

\begin{document}

\maketitle

\begin{abstract}
    These are my notes from the \gls{gls:SFI} Origins of Life Course\cite{sfi2019}. The course aims to push the field of Origins of Life research forward by bringing new and synthetic thinking to the question of how life emerged from an abiotic world.

\end{abstract}

\setcounter{tocdepth}{2}
\tableofcontents






% end of text 

% glossary
\printglossaries

% bibliography goes here
 
\bibliographystyle{unsrt}
\addcontentsline{toc}{section}{Bibliography}
\bibliography{origins}

\end{document}
