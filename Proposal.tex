\documentclass[]{article}
\usepackage{float,url,multirow}
\usepackage[nottoc,numbib]{tocbibind}

%opening
\title{Proposal}
\author{Simon Crase}

\begin{document}

\maketitle

\begin{abstract}

\end{abstract}


 
\section{What part of phenomenon would you like to model?}

\section{What are the principal types of agents involved in this phenomenon?}

\begin{tabular}{|l|l|} \hline
	Agent&Goal\\ \hline
	Investor & these are the Agents in the Problem Statement\\ \hline
	Pools & can be patches\\ \hline
	Predictors &these function closures, not Turtles \\ \hline
\end{tabular}

\section{What properties do these agents have?}


\begin{center}
	\begin{tabular}{ |l|l|l| } 
		\hline
		Agent & Property & Remarks \\
		\hline
		\multirow{5}{4em}{Investor} & wealth & accumulated payout, allowing for tau \\ 
		& favourite-predictor & tell Investor what course to follow \\ 
		& alternative-predictor & switch if favourite not doing well \\ 
		& payoffs & list of payouts, most recent first, before tau subtracted \\ 
		& choices & list of choices made by turtle, most recent first \\ 
		\hline
		\multirow{3}{4em}{Pool} & Total & cell3 \\ 
		& probability & cell6 \\ 
		\hline
		\multirow{3}{4em}{Predictor} & action & cell3 \\ 
		& history & cell6 \\ 
		& parameters & cell9 \\ 
		\hline
	\end{tabular}
\end{center}



\section{What actions (or behaviors) can these agents take?}

\begin{tabular}{|l|l|} \hline
	Agent&Goal\\ \hline
	Investor & Accumulate wealth\\ \hline
	Pools & \\ \hline
	Predictors & \\ \hline
\end{tabular}
\section{If the agents have goals, what are their goals?}

\begin{tabular}{|l|l|} \hline
	Agent&Goal\\ \hline
	Investor & Accumulate wealth\\ \hline
	Pools & \\ \hline
	Predictors & \\ \hline
\end{tabular}


\section{In what kind of environment do these agents operate?}
\section{How do these agents interact with this environment?}

\begin{thebibliography}{9}
\bibitem{ElFarol}
Inductive Reasoning and Bounded Rationality,
W. Brian Arthur,
The American Economic Review, 
Vol. 84, No. 2, Papers and Proceedings of the
Hundred and Sixth Annual Meeting of the American Economic Association (May, 1994), pp. 406-411
Published by: American Economic Association
\url{https://ocw.tudelft.nl/wp-content/uploads/ElFarolArtur1994.pdf}
\bibitem{PAISE2009}
The Kolkata Paise Restaurant Problem and Resource Utilization,
Anindya-Sundar Chakrabarti, Bikas K. Chakrabarti, Arnab Chatterjee,and Manipushpak Mitra,
New Journal of Physics. 12: 075033. 
\url{https://arxiv.org/pdf/0711.1639.pdf}
\bibitem{PAISE2007}
Kolkata Restaurant Problem as a generalised El Farol Bar Problem,
Bikas K. Chakrabarti,
in Econophysics of Markets and Business Networks, pages 239-246, Eds. A. Chatterjee and B. K. Chakrabarti, New Economic Windows Series, Springer,Milan (2007);
\url{https://arxiv.org/abs/0705.2098}
\end{thebibliography}
\end{document}
