\documentclass[]{article}
\usepackage{float,url,multirow}
\usepackage[nottoc,numbib]{tocbibind}

%opening
\title{Extinctions}
\author{Simon Crase}

\begin{document}

\maketitle

\begin{abstract}
I found that \cite{Wilensky:1999} exhibits behaviour that I did not expect: it appears stable, but when I increased the speed at which grass regrows slightly, the model run for some 1,200,000 generations, then the wolves become extinct. I started wondering whther there is some link between diversity and stability.
\end{abstract}


 
\section{What part of phenomenon would you like to model?}


\section{What are the principal types of agents involved in this phenomenon?}



\section{What properties do these agents have?}





\section{What actions (or behaviours) can these agents take?}



\section{If the agents have goals, what are their goals?}


\section{Agents operate in what kind of environment?}
\section{How do agents interact with environment?}

\medskip

\bibliographystyle{unsrt}
\bibliography{../complexity}

\end{document}
