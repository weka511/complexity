\documentclass[]{article}
\usepackage{caption,subcaption,graphicx,float,url,amsmath,amssymb,tocloft,wasysym,amsthm,thmtools}
\usepackage[hidelinks]{hyperref}
\usepackage[toc,acronym,nonumberlist]{glossaries}
\setacronymstyle{long-short}
\usepackage{glossaries-extra}
\graphicspath{{figs/}} 
\setlength{\cftsubsecindent}{0em}
\setlength{\cftsecnumwidth}{3em}
\setlength{\cftsubsecnumwidth}{3em}
\newcommand\numberthis{\addtocounter{equation}{1}\tag{\theequation}}
\newtheorem{thm}{Theorem}
\newtheorem{cor}[thm]{Corollary}
\setcounter{tocdepth}{1}
%opening
\title{
	Computation in Complex Systems\\
	Week 2\\
	 Algorithms \& Landscapes
}

%\makeglossaries


\begin{document}

\maketitle

\tableofcontents

\section{Maximum independent set}

\subsection{What is the maximum value of the independent set?}

\begin{figure}[H]
	\begin{center}
		\caption{What is the maximum value of the independent set?}
		\includegraphics[width=0.6\textwidth]{mwisQ}
	\end{center}
\end{figure}

Observations:
\begin{enumerate}
	\item A tree with $N$ nodes has $2^N-1$ non empty subsets, so a brute-force search is exponential. Let's not go there!
	\item Either the maximal independent set includes the root of the tree, or it doesn't.
	\item If the the maximal independent set doesn't include the root of the tree, each subtree must be maximal.
	\item If the maximal independent set includes the root of the tree, each subtree must have a set that excludes the node.
\end{enumerate}

So we can solve this problem as follows:
\begin{enumerate}
	\item Either the root node, $3$ is in $M$ or it isn't.
	\item If $3$ is in the tree, we have to exclude {4,1,5}, which gives a score of $3+(1+2)+2+(1+1)=9$
	\item If $3$ isn't in the tree, the maximum is the sum of the maxima for the 4 subtrees:
	\begin{enumerate}
		\item (4,(1,2)): since $1+2<4$, the maximum value is 4.
		\item (1,(2)): since $1<2$, the maximum value is 2. 
		\item ((5),(1,1)): since $1+1<5$, the maximum value is 5.
	\end{enumerate}
	\item Putting this together, if $3$ isn't in the tree, the maximum is $4+2+5=10$
\end{enumerate}

\subsection{Algorithm for maximum independent set}
Describe a dynamic programming algorithm that can find the maximum independent set from a tree - a rooted graph with no loops - of any size.

\section{Reductions \& Translations }

% glossary
%\printglossaries

% bibliography go here

%\bibliographystyle{unsrt}
%\addcontentsline{toc}{section}{Bibliography}
%\bibliography{origins,wikipedia}

\end{document}
