\documentclass[]{article}
\usepackage{caption,subcaption,graphicx,float,url,amsmath,amssymb,tocloft,wasysym,amsthm,thmtools,textcomp,listings,amsfonts,cancel}
\usepackage[hidelinks]{hyperref}
\usepackage[toc,acronym,nonumberlist]{glossaries}
\usepackage[]{algorithm2e}
\setacronymstyle{long-short}
\usepackage{glossaries-extra}
\graphicspath{{figs/}} 
\setlength{\cftsubsecindent}{0em}
\setlength{\cftsecnumwidth}{3em}
\setlength{\cftsubsecnumwidth}{3em}
\newcommand\numberthis{\addtocounter{equation}{1}\tag{\theequation}}
\newtheorem{thm}{Theorem}
\newtheorem{cor}[thm]{Corollary}
\setcounter{tocdepth}{1}

%opening
\title{Computation in Complex Systems\\
	Week 5
	}


\begin{document}

\maketitle

\section{Recursive Functions}

In Table \ref{table:recursive}, the first column contains either a reference to the exercise in the Exam, or a '-' for auxiliary functions:
\begin{itemize}
	\item \emph{bool} takes us into to the world of Boolean functions\footnote{\emph{mul} can serve as a replacement for \emph{and}; $bool(add)$ replaces \emph{or}};
	\item \emph{not} is our first Boolean function;
	\item \emph{gt} is used, alongside \emph{not}, to define \emph{max} \& \emph{min}.
\end{itemize}

\begin{table}[H]
	\caption{Recursive Functions}\label{table:recursive}
	\begin{tabular}{|c|c|l|l|}\hline
		&&Base case&General\\ \hline
		a&exp&$exp(x,0)=1$&$exp(x,y+1)=mult(exp(x,y),x)$\\ \hline
		b&pred&$pred(0)=0$&$pred(x+1) =x$\\ \hline
		c&sub&$sub(x, 0)=x $&$sub(x, y+1)=pred(sub(x,y)) $\\ \hline
		-&bool&$bool(0)=0$&$bool(x+1)=1$ \\ \hline
		-&not&$not(0)=1$&$not(x+1)=0$ \\ \hline
		-&gt&&$gt(x,y)=bool(sub(x,y))$\\ \hline
		d&min&-&min(x,y)=add(mult(gt(x,y),y),mult(not(gt(x,y)),x))\\ \hline
		e&max&-&max(x,y)=add(mult(gt(x,y),x),mult(not(gt(x,y)),y))\\ \hline
	\end{tabular}
\end{table}

\section{Turing Machines}

Table \ref{table:successor} shows the rules for a Turing machine for the successor function, and Tables  \ref{table:carry}--\ref{table:halt} show the rules through the three component Turing Machines from the exam. I have assumed that all Turing machines start in a special state \includegraphics[width=9pt]{gosign} and halt in state \includegraphics[width=9pt]{stop}: there are no rules that more the Turing machine into \includegraphics[width=9pt]{gosign} or out of \includegraphics[width=9pt]{stop}. Each table comprises two columns for the character under the head and the state, followed by three columns--the character to be written, the new state, and the movement of the head.
\begin{table}[H]
	\begin{center}
		\caption{Successor. There are two states, apart from the special states: in the \emph{Carrying} state we move to the left, setting 1s to 0s as we go, until we find a 0; in the \emph{Returning} state we }\label{table:successor}
		\begin{tabular}{|c|c||c|c|c|} \hline
			$\bullet$ &\includegraphics[width=9pt,height=7pt]{gosign}&$\bullet$ &Carrying&$\leftarrow$ \\ \hline
			1&Carrying&0&Carrying&$\leftarrow$ \\ \hline
			0&Carrying&1&Returning&$\rightarrow$ \\ \hline
			0&Returning&1&Returning&$\rightarrow$ \\ \hline
			1&Returning&1&Returning&$\rightarrow$ \\ \hline
			$\bullet$ &Returning&	$\bullet$&\includegraphics[width=9pt,height=7pt]{stop}& \\ \hline
		\end{tabular}
	\end{center}
\end{table}

The machine described in Table \ref{table:successor} is equivalent to the combination of the following 3 machines: $Halt(Return(Carry(x)))$.

\begin{table}[H]
	\begin{center}
		\caption{Carry: move left, replacing 1 by 0, until we find a  0.}\label{table:carry}
		\begin{tabular}{|c|c||c|c|c|} \hline
			$\bullet$ &\includegraphics[width=9pt,height=7pt]{gosign}&$\bullet$ &Carrying&$\leftarrow$ \\ \hline
			1&Carrying&0&Carrying&$\leftarrow$ \\ \hline
			0&Carrying&1&\includegraphics[width=9pt,height=7pt]{stop}&$\rightarrow$ \\ \hline
		\end{tabular}
	\end{center}
\end{table}

\begin{table}[H]
	\begin{center}
		\caption{Return: move the head back to the final position}\label{table:return}
		\begin{tabular}{|c|c||c|c|c|} \hline
			0&\includegraphics[width=9pt,height=7pt]{gosign}&1&Returning&$\rightarrow$ \\ \hline
			0&Returning&0&Returning&$\rightarrow$ \\ \hline
			1&Returning&1&Returning&$\rightarrow$ \\ \hline
			$\bullet$ &Returning&	$\bullet$&\includegraphics[width=9pt,height=7pt]{stop}& \\ \hline
		\end{tabular}
	\end{center}
\end{table}

\begin{table}[H]
	\begin{center}
		\caption{Halt}\label{table:halt}
		\begin{tabular}{|c|c||c|c|c|} \hline
			$\bullet$ &\includegraphics[width=9pt,height=7pt]{gosign}&$\bullet$ &\includegraphics[width=9pt,height=7pt]{stop}& \\ \hline
		\end{tabular}
	\end{center}
\end{table}

% bibliography go here

\bibliographystyle{unsrt}
\raggedright
\addcontentsline{toc}{section}{Bibliography}
\bibliography{computations}

\end{document}
