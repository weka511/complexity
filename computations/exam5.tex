\documentclass[]{article}
\usepackage{caption,subcaption,graphicx,float,url,amsmath,amssymb,tocloft,wasysym,amsthm,thmtools,textcomp,listings,amsfonts,cancel}
\usepackage[hidelinks]{hyperref}
\usepackage[toc,acronym,nonumberlist]{glossaries}
\usepackage[]{algorithm2e}
\setacronymstyle{long-short}
\usepackage{glossaries-extra}
\graphicspath{{figs/}} 
\setlength{\cftsubsecindent}{0em}
\setlength{\cftsecnumwidth}{3em}
\setlength{\cftsubsecnumwidth}{3em}
\newcommand\numberthis{\addtocounter{equation}{1}\tag{\theequation}}
\newtheorem{thm}{Theorem}
\newtheorem{cor}[thm]{Corollary}
\setcounter{tocdepth}{1}

%opening
\title{Computation in Complex Systems\\
	Week 5
	}


\begin{document}

\maketitle

\section{Recursive Functions}

\begin{table}[H]
	\begin{tabular}{|c|c|l|l|}\hline
		&Function&Base case&General\\ \hline
		a&exp&$exp(x,0)=1$&$exp(x,y+1)=mult(exp(x,y),x)$\\ \hline
		b&pred&$pred(0)=0$&$pred(x+1) =x$\\ \hline
		c&sub&$sub(x, 0) $&$sub(x, y+1)=pred(sub(x,y)) $\\ \hline
		d&min&$min(x,0)$&\\ \hline
		e&max&$max(x,0)$&\\ \hline
	\end{tabular}
\end{table}

\section{Turing Machines}

% bibliography go here

\bibliographystyle{unsrt}
\raggedright
\addcontentsline{toc}{section}{Bibliography}
\bibliography{computations}

\end{document}
