\documentclass[]{article}
\usepackage{caption,subcaption,graphicx,float,url,amsmath,amssymb,tocloft,wasysym,amsthm,thmtools,textcomp,listings,amsfonts,cancel}
\usepackage[hidelinks]{hyperref}
\usepackage[toc,acronym,nonumberlist]{glossaries}
\usepackage[]{algorithm2e}
\setacronymstyle{long-short}
\usepackage{glossaries-extra}
\graphicspath{{figs/}} 
\setlength{\cftsubsecindent}{0em}
\setlength{\cftsecnumwidth}{3em}
\setlength{\cftsubsecnumwidth}{3em}
\newcommand\numberthis{\addtocounter{equation}{1}\tag{\theequation}}
\newtheorem{thm}{Theorem}
\newtheorem{cor}[thm]{Corollary}
\setcounter{tocdepth}{1}

%opening
\title{Computation in Complex Systems\\
	Week 5
	}


\begin{document}

\maketitle

\section{Recursive Functions}

In Table \ref{fig:recursive}, the first column contains either a reference to the exercise in the Exam, or a '-' for auxiliary functions:
\begin{itemize}
	\item \emph{trunc} is our entre to the world of Boolean functions;
	\item \emph{not} is our first Boolean function;
	\item \emph{gt} is used, alongside \emph{not} to define \emph{max} \& \emph{min}.
\end{itemize}

\begin{table}[H]
	\caption{Recursive Functions}\label{fig:recursive}
	\begin{tabular}{|c|c|l|l|}\hline
		&&Base case&General\\ \hline
		a&exp&$exp(x,0)=1$&$exp(x,y+1)=mult(exp(x,y),x)$\\ \hline
		b&pred&$pred(0)=0$&$pred(x+1) =x$\\ \hline
		c&sub&$sub(x, 0)=x $&$sub(x, y+1)=pred(sub(x,y)) $\\ \hline
		-&trunc&$trunc(0)=0$&$trunc(x+1)=1$ \\ \hline
		-&not&$not(0)=1$&$not(x+1)=0$ \\ \hline
		-&gt&&$gt(x,y)=trunc(sub(x,y))$\\ \hline
		d&min&-&min(x,y)=add(mult(gt(x,y),x),mult(not(gt(x,y)),y))\\ \hline
		e&max&-&max(x,y)=add(mult(gt(x,y),y),mult(not(gt(x,y)),x))\\ \hline
	\end{tabular}
\end{table}

\section{Turing Machines}

% bibliography go here

\bibliographystyle{unsrt}
\raggedright
\addcontentsline{toc}{section}{Bibliography}
\bibliography{computations}

\end{document}
