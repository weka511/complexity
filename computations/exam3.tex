\documentclass[]{article}
\usepackage{caption,subcaption,graphicx,float,url,amsmath,amssymb,tocloft,wasysym,amsthm,thmtools,textcomp,listings,amsfonts,cancel}
\usepackage[hidelinks]{hyperref}
\usepackage[toc,acronym,nonumberlist]{glossaries}
\usepackage[]{algorithm2e}
\setacronymstyle{long-short}
\usepackage{glossaries-extra}
\graphicspath{{figs/}} 
\setlength{\cftsubsecindent}{0em}
\setlength{\cftsecnumwidth}{3em}
\setlength{\cftsubsecnumwidth}{3em}
\newcommand\numberthis{\addtocounter{equation}{1}\tag{\theequation}}
\newtheorem{thm}{Theorem}
\newtheorem{cor}[thm]{Corollary}
\setcounter{tocdepth}{1}

%opening
\title{Computation in Complex Systems\\
	Week 3\\
	P or NP? That is the Question.
	}


\begin{document}

\maketitle

\section{Circuits \& Formulae}

Table \ref{table:truth} is the truth table for the two formulae. I have used the symbol $\top$ to show that a formula evaluates to true, $\bot$ for false. The column headed ''AND'' denotes the truth value of $y_1 = x_1 \land x_2$, and that headed ''3-SAT'' denotes the truth value of $ (x_1\lor \bar{y_1}) \land (x_1\lor \bar{y_2}) \land (\bar{x_1} \lor \bar{x_2} \lor y_1)$. The demonstrandum asserts that these two columns are equal. Since the 3-clause is a conjunction, one 'F' in any of the 3 preceding columns is enough to falsify the clause.

\begin{table}[H]
	\caption{Truth Table--Circuits \& Formulae}\label{table:truth}
	\begin{center}
		\begin{tabular}{|c|c|c||c|c|c|c||c|}\hline
			$x_1$&$x_2$&$y_1$&AND&$x_1 \lor \bar{y_1}$&$x_2 \lor \bar{y_1}$&$\bar{x_1}\lor\bar{x_2}\lor y_1$&3-clause\\ \hline
			F&F&F&$\top$&T&T&T&$\top$\\ \hline
			F&F&T&$\bot$&F&F&T&$\bot$\\ \hline
			F&T&F&$\top$&T&T&T&$\top$\\ \hline
			F&T&T&$\bot$&F&&&$\bot$\\ \hline
			T&F&F&$\top$&T&T&T&$\top$\\ \hline
			T&F&T&$\bot$&T&F&&$\bot$\\ \hline
			T&T&F&$\bot$&T&T&F&$\bot$\\ \hline
			T&T&T&$\top$&T&T&T&$\top$\\ \hline
		\end{tabular}
	\end{center}
\end{table}

\begin{enumerate}
	\item Given than the three clauses are linked by logical AND statements, all three clauses must be true, or else the clause is false.
	
	\item If $x_1$ is false, $y_1$ must be false.
	
	\item If $x_2$ is false, $y_1$ must be false.
	
	\item If $x_1$ is true and $x_2$ is true, $y_1$ must be true.
\end{enumerate}

\section{Travelling Salesperson}

I shall use this criterion: \begin{quotation}
	A yes/no question is in NP if, whenever the answer is "yes", there is an easily checked proof or "witness" to that fact\cite[Lecture 3.5]{sfi2020computation}.
\end{quotation}

\begin{enumerate}
	\item Is $D<10,000$ (miles)?
	\begin{enumerate}
		\item Suppose someone presents we with a collection of edges, 
			$W$, say, which she asserts is a witness.
		\item I can easily check that it includes all node and returns to the origin, so it is a path.
		\item can easily verify that the length of $W$, $L(W)$, say, is less than 10,000. Now there are two possibilities:
		\begin{enumerate}
			\item W is the shortest path, so  $L(D)=L(W)<10000$, or
			\item W is not the shortest path, so $\exists D \mid L(D)<L(W)<10000$.
		\end{enumerate}
		Either way $L(D)<10000$.
		\item This problem is therefore in NP.
	\end{enumerate}
	
	\item Is D more than 8,000 miles?
		\begin{enumerate}
		\item The previous argument does not work in this case. If someone produces a path that is more 8,000 miles, I have no way of easily knowing whether or not there the minimum path is or less than this one. I can test all possible paths to see whether or not they are shorted than 8,000, and I can save a bit of time by rejecting a path as soon as I find a partial length over 8,000: at best this makes an exponential process a little shorter.
		\item IMHO this decision is not NP.  
	\end{enumerate}
	
	\item Is D exactly 9,219 miles? 
		\begin{enumerate}
		\item This is not NP.
		\item The argument is similar to the "more than 8,000" argument. Although I might get lucky and discredit the witness if I stumble across a path that is shorter than 9,219, this doesn't make the task of verifying a correct solution any easier. 
	\end{enumerate}
\end{enumerate}

\section{Complexity Hierarchy}

From \cite{sfi2020glossary}, a cellular automaton is: \begin{quotation}
	A mathematical or computational system in which simple elements (''cell'') are arrayed in a regular lattice. At a given time step, each cell is in some discrete state, and at each time step, each cell updates its state using a function of its current state and the states of its neighboring cells. To define a particular cellular automaton, one must specify the dimensionality of the lattice, the neighborhood of a cell, the set of possible states, and the state update function used by each cell. 
\end{quotation}
I am going to assume that I know the rule each cell uses to update its state, that there a $m$ cells, and the state of each cell is binary--0 or 1. If this is not true, if there the number of states is more than 2 but does not exceed $2^k$ for some $k$, I can replace the cellular automaton with another machine with $k\cdot m$ binary cells. 

\begin{thm}
	The problem " What will the state be at $t_{n+x}$?" is in P.
\end{thm}

\begin{proof}
	
	\begin{enumerate}
		\item I need to apply the rule $n$ times to $m$ cells, i.e. $m \cdot n$ times. \item This is linear in $n$ and $m$, i.e. polynomial, so the problem is in P. 
	\end{enumerate}
\end{proof}


\begin{thm}
	The problem " Does s have a predecessor?" is in NP.
\end{thm}

\begin{proof}
		\begin{enumerate}
		\item I can test for the existence of a predecessor by applying the rule to $2^m$ possible states and checking the result, which is exponential.
		\item OTOH if someone offers me a state which they claim to be a witness there being a predecessor, then I can test it in time $\propto m$, so the problem is in NP. 
	\end{enumerate}
\end{proof}

\begin{thm}
	The problem "On a lattice of size $n$, is $s$ on a periodic orbit?" is in PSPACE.
\end{thm}

\begin{proof}
			\begin{enumerate}
		\item On a lattice of size $n$ there are $2^n$ possible states. Therefore if I am presented with $s$ it could  $2^n$ steps.
		\item We are dealing with exponential time. 
	\end{enumerate}
\end{proof}

\begin{thm}
	The problem " On a lattice of infinite size, will $s$ ever die out?" is undecidable.
\end{thm}

\begin{proof}
		\begin{enumerate}
		\item ''Eternity is awful long time, especially towards the end.''-Woody Allen
		\item I claim that this is undecidable\label{sec:undecidable}.
		\item All we need to do is show that this is un-computable for a particular finite automaton and a particular $s$. We will chose Rule 110, since we know it is universal--\cite{cook2004universality}.
		\item  I assert that \ref{sec:undecidable} is equivalent to the halting problem. Since Rule 110 is universal, we can encode the halting problem for any Turing machine by a state, and ask whether it will ever die out. 
	\end{enumerate}
\end{proof}



% bibliography go here

\bibliographystyle{unsrt}
\raggedright
\addcontentsline{toc}{section}{Bibliography}
\bibliography{computations}

\end{document}
