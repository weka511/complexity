\documentclass[]{article}
\usepackage{caption,subcaption,graphicx,float,url,amsmath,amssymb,tocloft,wasysym,amsthm,thmtools}
\usepackage[hidelinks]{hyperref}
\usepackage[toc,acronym,nonumberlist]{glossaries}
\setacronymstyle{long-short}
\usepackage{glossaries-extra}
\graphicspath{{figs/}} 
\setlength{\cftsubsecindent}{0em}
\setlength{\cftsecnumwidth}{3em}
\setlength{\cftsubsecnumwidth}{3em}
\newcommand\numberthis{\addtocounter{equation}{1}\tag{\theequation}}
\newtheorem{thm}{Theorem}
\newtheorem{cor}[thm]{Corollary}

%opening
\title{
	Computation in Complex Systems\\
	Peer Review Assignment\\
	Week 1
}

%\makeglossaries


\begin{document}

\maketitle

\tableofcontents

\section{Polynomials \& Exponentials}

Today, your computer can do $T$ steps in a week. According to Moore's law, next year, your computer will be able to do $2T$ steps in a week. How does doubling T change the n that can be computed in a week? 

a) This year $T = n^2$. Next year my (new) computer can do $2T$ steps. Let $n^\prime$ denote the size of problem I can handle next year. Then we have:
\begin{align*}
	T =& n^2 \text{, and}\\
	2T =& (n^\prime)^2 \text{, from Moore's Law, whence}\\
	 (n^\prime)^2=& 2n^2 \text{, or, taking square roots of both sides}\\
	n^\prime=& \sqrt{2} n
\end{align*}
So $n$ changes by a factor of $\sqrt(2)$.

b) This year $T = 2^n$. Next year my (new) computer can do $2T$ steps. Once again let $n^\prime$ denote the size of problem I can handle next year. Then we have:

\begin{align*}
	T =&2^n \text{, and}\\
	2T =& 2^{n^\prime} \text{, from Moore's Law, whence}\\
	2^{n^\prime}=& 2\cdot 2^n\\
	=& 2^{n+1} \text{, so taking logarithms to base 2}\\
	n^\prime =& n+1
\end{align*}

\section{Divide \& Conquer}
Given:
\begin{align*}
	f(0)=&1 \numberthis \label{eq:hypothesis}\\
	f(n)=&2f(n-1)+1 \numberthis \label{eq:recurrence}
\end{align*}
I'll begin by computing the first few values:
\begin{table}[H]
	\begin{center}
		\begin{tabular}{|l|r|}
			n&$f(n)$\\
			1&1\\
			2&3\\
			3&7\\
			4&15\\
			5&32\\
			6&63
		\end{tabular}
	\end{center}
\end{table}
We already have enough values to formulate a hypothesis\footnote{this sounds better than ''guess''}--$f(n)=2^{-+1}-1$:
\begin{table}[H]
	\begin{center}
		\begin{tabular}[H]{|l|r|r|}
			n&$f(n)$&$2^n-1$\\
			1&1&2-1\\
			2&3&4-1\\
			3&7&8-1\\
			4&15&16-1\\
			5&32&32-1\\
			6&63&64-1
		\end{tabular}
	\end{center}
\end{table}

\begin{thm}
	If $f$ satisfies (\ref{eq:hypothesis}) and (\ref{eq:recurrence}), and $n>1$, $f(n)=2^n-1$
\end{thm}

\begin{proof}
	We will use mathematical induction. 
	
	From (\ref{eq:hypothesis}), $f(0)=1$ and (\ref{eq:recurrence}) gives $f(1)= 2f(0)+1 = 2\cdot0+1 =1$. But if $n=1$, $2^n-1=2-1=1$, so the hypothesis is correct for $n=1$.
	
	Assume the hypothesis is correct for a particular $n$. We want to show it is correct for $n+1$, i.e. that $f(n+1)=2^{n+1}-1$.
	\begin{align*}
		f(n)=&2^n-1	\text{ by hypothesis. We use (\ref{eq:recurrence})}\\
		f(n+1)=&2f(n) + 1 \\
		=& 2(2^n-1) + 1\\
		=& 2\cdot 2^n -2 +1\\
		=& 2^{n+1} -1
	\end{align*}
\end{proof}

\begin{cor}
	\begin{align*}
		f(64) =& 2^{65}-1\\
		=& 36,893,488,147,419,103,231
	\end{align*}
\end{cor}

\begin{proof}
	The value was calculated by the following Python code:
	\begin{verbatim}
		N = 1
		for i in range(65):
	 	     N*=2
		print (N-1)
	\end{verbatim}
\end{proof}
% glossary
%\printglossaries

% bibliography go here

%\bibliographystyle{unsrt}
%\addcontentsline{toc}{section}{Bibliography}
%\bibliography{origins,wikipedia}

\end{document}
