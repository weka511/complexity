\documentclass[]{article}
\usepackage{caption,subcaption,graphicx,float,url,amsmath,amssymb,tocloft,wasysym,amsthm,thmtools}
\usepackage[hidelinks]{hyperref}
\usepackage[toc,acronym,nonumberlist]{glossaries}
\setacronymstyle{long-short}
\usepackage{glossaries-extra}
\graphicspath{{figs/}} 
\setlength{\cftsubsecindent}{0em}
\setlength{\cftsecnumwidth}{3em}
\setlength{\cftsubsecnumwidth}{3em}
\newcommand\numberthis{\addtocounter{equation}{1}\tag{\theequation}}
\newtheorem{thm}{Theorem}
\newtheorem{cor}[thm]{Corollary}

%opening
\title{
	Computation in Complex Systems\\
	Peer Review Assignment\\
	Week 1
}

%\makeglossaries


\begin{document}

\maketitle

\tableofcontents

\section{Polynomials \& Exponentials}

\begin{quotation}
	Today, your computer can do $T$ steps in a week. According to Moore's law, next year, your computer will be able to do $2T$ steps in a week. How does doubling $T$ change the $n$ that can be computed in a week? 
\end{quotation}

a) This year $T = n^2$. Next year my (new) computer can do $2T$ steps. Let $n^\prime$ denote the size of problem I can handle next year. Then we have:
\begin{align*}
	T =& n^2 \text{, and}\\
	2T =& (n^\prime)^2 \text{, from Moore's Law, whence}\\
	 (n^\prime)^2=& 2n^2 \text{, or, taking square roots of both sides}\\
	n^\prime=& \sqrt{2} n
\end{align*}
So $n$ changes by a factor of $\sqrt{2}$.

b) This year $T = 2^n$. Next year my (new) computer can do $2T$ steps. Once again let $n^\prime$ denote the size of problem I can handle next year. Then we have:

\begin{align*}
	T =&2^n \text{, and}\\
	2T =& 2^{n^\prime} \text{, from Moore's Law, whence}\\
	2^{n^\prime}=& 2\cdot 2^n\\
	=& 2^{n+1} \text{, so taking logarithms to base 2}\\
	n^\prime =& n+1
\end{align*}

\section{Divide \& Conquer}
Given:
\begin{align*}
	f(1)=&1 \numberthis \label{eq:hypothesis}\\
	f(n)=&2f(n-1)+1 \numberthis \label{eq:recurrence}
\end{align*}
I'll begin by computing the first few values:
\begin{table}[H]
	\begin{center}
		\begin{tabular}{|l|r|} \hline
			n&$f(n)$\\\hline
			1&1\\\hline
			2&3\\\hline
			3&7\\\hline
			4&15\\\hline
			5&32\\\hline
			6&63\\\hline
		\end{tabular}
	\end{center}
\end{table}
We already have enough values to formulate a hypothesis\footnote{this sounds better than ''guess''}--$f(n)=2^{n+1}-1$:
\begin{table}[H]
	\begin{center}
		\begin{tabular}[H]{|l|r|r|}\hline
			n&$f(n)$&$2^n-1$\\\hline
			1&1&2-1\\\hline
			2&3&4-1\\\hline
			3&7&8-1\\\hline
			4&15&16-1\\\hline
			5&32&32-1\\\hline
			6&63&64-1\\\hline
		\end{tabular}
	\end{center}
\end{table}

\begin{thm}
	If $f$ satisfies (\ref{eq:hypothesis}) and (\ref{eq:recurrence}), $f(n)=2^n-1$
\end{thm}

\begin{proof}[Mathematical Induction]
	
	From (\ref{eq:hypothesis}), $f(1)=1$ and (\ref{eq:recurrence}) gives $f(2)= 2f(1)+1 = 2\cdot1+1 =3$. But if $n=2$, $2^n-1=4-1=3$, so the hypothesis is correct for $n=1$.
	
	Assume the hypothesis is correct for a particular $n$. We want to show it is correct for $n+1$, i.e. that $f(n+1)=2^{n+1}-1$.
	\begin{align*}
		f(n)=&2^n-1	\text{ by hypothesis. We use (\ref{eq:recurrence})}\\
		f(n+1)=&2f(n) + 1 \\
		=& 2(2^n-1) + 1\\
		=& 2\cdot 2^n -2 +1\\
		=& 2^{n+1} -1 
	\end{align*}
\end{proof}

Although this proof is valid, it doesn't shed much light on why $f$ should be close to an exact exponential. The following proof is motivated by the fact that \ref{eq:recurrence} is close to an exponential doubling assuming that $f(n)$ is large when $n$ is large. Can we find a quantity that is close to $f(n)$ and which satisfies an exact exponential recurrence relation. It turn out that we can.
\begin{proof}[Alternative]
	We define a new quantity:
	\begin{align*}
		g(n)\triangleq& f(n) +1 \\
		g(n+1) = & f(n+1) +1 \text{, so (\ref{eq:recurrence}) gives:}\\
		=& \big(2f(n)+1\big)+1\\
		=& \big[2\big(g(n) -1\big) +1\big] +1\\
		=& 2 g(n) \text{, so the exponential growth of $g(n)$ is obvious, i.e.} \\
		g(n) =& C\cdot 2^n\text{, for some constant $C>0$. Now (\ref{eq:hypothesis}) becomes:}\\
		g(1) =&2\text{, whence}\\
		C=& 1 \text{, so}\\
		g(n) =& 2^n \text{, and} \\
		f(n) =& 2^n - 1
	\end{align*}
\end{proof}
\begin{cor}
	\begin{align*}
		f(64) =& 2^{64}-1\\
		=& 36,893,488,147,419,103,231
	\end{align*}
\end{cor}

\begin{proof}
	The value was calculated by the following Python code:
	\begin{verbatim}
		N = 1
		for i in range(65):
	 	     N*=2
		print (N-1)
	\end{verbatim}
\end{proof}
% glossary
%\printglossaries

% bibliography go here

%\bibliographystyle{unsrt}
%\addcontentsline{toc}{section}{Bibliography}
%\bibliography{origins,wikipedia}

\end{document}
